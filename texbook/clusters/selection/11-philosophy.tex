\chapter{The Philosophy of Choice}

Di dunia yang ideal, teknologi dipilih berdasarkan kemampuannya untuk memecahkan masalah.
Di dunia nyata, teknologi sering kali dipilih karena \textit{hype}, rasa takut tertinggal (FOMO), atau keinginan untuk mempercantik resume (Resume Driven Development).

Sebagai seorang Artisan, langkah pertama dan terpenting dalam karir Anda bukanlah belajar \textit{cara} menulis kode, melainkan belajar \textit{cara memilih} apa yang akan ditulis.
Ini adalah bab tentang \textbf{Penolakan}.
Seni memilih adalah seni menolak 99\% gangguan yang berkilauan demi 1\% substansi yang abadi.

\section{The Burden of Choice (Beban Pilihan)}

Kita hidup di zaman "Kelimpahan yang Melumpuhkan" (\textit{Paralyzing Abundance}).
Jika Anda ingin membangun aplikasi web hari ini, Anda dihadapkan pada:
\begin{itemize}
    \item 15 Bahasa Pemrograman utama (JS, TS, Python, Go, Rust, dll).
    \item 50 Framework Frontend (React, Vue, Svelte, Solid, Angular, dll).
    \item 20 Jenis Database (SQL, NoSQL, Graph, Vector, Time-series).
    \item 10 Platform Cloud (AWS, GCP, Azure, Vercel, Cloudflare, dll).
\end{itemize}

Kombinasi permutasi dari pilihan ini mencapai jutaan.
Seorang pemula melihat ini dan merasa panik. Mereka mencoba mempelajari semuanya. Mereka menjadi "Tutorial Hell Survivor"—tahu sedikit tentang banyak hal, tapi tidak bisa membangun apa pun sampai selesai.

Seorang Artisan melihat ini dan merasa tenang.
Mengapa? Karena Artisan tahu rahasia yang tidak diketahui pemula: \textbf{Sebagian besar pilihan itu tidak relevan.}
Sebagian besar teknologi diciptakan bukan untuk memecahkan masalah \textit{Anda}, tetapi untuk memecahkan masalah penciptanya (biasanya perusahaan raksasa seperti Google atau Facebook), atau lebih buruk lagi, untuk menjual konferensi dan kursus.

Prinsip pertama Artisan adalah: \textbf{Abaikan Default Industri. Cari Konteks Spesifik.}
Hanya karena Google menggunakan Kubernetes, tidak berarti startup Anda dengan 5 pengguna membutuhkannya. Google memecahkan masalah skala planet. Anda memecahkan masalah validasi ide. Alatnya harus berbeda.

\section{Resume Driven Development (RDD)}

Musuh terbesar dari arsitektur yang sehat adalah ego pengembangnya sendiri.
Kita sering tergoda untuk memilih teknologi yang sedang tren ("Hot new tech") agar kita terlihat relevan di pasar kerja.
"Mari kita tulis ulang backend ini menggunakan Rust dan gRPC!" seru seorang Senior Engineer.
"Kenapa?" tanya manajer.
"Karena... performansinya lebih cepat dan aman memori!" (Alasan sebenarnya: "Karena saya ingin belajar Rust dan menaruhnya di LinkedIn saya").

Ini adalah \textbf{Resume Driven Development (RDD)}.
RDD adalah kanker. Ia menciptakan sistem yang terlalu rumit (\textit{over-engineered}) yang sulit dipelihara setelah pengembang aslinya pergi.
Artisan sejati memiliki keberanian untuk menjadi "Membosankan".

\section{In Praise of Boring Technology}

Dan McKinley, mantan insinyur utama di Etsy, menciptakan istilah \textbf{"Boring Technology"} (Teknologi Membosankan).
Teknologi membosankan adalah teknologi yang:
\begin{enumerate}
    \item Sudah ada setidaknya 10 tahun.
    \item Mode kegagalannya (\textit{failure modes}) sudah dipahami dengan baik.
    \item Memiliki jawaban untuk setiap pertanyaan di StackOverflow.
    \item Stabil dan jarang berubah secara drastis.
\end{enumerate}

Contoh: PostgreSQL, Cron, PHP, Java, Rails, Django.
Sebaliknya, "Teknologi Menarik" adalah teknologi yang:
\begin{enumerate}
    \item Baru rilis versi 0.x atau 1.0.
    \item Dokumentasinya masih berubah-ubah.
    \item Mode kegagalannya belum diketahui (Anda adalah kelinci percobaan).
    \item Anda harus membaca kode sumbernya untuk mengerti cara kerjanya karena tidak ada yang tahu di Google.
\end{enumerate}

Artisan memilih teknologi membosankan untuk \textbf{Infrastruktur Kritis}.
Kita menyimpan data pengguna di PostgreSQL, bukan di database NoSQL eksperimental yang baru rilis minggu lalu. Kenapa? Karena jika database itu rusak, bisnis mati. Kita butuh kepastian, bukan petualangan.

Gunakan teknologi menarik hanya di \textbf{Pinggiran} (\textit{Edges}), di mana kegagalan bisa ditoleransi. Eksperimen dengan framework UI baru di halaman admin internal, bukan di halaman checkout utama.

\section{The Innovation Tokens (Token Inovasi)}

Bayangkan setiap proyek dimulai dengan \textbf{3 Token Inovasi}.
Setiap kali Anda memilih teknologi yang belum pernah Anda gunakan sebelumnya, atau teknologi yang belum stabil, Anda membayar 1 token.
- Menggunakan database baru? -1 Token.
- Menggunakan bahasa baru? -1 Token.
- Menggunakan arsitektur microservices? -1 Token.

Jika Anda kehabisan token (0), Anda tidak boleh lagi memilih teknologi aneh. Anda harus menggunakan pilihan \textit{default} yang membosankan.
Banyak proyek gagal karena mereka mencoba menghabiskan 10 token sekaligus: "Kita akan membangun AI agent (1) menggunakan Rust (2) di atas Kubernetes (3) dengan database Vector baru (4) dan komunikasi GraphQL (5)..."
Proyek seperti ini hampir pasti gagal. Bukan karena idenya buruk, tapi karena \textit{kognitif load}-nya (beban pikiran) terlalu berat. Tim sibuk memadamkan api dari 5 teknologi baru sekaligus, sehingga lupa membangun fitur bisnisnya.

Artisan yang bijak menghabiskan Token Inovasi hanya pada \textbf{Masalah Utama} (\textit{Core Domain}).
Jika Anda membangun startup AI, habiskan token Anda di AI-nya. Gunakan database SQL biasa, server Python biasa, dan frontend standar. Jangan habiskan token Anda untuk mengimprovisasi infrastruktur yang tidak perlu.

\section{The OODA Loop of Selection}

Bagaimana cara Artisan memilih teknologi secara taktis?
Kita menggunakan siklus \textbf{OODA} (\textit{Observe, Orient, Decide, Act}) yang diadaptasi untuk rekayasa:

\textbf{1. Observe (Amati Masalah)}
Jangan mulai dari solusi ("Ayo pakai React!"). Mulai dari masalah.
Apa kendala utamanya?
- Apakah ini masalah \textit{Throughput} (banyak data)?
- Apakah ini masalah \textit{Latency} (harus cepat)?
- Apakah ini masalah \textit{Consistency} (uang tidak boleh hilang)?
- Apakah ini masalah \textit{Time-to-Market} (harus rilis besok)?

\textbf{2. Orient (Orientasi Pilihan)}
Petakan opsi yang tersedia.
Buat daftar kandidat. Baca "Kisah Horor" (\textit{War Stories}) dari orang lain yang menggunakan teknologi tersebut. Jangan hanya membaca testimoni sukses di halaman depan website mereka. Cari artikel dengan judul: "Why we migrated AWAY from X". Itu adalah sumber kebenaran tertinggi.

\textbf{3. Decide (Putuskan dengan Percobaan Kecil)}
Jangan berdebat di ruang rapat selama berbulan-bulan. Lakukan \textbf{Spike Solution}.
Ambil waktu 1-2 hari. Cobalah membangun prototipe kotor (\textit{quick and dirty}) menggunakan teknologi tersebut.
Tujuannya bukan untuk membangun produk, tapi untuk merasakan \textbf{Developer Experience (DX)}.
Apakah instalasinya mudah? Apakah pesan error-nya jelas? Apakah dokumentasinya membantu?
Jika dalam 2 jam Anda masih berkutat dengan konfigurasi, itu tanda bahaya. Buang dan cari yang lain.

\textbf{4. Act (Eksekusi dan Komitmen)}
Setelah memilih, berkomitmenlah.
Jangan melihat ke belakang (\textit{second-guessing}). "Ah, mungkin seharusnya kita pakai itu..."
Setiap teknologi punya kekurangan. Rumput tetangga selalu lebih hijau. Tugas Anda adalah \textbf{Membuatnya Berhasil}.
Dokumentasikan \textit{mengapa} Anda memilihnya (lewat \textit{Architecture Decision Record} - ADR), sehingga orang di masa depan tidak mengutuk Anda.

\section{Kesimpulan: Jadilah Skeptis yang Optimis}

Seorang Artisan adalah seorang \textbf{Skeptis Teknologi}.
Kita tidak mudah terkesima oleh demo "Hello World" yang mulus. Kita tahu bahwa demo selalu menyembunyikan kerumitan asli.
Namun, kita juga seorang \textbf{Optimis Solusi}.
Kita percaya bahwa dengan kombinasi alat yang tepercaya, yang dipilih dengan hati-hati, kita bisa membangun sesuatu yang melampaui zaman.

Pilihlah teknologi seperti Anda memilih pasangan hidup: Bukan yang paling cantik atau paling populer, tetapi yang paling bisa diandalkan saat badai datang, yang paling mengerti bahasa Anda, dan yang bisa tumbuh bersama Anda dalam jangka panjang.
Di bab-bab selanjutnya, kita akan membedah pilihan-pilihan spesifik ini—Bahasa, Database, Jaringan, dan Infrastruktur—dengan pisau bedah filosofi ini.
