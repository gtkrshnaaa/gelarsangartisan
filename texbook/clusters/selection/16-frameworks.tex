\chapter{The Cathedral and the Bazaar (Frameworks)}

Jika bahasa pemrograman adalah batu bata, maka \textbf{Framework} adalah cetak biru rumah yang sudah jadi.
Pertanyaan abadi bagi setiap pengembang:
"Apakah saya harus membangun dari nol, atau menggunakan kerangka kerja yang membatasi tapi mempercepat?"

Artisan 2026 menghadapi dua kubu filosofi:
\begin{enumerate}
    \item \textbf{The Cathedral (Katedral)}: Framework Opiniatif (Opinionated). Segalanya sudah diputuskan. (Rails, Laravel, Django, Spring Boot).
    \item \textbf{The Bazaar (Pasar)}: Micro-framework. Anda merakit sendiri dari komponen kecil. (Express, Flask, Go Chi, FastAPI).
\end{enumerate}

\section{The Cathedral: Baterai Sudah Termasuk}

Rails (Ruby), Laravel (PHP), dan Django (Python) menganut filosofi \textbf{"Convention over Configuration"}.
Mereka berasumsi mereka tahu cara terbaik melakukan sesuatu.
- Struktur folder? Sudah ditentukan.
- ORM Database? Sudah ada.
- Sistem Autentikasi? Tinggal nyalakan.
- Keamanan (CSRF, XSS)? Otomatis.

\textbf{Kapan Memilih Katedral:}
Saat Anda membangun \textbf{Produk Bisnis Standar (CRUD)}.
Toko online, SaaS, Blog, Portal Admin.
Jangan buang waktu Anda merakit sistem login sendiri. Ribuan orang pintar sudah memecahkannya di framework ini. Fokuslah pada fitur unik bisnis Anda.
"Bosan" itu bagus. Produktivitas itu seksi.

\textbf{Bahaya Katedral:}
Magic. Terlalu banyak hal terjadi secara otomatis di balik layar.
Saat Anda ingin melakukan sesuatu yang \textit{tidak} standar, Anda harus bertarung melawan framework. Ini bisa sangat menyakitkan.

\section{The Bazaar: Kebebasan yang Melelahkan}

Express (Node.js), Flask (Python), dan Go net/http.
Mereka memberi Anda router HTTP, dan... itu saja.
Sisanya terserah Anda.
- Mau database apa? Terserah.
- Mau struktur folder gimana? Bebas.
- Mau middleware apa? Cari sendiri di NPM/Pip.

\textbf{Kapan Memilih Pasar:}
Saat Anda membangun \textbf{Microservices Spesifik} atau \textbf{High-Performance API}.
Jika Anda hanya butuh endpoint sederhana yang menerima JSON dan memprosesnya dengan cepat, Katedral terlalu berat.
Pasar memberi Anda kontrol penuh atas setiap byte yang keluar masuk.

\textbf{Bahaya Pasar:}
\textbf{Decision Fatigue (Kelelahan Keputusan)}.
Anda menghabiskan 2 minggu pertama proyek hanya untuk memilih library validasi, memilih ORM, memilih logger, dan menyambungkan semuanya.
Setiap proyek Express terlihat berbeda dari proyek Express lainnya karena tidak ada standar.
Ini disebut \textit{Spaghetti Code} dalam skala arsitektur.

\section{Jebakan Ketergantungan (Framework Lock-in)}

Framework datang dan pergi.
Ingat \textbf{AngularJS 1.x}? \textbf{Backbone.js}? \textbf{Meteor}?
Banyak startup mati karena kode mereka terikat mati dengan framework yang ditinggalkan pengembangnya.
Artisan harus melindungi kode bisnis-nya dari keusangan framework.

Cara terbaik adalah menerapkan prinsip \textbf{Hexagonal Architecture} (Ports and Adapters).
Jangan biarkan logika bisnis Anda (Core Domain) tahu bahwa ia sedang berjalan di dalam Laravel atau Express.
Logika bisnis harus murni (Pure PHP/JS/Python classes).
Framework hanyalah \textbf{Mekanisme Pengiriman} (Delivery Mechanism).
Controller di framework hanya bertugas menerima request HTTP, memanggil Logika Bisnis murni, lalu mengembalikan respons.
Jangan menaruh logika bisnis di Controller!

\section{Kesimpulan: Mulailah dengan Katedral}

Saran untuk Artisan 2026:
Untuk proyek baru, default Anda haruslah \textbf{Katedral (Opinionated Framework)}.
Next.js (React), Laravel, atau Django.
Kecepatan iterasi di tahap awal adalah segalanya.
Katedral memberi Anda kecepatan itu secara gratis.
Hanya ketika Katedral itu mulai terasa sempit (masalah skala ekstrem atau kebutuhan \textit{custom} yang sangat aneh), barulah Anda mempertimbangkan untuk memecah bagian itu menjadi layanan kecil menggunakan Pasar (\textit{Micro-framework}).

Ingat: Pengguna tidak peduli framework apa yang Anda pakai. Mereka peduli apakah aplikasinya bekerja dan bug-nya cepat diperbaiki.
Framework yang bagus adalah framework yang membuat Anda melupakan bahwa Anda sedang menggunakannya.
