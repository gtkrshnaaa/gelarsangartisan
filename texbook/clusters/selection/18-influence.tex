\chapter{The Art of Influence}

Menjadi Artisan yang hebat tidak cukup hanya dengan menulis kode yang brilian.
Jika Anda tidak bisa meyakinkan orang lain untuk menggunakan kode Anda, kode itu akan mati di repositori git yang sepi.
Teknologi adalah masalah manusia.
Sebagai Artisan, tugas Anda bukan hanya membangun solusi, tapi \textbf{Menjual Solusi}.

\section{Leadership Without Authority (Memimpin Tanpa Jabatan)}

Banyak pengembang berpikir: "Saya akan bisa mengubah arah tim kalau saya jadi Tech Lead atau Manager."
Salah.
Pengaruh sejati tidak datang dari jabatan. Pengaruh datang dari \textbf{Kepercayaan} (\textit{Trust Battery}).
Setiap kali Anda:
\begin{itemize}
    \item Memperbaiki bug kritis di tengah malam -> Baterai Kepercayaan naik.
    \item Membantu junior memahami konsep sulit -> Baterai naik.
    \item Mengusulkan teknologi baru yang gagal total -> Baterai turun drastis.
\end{itemize}

Artisan memimpin dengan \textbf{Kompetensi dan Empati}.
Jangan menjadi "Arsitek Menara Gading" yang hanya memberi perintah lewat diagram UML. Turunlah ke parit. Tulis kode bersama tim. Rasakan sakitnya sistem CI/CD yang lambat.
Hanya ketika Anda merasakan sakit mereka, mereka akan mendengarkan solusi Anda.

\section{The Power of the Written Word: RFCs \& Design Docs}

Di Amazon, presentasi PowerPoint dilarang.
Jeff Bezos memaksa eksekutif menulis memo naratif 6 halaman dan membacanya dalam diam di awal rapat.
Mengapa? Karena menulis memaksa Anda berpikir jernih. PowerPoint menyembunyikan ide yang dangkal di balik poin-poin singkat (\textit{bullet points}).

Di Google, Uber, dan perusahaan top lainnya, perubahan teknis besar dimulai dengan \textbf{RFC (Request for Comments)} atau \textbf{Design Doc}.
Dokumen ini berisi:
\begin{enumerate}
    \item \textbf{Context}: Kenapa kita melakukan ini?
    \item \textbf{Proposed Solution}: Apa solusi teknisnya?
    \item \textbf{Alternatives Considered}: Apa opsi lain yang kita tolak, dan kenapa? (Ini bagian terpenting).
    \item \textbf{Risks}: Apa yang bisa salah?
\end{enumerate}

Artisan tidak melempar kode ("PR") tiba-tiba.
Artisan menulis RFC dulu.
"Saya berencana memigrasikan database user ke PostgreSQL. Ini alasannya..."
Biarkan tim berkomentar di dokumen itu. Berdebatlah di atas kertas. Jauh lebih murah memperbaiki kesalahan desain di dokumen Google Docs daripada me-refactor kode yang sudah berjalan di produksi.

\section{Selling Technical Debt Payoff (Menjual Utang Teknis)}

Bisnis tidak peduli tentang "kode yang bersih" atau "refactoring".
Bisnis peduli tentang \textbf{Kecepatan} dan \textbf{Risiko}.
Jangan bilang ke manajer: "Kita perlu refactoring modul Order karena kodenya jelek."
Bilanglah: "Modul Order saat ini sangat rapuh. Jika kita tidak membereskannya sekarang, penambahan fitur Diskon Natal nanti akan memakan waktu 2 minggu, bukan 2 hari, dan berisiko bug ganda."

Gunakan bahasa mereka: \textbf{Risiko Kerugian} vs \textbf{Investasi Kecepatan}.
Technical Debt itu seperti utang kartu kredit. Sedikit oke untuk beli rumah (fitur cepat). Tapi jika tidak dibayar bunganya, Anda bangkrut.

\section{Disagree and Commit}

Di tim yang sehat, konflik itu perlu.
Jika semua orang setuju, berarti tidak ada yang berpikir kritis.
Tapi konflik harus ada akhirnya.
Prinsip \textbf{"Disagree and Commit"} (Tidak setuju tapi berkomitmen) dari Intel/Amazon adalah kunci.
"Saya tidak setuju kita pakai GraphQL, saya lebih suka REST. Tapi karena tim memutuskan GraphQL, saya akan berusaha sekuat tenaga membuat implementasi GraphQL ini sukses."

Jangan menjadi racun yang diam-diam berharap proyek gagal supaya bisa bilang "Tuh kan, saya bilang juga apa!".
Itu bukan Artisan. Itu sabotase.

\section{Mentorship: Warisan Terbesar}

Kode Anda akan dihapus dalam 5-10 tahun.
Sistem yang Anda bangun akan diganti.
Satu-satunya yang abadi adalah \textbf{Orang-orang yang Anda bimbing}.
Junior yang Anda ajari cara debugging yang sabar, cara menulis tes yang baik, cara berpikir sistematis.
Mereka akan menjadi Senior dan Tech Lead di masa depan, membawa filosofi Anda.

Artisan sejati tidak takut digantikan. Artisan sejati \textbf{mencetak penggantinya}.
Karena ketika murid sudah siap, guru bisa move on ke tantangan baru yang lebih besar.

\section{Penutup Bagian 2: Jalan Pedang}

Memilih teknologi, membangun arsitektur, dan memimpin manusia bukanlah ilmu pasti. Itu adalah seni.
Tidak ada jawaban "Benar" atau "Salah". Yang ada hanya \textbf{Trade-offs}.
Setiap keputusan yang Anda buat memiliki harga yang harus dibayar.
Artisan adalah dia yang sadar akan harga itu, dan memilih membayarnya dengan sengaja demi visi yang lebih besar.

Selamat datang di jalan pedang.
Sekarang, mari kita lihat bagaimana menghidupi jalan ini sehari-hari di Bagian 3: \textbf{Living the Tech}.
