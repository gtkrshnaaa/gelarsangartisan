\chapter{The Ground We Walk On (Infrastructure)}

Infrastruktur adalah kanvas tempat kita melukis kode.
Tanpa infrastruktur yang stabil, kode terbaik pun tidak ada gunanya. "It works on my machine" adalah kalimat paling mahal dalam sejarah IT.
Tujuan Artisan bukan untuk menjadi SysAdmin yang begadang menjaga server, tapi untuk membangun \textbf{Fondasi Otomatis} yang bisa menyembuhkan diri sendiri.

\section{Evolusi Abstraksi: Dari Logam ke Udara}

Sejarah infrastruktur adalah sejarah \textbf{menjauh dari perangkat keras}.

\subsection{1. Bare Metal (Zaman Batu)}
Anda beli server fisik, pasang di rak, instal OS, dan berdoa AC ruangan tidak mati.
\textbf{Kelebihan:} Performa maksimal. Tidak ada tetangga berisik (\textit{Noisy Neighbors}).
\textbf{Kekurangan:} Mahal, lambat disiapkan (berminggu-minggu), dan Anda harus mengurus harddisk rusak sendiri.
\textbf{Vonis Artisan:} Hanya untuk raksasa (Facebook/Google) atau kebutuhan khusus (High Frequency Trading). Jangan lakukan ini di 2026.

\subsection{2. Virtual Machines (Zaman Perunggu)}
EC2, DigitalOcean Droplets.
Satu server fisik dibagi menjadi banyak server virtual.
\textbf{Kelebihan:} Cepat (menit). Isolasi cukup baik.
\textbf{Kekurangan:} Anda masih harus mengurus OS patching, security updates, dan konfigurasi server "Pet" (hewan peliharaan yang harus dirawat).

\subsection{3. Containers (Zaman Besi)}
Docker, Kubernetes.
Kita membungkus aplikasi + semua dependensinya ke dalam satu kotak standar.
\textbf{Kelebihan:} \textbf{Portabilitas Ekstrem}. Jalan di laptop = Jalan di server. "It works on my machine" mati di sini. Efisiensi sumber daya sangat tinggi.
\textbf{Kekurangan:} Kompleksitas orkestrasi (Kubernetes itu sulit).

\subsection{4. Serverless (Zaman Awan)}
AWS Lambda, Vercel, Cloudflare Workers.
Tidak ada server. Anda hanya mengupload fungsi kode. Cloud provider yang menjalankannya saat ada permintaan.
\textbf{Kelebihan:} Skala nol hingga tak terhingga secara instan. Bayar per milidetik. Nol maintenance OS.
\textbf{Kekurangan:} \textbf{Cold Start} (lambat saat pertama dipanggil). Vendor Lock-in (kode Lambda sulit dipindah ke Google Cloud Functions tanpa ubahan). Mahal pada skala trafik tinggi.

\section{The Cost of Abstraction (Biaya Kenyamanan)}

Hukum kekekalan kerumitan (\textit{Conservation of Complexity}) berlaku:
"Kerumitan tidak hilang, ia hanya bergeser."
Di Serverless, kerumitan operasional hilang, tapi kerumitan debugging dan biaya bertambah.

Artisan harus menghitung \textbf{Total Cost of Ownership (TCO)}.
Serverless sangat murah untuk startup kecil (Hampir gratis).
Tapi saat trafik naik, tagihan Serverless bisa meledak.
Sering kali, menyewa satu VPS gemuk seharga \$20/bulan bisa menangani trafik yang sama dengan tagihan Lambda \$500/bulan.
Jangan buta karena kemudahan. Hitunglah.

\section{Infrastructure as Code (IaC): Server sebagai Ternak}

Jangan pernah mengkonfigurasi server produksi secara manual (SSH -> `apt-get install nginx`).
Jika server itu meledak, Anda tidak bisa membuatnya lagi dengan persis sama. Anda lupa langkah-langkahnya.
Gunakan \textbf{Infrastructure as Code} (Terraform, Pulumi, Ansible).
Definisikan infrastruktur Anda dalam file teks:
\texttt{resource "aws\_instance" "web" \{ ... \}}

Dengan IaC:
1.  Infrastruktur terversioning di Git.
2.  Anda bisa menduplikasi lingkungan (Staging, Prod) dalam hitungan detik.
3.  Server menjadi \textbf{Immutable} (Tak Berubah). Jika rusak, hancurkan dan buat baru dari cetakan kode. Server adalah Ternak (\textit{Cattle}), bukan Hewan Peliharaan (\textit{Pets}).

\section{Exit Strategy: Menghindari Penjara Vendor}

Setiap Cloud Provider (AWS, GCP, Azure) ingin mengunci Anda. Mereka memberi fitur-fitur canggih (DynamoDB, BigQuery) yang tidak ada di tempat lain.
Jika Anda memakai fitur \textit{proprietary} ini, Anda tidak bisa pindah.
Apakah itu buruk? Tidak selalu. Kadang kecepatan pengembangan sepadan dengan kuncian itu.

Tapi Artisan yang bijak selalu punya \textbf{Rencana Keluar (Exit Strategy)}.
Strategi keluar terbaik adalah \textbf{Container (Docker)}.
Selama aplikasi Anda dibungkus dalam Docker, Anda bisa memindahkannya dari AWS ECS ke Google Cloud Run atau bahkan ke server besi tua di basement kantor dalam hitungan jam.
Standarisasi pada Container adalah polis asuransi kebebasan Anda.

\section{Rekomendasi Jalur Artisan}

Untuk proyek baru di 2026:
\begin{enumerate}
    \item \textbf{Mulai dengan PaaS (Platform as a Service)}: Vercel, Railway, Render.
    Fokus pada kode produk. Biarkan mereka mengurus SSL, Deploy, dan Scaling. Mahal sedikit tidak apa-apa karena waktu Anda lebih berharga.
    \item \textbf{Tumbuh ke Container (CaaS)}: Saat tagihan PaaS mulai tidak masuk akal (biasanya >\$500/bulan), bungkus aplikasi ke Docker dan pindah ke layanan Container terkelola (AWS Fargate / Google Cloud Run).
    \item \textbf{Hindari Kubernetes Sendiri}: Kecuali Anda punya tim DevOps khusus, jangan mengelola kluster Kubernetes sendiri. Itu adalah lubang hitam waktu.
\end{enumerate}

Infrastruktur terbaik adalah infrastruktur yang tidak perlu Anda pikirkan. Ia harus "membosankan", tidak terlihat, dan sekokoh tanah yang kita pijak.
