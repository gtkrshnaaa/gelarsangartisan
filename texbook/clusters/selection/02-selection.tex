\chapter{The Artisan's Choice}

\chapter{The Artisan's Choice}

Dunia IT hari ini adalah sebuah rimba raya pilihan. Setiap minggu ada \textit{framework} baru, setiap bulan ada bahasa baru yang mengklaim lebih cepat. Sebagai seorang \textit{artisan}, tugas kita bukan untuk mempelajari semuanya, tapi memilih yang paling tepat.

\section{Filosofi Pemilihan Stack}

Saya memiliki tiga pilar utama saat harus memutuskan teknologi apa yang akan saya gunakan dalam sebuah proyek:

\begin{enumerate}
    \item \textbf{Purpose (Tujuan)}: Apakah teknologi ini menyelesaikan masalah utamanya, atau ia justru menambah kompleksitas baru? Jangan gunakan \textit{microservices} jika \textit{monolith} sudah lebih dari cukup.
    \item \textbf{Craftsmanship (Kualitas Rancang Bangun)}: Seberapa indah kodenya? Seberapa konsisten dokumentasinya? Teknologi yang baik adalah teknologi yang membuat pengembangnya merasa bangga saat menulisnya.
    \item \textbf{Longevity (Keberlanjutan)}: Apakah teknologi ini akan tetap ada dalam 5 tahun ke depan? Sebagai artisan, kita membangun untuk jangka panjang.
\end{enumerate}

\section{The Artisan Framework}

Saat menghadapi kasus spesifik, saya menggunakan matriks sederhana:

\vspace{0.5cm}
\begin{center}
\begin{tabular}{|l|l|p{6cm}|}
    \hline
    \textbf{Kebutuhan} & \textbf{Prioritas} & \textbf{Rekomendasi Pendekatan} \\
    \hline
    Eksperimen Cepat & Kecepatan Iterasi & Gunakan \textit{batteries-included framework} seperti Django atau Next.js. \\
    \hline
    Sistem Kritis & Keamanan \& Performa & Kembalilah ke bahasa yang lebih rendah seperti Rust atau C++. \\
    \hline
    Skalabilitas Tinggi & Konkuransi & Cari teknologi yang menangani \textit{async} dengan elegan seperti Go atau Elixir. \\
    \hline
\end{tabular}
\end{center}
\vspace{0.5cm}

\section{Studi Kasus: Memilih Stack untuk 2026}

Di tahun 2026, kita tidak lagi hanya memilih bahasa pemrograman. Kita memilih \textit{ecosystem synergy}. \textit{Pada saat buku ini dibuat}, alat-alat seperti \textit{Agentic AI} memiliki kemampuan untuk meracik dependensi secara otomatis, namun seorang artisan harus tetap memegang kendali atas hasil akhirnya. Integrasi antara \textit{AI-native tools} dan \textit{runtime} yang hemat energi menjadi prioritas utama saya saat ini.
