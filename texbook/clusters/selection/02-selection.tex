\chapter{The Artisan's Choice}

Dunia IT adalah sebuah rimba raya pilihan yang sengaja dibuat untuk membingungkan mereka yang tidak memiliki pijakan. Sebagai seorang \textit{artisan}, kita tidak memilih teknologi untuk sekadar menjadi pengguna. Kita memilih untuk memahami cara kerja sistem agar kita bisa menuntunnya sesuai dengan visi kita.

Banyak orang terjebak dalam perdebatan \textit{framework} mana yang terbaik. Seorang artisan melihat melampaui itu. Kita memilih alat bukan karena popularitasnya, tapi karena bagaimana alat tersebut memungkinkan kita untuk tetap memegang kendali atas narasi teknis dalam sebuah proyek.

\section{Filosofi Pemilihan Stack}

Saya memiliki tiga pilar utama saat harus memutuskan teknologi apa yang akan saya gunakan dalam sebuah proyek:

\begin{enumerate}
    \item \textbf{Purpose (Tujuan)}: Apakah teknologi ini menyelesaikan masalah utamanya, atau ia justru menambah kompleksitas baru?
    \item \textbf{Influence (Pengaruh)}: Seberapa besar kendali yang diberikan teknologi ini kepada pengembangnya untuk mengarahkan arah sistem?
    \item \textbf{Longevity (Keberlanjutan)}: Apakah teknologi ini akan tetap ada dalam 5 tahun ke depan?
\end{enumerate}

\section{Studi Kasus: Memilih Stack untuk 2026}

Di tahun 2026, kita tidak lagi hanya memilih bahasa pemrograman. Kita memilih \textit{ecosystem synergy}. \textit{Pada saat buku ini dibuat}, alat-alat seperti \textit{Agentic AI} memiliki kemampuan untuk meracik dependensi secara otomatis, namun seorang artisan harus tetap memegang kendali atas hasil akhirnya. Integrasi antara \textit{AI-native tools} dan \textit{runtime} yang hemat energi menjadi prioritas utama saya saat ini.
