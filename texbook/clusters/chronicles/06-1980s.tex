\chapter{The Era of Interfaces (1980 -- 1989)}

Pada tahun 1970-an, kita belajar bagaimana membuat komputer menjadi \textit{personal}. Kita belajar bahwa kekuatan silikon bisa diletakkan di atas meja kita, di bawah kendali jari-jari kita sendiri.
Namun, pada awal 1980-an, kita menghadapi masalah baru yang jauh lebih rumit: \textbf{Isolasi} dan \textbf{Kompleksitas}.

Komputer-komputer pribadi itu seperti pulau-pulau yang terisolasi. Mereka memiliki bahasa yang berbeda, protokol yang berbeda, dan sistem file yang tidak kompatibel.
Di sisi lain, perangkat lunak menjadi semakin besar dan rumit. Menulis sistem operasi dengan bahasa prosedural (seperti C) mulai terasa seperti membangun gedung pencakar langit dengan tumpukan batu bata tanpa semen yang kuat. Satu bata salah, seluruh gedung runtuh.

Dekade 1980-an adalah dekade \textbf{Antarmuka} (\textit{Interfaces}).
Dunia membutuhkan cara standar agar manusia bisa berbicara dengan mesin (GUI), agar mesin bisa berbicara dengan mesin (TCP/IP), dan agar programmer bisa berbicara dengan kompleksitas (OOP).

Bagi seorang Artisan di tahun 2026, era ini mengajarkan tentang \textbf{Abstraksi yang Elegan}.
Kita belajar bahwa untuk menangani sistem yang besar, kita harus membungkus kerumitan di dalam kotak hitam (Objek/Paket/Jendela) dan hanya mengekspos tombol-tombol yang diperlukan keluar. Inilah seni menyembunyikan detail demi kewarasan mental.

\section{1980 -- 1983: Menyatukan Bahasa yang Terpecah}

Dunia jaringan sebelum 1983 adalah Menara Babel. Departemen Pertahanan AS memiliki ARPANET. Universitas memiliki CSNET. Perusahaan memiliki jaringan proprieter mereka sendiri (seperti DECNET atau SNA milik IBM). Mereka tidak bisa saling bicara. Data terperangkap dalam silo masing-masing.

Vint Cerf dan Bob Kahn memiliki visi gila: "Jaringan dari jaringan" (\textit{Inter-net}).
Pada 1 Januari 1983, ARPANET secara resmi beralih menggunakan protokol baru mereka: \textbf{TCP/IP} (\textit{Transmission Control Protocol/Internet Protocol}).

Ini adalah momen "Big Bang" infrastruktur digital.
Kejeniusan TCP/IP terletak pada desainnya yang agnostik. Ia tidak peduli data apa yang dibawanya (email, gambar, suara) dan ia tidak peduli lewat media apa ia dikirim (kabel tembaga, serat optik, radio, atau bahkan merpati pos).
Filosofinya adalah: \textbf{Jaringan itu Bodoh, Ujungnya yang Pintar} (\textit{End-to-End Principle}).
Jaringan hanya bertugas mengantarkan paket. Intelegensia untuk merakit kembali paket berada di komputer pengirim dan penerima.
Keputusan desain inilah yang memungkinkan internet bertahan hingga hari ini. Ia bisa menelan teknologi baru (seperti streaming video 4K atau panggilan Zoom) tanpa perlu mengubah infrastruktur intinya. Bagi Artisan, ini adalah pelajaran tertinggi dalam \textbf{Desain Skalabilitas}.

Namun, di saat dunia jaringan mulai bersatu, dunia perangkat lunak mulai tertutup.
Perusahaan-perusahaan mulai menyadari bahwa kode adalah aset berharga. Mereka mulai menguncinya dengan lisensi, merahasiakan kode sumber, dan melarang pengguna untuk memodifikasinya. Budaya berbagi kode ala hacker tahun 70-an mulai mati.

Seorang programmer di MIT bernama \textbf{Richard Stallman} (RMS) merasa tercekik.
Ketika ia tidak bisa memperbaiki driver printer yang macet karena kodenya tertutup, ia meledak. Ia melihat ini bukan sebagai masalah teknis, tapi masalah moral. "Perangkat lunak yang mengontrol hidup kita harus transparan bagi kita."
Pada 27 September 1983, ia mengumumkan proyek \textbf{GNU} (\textit{GNU's Not Unix}).
Tujuannya: Membuat sistem operasi lengkap yang 100\% bebas (\textit{Free Software}). Bebas bukan berarti gratis harga (\textit{free beer}), tapi bebas kebebasan (\textit{free speech}).
Stallman menciptakan lisensi \textbf{GPL} (\textit{General Public License}) yang revolusioner: "Anda boleh menggunakan kode ini, memodifikasinya, dan menjualnya. Tapi jika Anda mendistribusikannya, Anda harus memberikan kode sumbernya juga kepada penerima."
Ini adalah \textbf{Copyleft}. Ia menggunakan hukum hak cipta untuk menjamin kebebasan, bukan untuk membatasinya. Tanpa langkah radikal RMS di tahun 1983 ini, kita tidak akan pernah memiliki Linux, Git, atau ekosistem Open Source modern.

\section{1984 -- 1985: Revolusi Otak Kanan}

Selama 40 tahun, komputer adalah alat untuk "Otak Kiri": Logika, Angka, Teks, Baris Perintah.
Jika Anda ingin menyalin file, Anda mengetik: \texttt{cp file.txt /destination}.
Itu efisien, tapi dingin. Itu menuntut hafalan, bukan intuisi.

Pada Januari 1984, Steve Jobs dan Apple memperkenalkan \textbf{Macintosh}.
Dalam iklan Super Bowl "1984" yang disutradarai Ridley Scott, mereka menjanjikan pembebasan dari tirani keseragaman (yang disimbolkan oleh IBM).
Macintosh berbeda.
Ia memiliki \textbf{Mouse}. Ia memiliki \textbf{Jendela}. Ia memiliki \textbf{Ikon} tempat sampah. Ia memiliki \textbf{Font} yang indah (Chicago, Geneva, Monaco).
Tiba-tiba, komputer bisa digunakan oleh "Otak Kanan": Seniman, Musisi, Penulis. Anda bisa \textit{merasakan} data Anda. Anda bisa menyeret (\textit{drag}) dokumen ke folder. Itu spasial. Itu manusiawi.

Namun, di balik layar, membuat GUI itu jauh lebih sulit daripada CLI.
Dalam CLI, program mengontrol alur: "Tanya nama -> Tunggu input -> Cetak halo".
Dalam GUI, pengguna yang mengontrol alur: "Pengguna bisa mengklik menu A, atau menggeser jendela B, atau menekan tombol C kapan saja." Program harus siap bereaksi terhadap \textit{event} apa saja.

Kompleksitas kode meledak. Struktur C prosedural menjadi berantakan ("Spaghetti Code") untuk menangani ribuan state tombol dan jendela.
Dunia membutuhkan cara baru untuk mengorganisir kode.
Pada tahun 1985, Bjarne Stroustrup di Bell Labs merilis \textbf{C++}.
Ia mengambil efisiensi bahasa C dan menambahkan konsep \textbf{Kelas} (\textit{Classes}) dari Simula.
Ini adalah kelahiran \textbf{Object-Oriented Programming (OOP)} di arus utama.
Dengan OOP, Artisan tidak lagi berpikir tentang "fungsi yang mengubah variabel global". Kita berpikir tentang \textbf{Objek}.
Jendela adalah Objek. Tombol adalah Objek. Menu adalah Objek. Objek memiliki data sendiri (properti) dan perilaku sendiri (metode). Mereka saling berkirim pesan.
C++ memungkinkan kita membangun sistem operasi GUI yang sangat kompleks (seperti Windows dan macOS) tanpa kehilangan kewarasan. Ini adalah alat manajemen kompleksitas terbaik pada masanya.

Pada November 1985, Microsoft merilis \textbf{Windows 1.0}.
Awalnya, itu gagal. Lambat. Jelek. Sedikit aplikasi.
Apple menertawakannya. Tapi Bill Gates memiliki senjata rahasia: \textbf{Kesabaran Ekosistem}.
Ia melisensikan Windows ke setiap pembuat PC di dunia. Ia memberi alat pengembangan ke ribuan programmer. Ia tahu bahwa dalam jangka panjang, platform dengan aplikasi terbanyaklah yang akan menang, bukan platform yang paling elegan.
Strategi ini—\textit{Worse is Better} jika distribusinya lebih luas—adalah pelajaran brutal tapi penting bagi setiap idealis teknologi.

\section{1986 -- 1988: Standar Data dan Hilangnya Kepolosan}

Saat komputer semakin terhubung, data menjadi mata uang baru.
Pada tahun 1986, \textbf{SQL} (\textit{Structured Query Language}) diadopsi sebagai standar ANSI.
Sebelumnya, setiap database punya bahasanya sendiri. Sekarang, Artisan di seluruh dunia bisa berbicara bahasa yang sama untuk bertanya pada data: \texttt{SELECT * FROM users WHERE active = true}.
Standarisasi ini memungkinkan ledakan industri perangkat lunak perusahaan (\textit{Enterprise Software}). Oracle, IBM, dan Microsoft berlomba membuat mesin database terbaik, tetapi bahasanya tetap sama.

Namun, konektivitas yang semakin luas membawa konsekuensi gelap.
Pada 2 November 1988, Robert Tappan Morris, seorang mahasiswa pascasarjana di Cornell, melepaskan sebuah program eksperimental. Ia ingin mengukur seberapa besar internet itu.
Program itu dirancang untuk menyalin dirinya sendiri dari satu mesin Unix ke mesin lain, memanfaatkan celah keamanan di \texttt{sendmail} dan \texttt{finger}.
Tapi Morris membuat kesalahan logika fatal: Program itu menggandakan diri terlalu cepat, bahkan menginfeksi mesin yang sudah terinfeksi berkali-kali.

Dalam hitungan jam, 10\% dari seluruh internet (sekitar 6.000 komputer) lumpuh. Server-server di MIT, Pentagon, dan NASA macet terbebani proses virus tersebut.
Ini dikenal sebagai \textbf{Morris Worm}.
Hari itu, "Arsitektur Kepercayaan" internet runtuh. Sebelumnya, internet dijalankan oleh para akademisi yang saling percaya. Administrator sistem saling berbagi akses root. Setelah Morris Worm, tembok api (\textit{Firewalls}) didirikan. Keamanan siber (\textit{Cybersecurity}) lahir sebagai disiplin ilmu pertahanan hidup.
Kita belajar bahwa setiap koneksi adalah potensi serangan.

\section{1989: Proposal yang Mengubah Peradaban}

Dekade ini ditutup dengan kesunyian di sebuah koridor di CERN, Swiss.
\textbf{Tim Berners-Lee}, seorang fisikawan Inggris, frustrasi. CERN memiliki ribuan peneliti dengan ribuan dokumen yang tersimpan di komputer yang berbeda-beda. Tidak ada cara mudah untuk menautkan satu dokumen ke dokumen lain di komputer yang berbeda.

Pada Maret 1989, ia mengajukan proposal berjudul \textit{"Information Management: A Proposal"}.
Atasannya, Mike Sendall, menulis catatan kecil di sampulnya: \textit{"Vague but exciting"} (Samar tapi menarik). Ia memberi Tim waktu untuk mengerjakannya.

Tim tidak menemukan Internet (itu sudah ada berkat TCP/IP).
Tim tidak menemukan Hypertext (konsep itu sudah ada sejak Engelbart dan Ted Nelson).
Kejeniusan Tim adalah \textbf{Menggabungkan Keduanya}.

Dia menciptakan tiga teknologi sekaligus:
1.  \textbf{HTML} (\textit{HyperText Markup Language}): Format sederhana untuk menulis dokumen berantai.
2.  \textbf{HTTP} (\textit{HyperText Transfer Protocol}): Cara sederhana untuk meminta dokumen tersebut.
3.  \textbf{URL} (\textit{Uniform Resource Locator}): Alamat unik untuk setiap dokumen di dunia.

Ia menyebut sistem ini \textbf{World Wide Web}.
Dan keputusan terbesarnya bukanlah pada kodenya, tapi pada filosofinya: Ia membuat Web itu \textbf{Permissionless} (Tanpa Izin).
Siapa pun bisa membuat tautan ke halaman siapa pun tanpa perlu meminta izin. Tautan bisa saja putus (\textit{Error 404}). Itu tidak masalah. Web tidak harus sempurna; ia harus mudah tumbuh.

Web adalah antarmuka pamungkas.
Ia membungkus kerumitan TCP/IP, server, dan database di balik satu konsep sederhana: \textbf{Klik Tautan Biru}.
Dengan ini, internet bukan lagi sekadar milik ilmuwan komputer. Ia siap menjadi milik seluruh umat manusia.

Di penghujung dekade, Nintendo merilis \textbf{Game Boy} (1989).
Para pesaingnya (Atari Lynx, Sega Game Gear) memiliki layar berwarna dan lampu latar. Game Boy hanya hitam-putih (hijau-hitam, tepatnya) dan tanpa lampu.
Tapi Game Boy menang telak. Mengapa?
Karena baterainya tahan 30 jam (lawan 3 jam) dan ia muat di saku.
Gunpei Yokoi, perancangnya, mengajarkan filosofi \textbf{"Lateral Thinking with Withered Technology"}. Gunakan teknologi lama yang sudah murah dan matang, tapi aplikasikan dengan cara baru yang kreatif. Jangan terobsesi dengan spesifikasi tertinggi; terobsesilah dengan pengalaman pengguna dan konteks penggunaan.

\section{Refleksi Dekade: Kemenangan Struktur}

Jika kita melihat kembali tahun 80-an, kita melihat dekade di mana kita "Merestrukturisasi Kekacauan".
Kita membangun struktur visual (GUI) di atas baris perintah.
Kita membangun struktur objek (OOP/C++) di atas kode prosedural.
Kita membangun struktur jaringan (TCP/IP) di atas kabel-kabel terpisah.
Dan kita membangun struktur informasi (WWW) di atas tumpukan file.

Para Artisan tahun 80-an memberikan kita \textbf{Alat untuk Bermimpi Besar}.
Tanpa abstraksi-abstraksi ini, kita tidak akan pernah sanggup membangun sistem global seperti Google atau Facebook (yang terdiri dari miliaran baris kode). Otak manusia memiliki batas kognitif, dan tahun 80-an memberikan kita cara untuk melampaui batas itu melalui organisasi yang lebih baik.

Warisan mereka adalah pesan: \textbf{Rapikan Imajinasimu}.
Jangan hanya menulis kode yang bekerja; tulislah kode yang terstruktur, yang bisa dibaca, yang bisa digunakan kembali, dan yang bisa terhubung dengan dunia.
Kemenangan 1980-an adalah kemenangan Arsitektur di atas sekadar Konstruksi.
