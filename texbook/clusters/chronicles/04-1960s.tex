\chapter{The Era of Interactivity (1960 -- 1969)}

Jika dekade 1950-an adalah tentang bagaimana kita belajar berbicara "bahasa" mesin melalui Compiler dan Assembly, maka dekade 1960-an adalah tentang bagaimana kita mulai mengubah "sifat" percakapan tersebut.

Hingga akhir tahun 1959, hubungan antara manusia dan komputer masih bersifat \textit{Batch Processing}. Bayangkan ini seperti mengirim surat: Anda menulis kode di atas kertas, memberikannya kepada operator (Sang Imam Besar), dan menunggu berjam-jam atau berhari-hari untuk mendapatkan balasan. Tidak ada dialog. Tidak ada interaksi. Anda tidak bisa meralat kesalahan saat itu juga. Mesin adalah entitas yang dingin, jauh, dan agung, tersembunyi di balik dinding kaca ber-AC yang steril.

Namun, di awal dekade ini, sekelompok visioner—para Artisan awal yang menolak status quo—mulai membayangkan sesuatu yang radikal. Bagaimana jika kita tidak perlu menunggu? Bagaimana jika kita bisa mengetikkan perintah, dan mesin menjawab \textit{saat itu juga}? Bagaimana jika komputer bukan sekadar kalkulator raksasa, tetapi alat untuk memperluas kemampuan berpikir manusia?

Dekade 1960-an adalah dekade "Pemberontakan Interaktif". Ini adalah saat di mana \textit{Time-Sharing} menghancurkan monopoli waktu mainframe. Ini adalah saat di mana \textit{Minicomputer} membawa mesin keluar dari katedral korporat. Ini adalah dekade di mana J.C.R. Licklider memimpikan \textit{Man-Computer Symbiosis}, dan Douglas Engelbart menunjukkan kepada dunia bagaimana wujud masa depan itu. Dan di penghujung dekade, di tengah Perang Dingin yang memanas, benih internet (ARPANET) dan sistem operasi modern (UNIX) ditanam.

Bagi seorang Artisan di tahun 2026, era ini mengajarkan satu prinsip fundamental: \textbf{Responsivitas}. Alat yang baik adalah alat yang memberikan umpan balik instan. Tanpa latensi tahun 60-an yang memaksa lahirnya interaktivitas, kita tidak akan pernah memiliki \textit{REPL}, \textit{IntelliSense}, atau \textit{Generative AI} yang kita nikmati hari ini.

\section{1960 -- 1962: Visi Simbiosis dan Hacking Pertama}

Segalanya dimulai bukan dengan sebuah chip, tetapi dengan sebuah ide. Pada tahun 1960, J.C.R. Licklider, seorang psikolog yang beralih menjadi ilmuwan komputer di MIT dan kemudian ARPA, menerbitkan sebuah makalah mani berjudul \textit{"Man-Computer Symbiosis"}.

Dalam makalah tersebut, Licklider menuliskan visi yang melampaui zamannya:
\begin{quote}
"Harapan saya adalah bahwa dalam waktu yang tidak terlalu lama, otak manusia dan mesin komputasi akan digabungkan bersama dengan sangat erat, dan mesin yang dihasilkan akan berpikir sebagaimana tidak pernah dipikirkan oleh otak manusia manapun dan memproses data dengan cara yang tidak pernah didekati oleh mesin pemrosesan informasi yang kita kenal sekarang."
\end{quote}

Licklider tidak melihat komputer sebagai pengganti manusia, melainkan sebagai \textit{mitra}. Ia membayangkan sebuah masa depan di mana mesin menangani tugas-tugas rutin yang membosankan (kalkulasi, pencarian data), sementara manusia menangani wawasan, intuisi, dan pengambilan keputusan. Visi inilah yang memicu pendanaan masif dari ARPA (Advanced Research Projects Agency) untuk proyek-proyek yang berfokus pada interaksi manusia-komputer, bukan sekadar kecepatan hitung.

Manifestasi fisik pertama dari "interaksi" ini datang bukan dari IBM yang raksasa, tetapi dari perusahaan kecil bernama DEC (Digital Equipment Corporation). Pada tahun 1960, mereka merilis \textbf{PDP-1} (\textit{Programmed Data Processor-1}).

Berbeda dengan mainframe IBM 7090 yang membutuhkan satu ruangan penuh dan biaya jutaan dolar, PDP-1 "hanya" seukuran tiga lemari es dan berharga \$120.000 (murah untuk ukuran masa itu). Tapi yang paling penting: ia dilengkapi dengan \textit{keyboard} dan layar \textit{CRT} (Cathode Ray Tube). Anda bisa duduk di depannya, menyalakannya, dan langsung mengetik.

Di MIT, sekelompok mahasiswa yang menyebut diri mereka sebagai "hackers" (dalam arti positif: pengulik yang antusias) jatuh cinta pada mesin ini. Mereka tidak menggunakan PDP-1 untuk menghitung lintasan rudal atau sensus penduduk. Mereka menggunakannya untuk bersenang-senang.

Pada tahun 1962, Steve Russell, Martin Graetz, dan Wayne Wiitanen menciptakan \textbf{Spacewar!}.
Ini adalah video game digital pertama yang sesungguhnya. Dua pesawat ruang angkasa ("The Wedge" dan "The Needle") saling menembak di layar CRT vektor, dipengaruhi oleh gravitasi bintang pusat.

Bayangkan betapa revolusionernya ini: Komputer, alat militer yang sangat serius, digunakan untuk \textit{bermain}.
Di balik layar, kode \textit{Spacewar!} adalah pengajaran agung tentang optimasi. Russell dan timnya harus menulis subrutin sinus/kosinus yang sangat efisien agar gerakan pesawat terasa mulus secara \textit{real-time}. Mereka meretas perangkat keras untuk mendapatkan performa maksimal.

\textit{Spacewar!} menyebar ke setiap instalasi PDP-1 di seluruh Amerika. Ia membuktikan bahwa komputer bisa menjadi media ekspresi yang menyenangkan. Bagi Artisan, ini adalah momen kelahiran \textit{Hacker Culture}: semangat untuk mengeksplorasi batas kemampuan mesin demi kesenangan murni penciptaan, bukan sekadar utilitas bisnis.

\section{1963 -- 1965: Memecah Waktu dan Menaklukkan Kompleksitas}

Seiring dengan meningkatnya permintaan akan akses komputer, model "satu orang, satu mesin" (seperti pada PDP-1) menjadi tidak ekonomis, sementara model "antrian batch" (seperti pada IBM) terlalu lambat untuk inovasi. Dunia membutuhkan jalan tengah.

Jawabannya adalah \textbf{Time-Sharing}.
Konsep ini dikembangkan secara serius di MIT melalui proyek \textbf{CTSS} (\textit{Compatible Time-Sharing System}) yang dipimpin oleh Fernando Corbató. Idenya brilian: Komputer sangat cepat, sedangkan manusia sangat lambat (mengetik mungkin hanya 2 karakter per detik). Di antara jeda ketukan tombol manusia, prosesor komputer sebenarnya "menganggur" selama jutaan siklus.

Mengapa tidak memanfaatkan waktu nganggur itu untuk melayani orang lain?
Dalam sistem \textit{Time-Sharing}, sebuah komputer melayani puluhan pengguna secara bergantian dengan sangat cepat. Setiap pengguna merasa seolah-olah mereka memiliki mesin itu untuk diri mereka sendiri, padahal mesin sedang melompat (\textit{context switching}) dari satu terminal ke terminal lain dalam hitungan milidetik.

Untuk memungkinkan ini, kita membutuhkan evolusi besar dalam perangkat lunak dan perangkat keras. Kita membutuhkan \textbf{Multiprogramming} (menjalankan banyak program di memori sekaligus) dan \textbf{Memory Protection} (mencegah program User A menimpa memori User B).

Pada tahun 1964, IBM, raksasa yang awalnya tidur, terbangun. Mereka melakukan taruhan terbesar dalam sejarah bisnis korporat (\$5 Miliar kala itu) untuk menciptakan \textbf{IBM System/360}.
Sebelum 360, setiap model komputer IBM memiliki set instruksi yang berbeda. Jika Anda mengganti mesin lama dengan yang baru, Anda harus menulis ulang semua kode Anda.
System/360 mengubah segalanya. Ia memperkenalkan konsep \textbf{Arsitektur Set Instruksi (ISA)} yang kompatibel ke belakang. Kode yang ditulis untuk Model 30 yang kecil bisa berjalan tanpa perubahan di Model 75 yang raksasa.

Ini adalah kelahiran konsep "Kompatibilitas Software".
Fred Brooks, manajer proyek System/360, kemudian menulis buku legendaris \textit{"The Mythical Man-Month"} berdasarkan pengalamannya mengelola proyek raksasa ini (dan sistem operasinya yang terkenal rumit, OS/360). Ia mengajarkan pelajaran abadi bagi setiap Artisan manajer proyek: "Menambahkan tenaga kerja ke proyek perangkat lunak yang terlambat hanya akan membuatnya semakin terlambat."

Sementara dunia korporat sibuk dengan 360, di Dartmouth College, John Kemeny dan Thomas Kurtz memiliki misi yang lebih demokratis. Mereka percaya bahwa komputer harus bisa diakses oleh mahasiswa non-teknik. FORTRAN terlalu rumit. Assembly terlalu menakutkan.
Pada tahun 1964, mereka menciptakan \textbf{BASIC} (\textit{Beginner's All-purpose Symbolic Instruction Code}).

BASIC dirancang untuk menjadi ramah.
\texttt{10 PRINT "HELLO WORLD"}
\texttt{20 GOTO 10}
Dua baris kode ini telah menjadi pintu gerbang bagi jutaan programmer di dekade-dekade berikutnya (termasuk Bill Gates dan Elon Musk). BASIC mungkin tidak efisien, dan strukturnya yang penuh `GOTO` sering dikritik oleh ilmuwan komputer (seperti Edsger Dijkstra), tetapi ia memiliki satu kualitas Artisan yang tak ternilai: \textbf{Aksesibilitas}. Ia menurunkan tangga bagi orang biasa untuk memanjat menara gading komputasi.

Di tahun 1965, Gordon Moore dari Fairchild Semiconductor (sebelum mendirikan Intel) mengamati sebuah pola: jumlah transistor dalam sirkuit terpadu (IC) berlipat ganda setiap tahun (kemudian direvisi menjadi setiap 2 tahun), sementara biayanya tetap.
Ini adalah \textbf{Hukum Moore}.
Hukum ini bukan hukum fisika; ia adalah \textit{Self-Fulfilling Prophecy} industri. Ia menjadi metronom yang mengatur ritme inovasi selama 50 tahun ke depan, menjanjikan bahwa komputer akan selalu menjadi lebih cepat, lebih kecil, dan lebih murah. Bagi Artisan, Hukum Moore adalah janji bahwa batasan perangkat keras hari ini akan hilang esok hari, jadi jangan takut untuk memimpikan perangkat lunak yang "berat".

\section{1966 -- 1968: The Mother of All Demos dan Jaringan Antargalaksi}

Puncak dari visi Licklider tentang simbiosis manusia-komputer terjadi pada sore hari tanggal 9 Desember 1968 di San Francisco.
Acara: Fall Joint Computer Conference.
Pembicara: \textbf{Douglas Engelbart} dari Stanford Research Institute (SRI).

Selama 90 menit, Engelbart mendemonstrasikan sebuah sistem bernama \textbf{oN-Line System (NLS)}.
Apa yang ia tunjukkan hari itu membuat para hadirin ternganga. Ingat, saat itu cara umum berinteraksi dengan komputer adalah kartu punch atau baris perintah teletype.
Engelbart menunjukkan:
\begin{itemize}
    \item Sebuah kotak kayu kecil dengan roda di bawahnya (Mouse).
    \item Layar yang dibagi menjadi beberapa jendela (Windows).
    \item Teks yang bisa diklik untuk menuju ke halaman lain (Hypertext).
    \item Kolaborasi dokumen secara \textit{real-time} dengan video dan audio (Video Conferencing/Google Docs).
    \item Pengeditan teks yang dinamis (Word Processing).
\end{itemize}

Demo ini kemudian dikenal sebagai \textbf{"The Mother of All Demos"}.
Engelbart tidak sedang mempresentasikan produk jualan. Ia sedang mempresentasikan sebuah filosofi: \textbf{Augmentasi Kecerdasan Manusia}.
Ia percaya bahwa masalah dunia menjadi semakin kompleks, dan satu-satunya cara manusia bisa menyelesaikannya adalah dengan meningkatkan (augment) kemampuan intelektual kolektif kita melalui alat bantu teknologi.

Bagi Artisan 2026, Engelbart adalah Santo Pelindung Interaksi (\textit{Patron Saint of Interaction}). Semua yang kita gunakan hari ini—mouse, GUI, link, kolaborasi cloud—semuanya ditarik dari visi Engelbart tahun 1968.

Sementara itu, di kantor ARPA, penerus Licklider, Bob Taylor dan Lawrence Roberts, sedang bergumul dengan masalah praktis. Mereka mendanai komputer-komputer hebat di berbagai universitas (MIT, Utah, UCLA, SRI), tetapi komputer-komputer ini terisolasi. Jika Anda ingin menggunakan komputer di Utah, Anda harus pergi ke Utah.

Mereka membutuhkan cara untuk menghubungkan mesin-mesin ini. Mereka membutuhkan "Jaringan Komputer Antargalaksi" (\textit{Intergalactic Computer Network}), istilah bercanda Licklider yang menjadi serius.
Tantangannya: Jaringan telepon yang ada (\textit{Circuit Switching}) tidak efisien untuk data komputer yang bersifat "meledak-ledak" (\textit{bursty}). Jika Anda membangun koneksi sirkuit, jalur itu didedikasikan untuk Anda meskipun Anda diam. Itu mahal.

Solusinya datang dari tiga pemikir terpisah: Paul Baran (RAND), Donald Davies (NPL Inggris), dan Leonard Kleinrock (UCLA). Konsepnya adalah \textbf{Packet Switching}.
Pecah data menjadi paket-paket kecil. Beri label alamat tujuan pada setiap paket. Lempar paket-paket itu ke jaringan seperti surat di kantor pos. Biarkan setiap \textit{router} (waktu itu disebut IMP - \textit{Interface Message Processor}) memutuskan jalur mana yang terbaik untuk setiap paket. Di tujuan, rakit kembali paket-paket itu.

Ini adalah ide yang radikal. AT\&T (perusahaan telepon raksasa) menolaknya dan mengatakan itu tidak akan berhasil. Tapi para "anak muda" di ARPA tidak peduli. Mereka mengontrak BBN (Bolt, Beranek and Newman) untuk membangun IMP tersebut.

\section{1969: Tahun Keajaiban - Bulan, Kabel, dan UNIX}

Tahun terakhir dekade ini, 1969, mungkin adalah tahun paling ajaib dalam sejarah teknologi dan kemanusiaan. Tiga peristiwa monumental terjadi hampir bersamaan.

\textbf{Pertama, Apollo 11.}
Pada bulan Juli, Neil Armstrong dan Buzz Aldrin mendarat di Bulan. Di balik keberhasilan ini terdapat \textbf{Apollo Guidance Computer (AGC)}. Dibuat oleh MIT Instrumentation Lab, ini adalah komputer portabel pertama yang menggunakan \textit{Integrated Circuits} (IC). Dengan memori hanya 72KB (ROM) dan 4KB (RAM), perangkat lunak yang ditulis oleh tim Margaret Hamilton ini harus menavigasi pesawat ruang angkasa sejauh 240.000 mil, mendarat dengan presisi, dan kembali.
Pelajaran Artisan dari AGC adalah \textbf{Keandalan Ekstrem} dan \textbf{Prioritas}. Saat alarm kesalahan "1202" berbunyi tepat sebelum pendaratan (karena memori penuh), sistem operasi AGC cukup pintar untuk membuang tugas prioritas rendah (seperti radar) dan fokus pada tugas prioritas tinggi (pendaratan). \textit{Graceful degradation} menyelamatkan misi.

\textbf{Kedua, ARPANET Online.}
Pada 29 Oktober, dari sebuah ruangan di UCLA, mahasiswa Charley Kline mencoba login ke komputer di Stanford Research Institute (SRI). Ia mengetik "L". Telepon berdering, "Dapat L?". "Ya."
Ia mengetik "O". "Dapat O?". "Ya."
Ia mengetik "G". Sistem \textit{crash}.
Pesan pertama di internet adalah "LO". (Mungkin singkatan profetik untuk \textit{Lo and Behold!}).
Meskipun \textit{crash}, koneksi pertama ini menandai lahirnya jaringan yang kelak menjadi Internet. Empat simpul pertama (UCLA, SRI, UCSB, Utah) terhubung di akhir tahun. Dunia tidak lagi terdiri dari pulau-pulau data yang terisolasi.

\textbf{Ketiga, Kelahiran UNIX.}
Di Bell Labs, Ken Thompson, Dennis Ritchie, dan Rudd Canaday merasa frustrasi. Proyek sistem operasi raksasa mereka sebelumnya, Multics (bersama MIT dan GE), gagal karena terlalu ambisius dan rumit. Bell Labs menarik diri.
Kehilangan akses ke mainan Multics yang canggih (dan game \textit{Space Travel} favoritnya), Ken Thompson menemukan sebuah PDP-7 tua yang tidak terpakai.
Ia memutuskan untuk menulis sistem operasi sendiri. Tapi kali ini, filosofinya berbeda dari Multics. Alih-alih sistem raksasa yang melakukan segalanya, ia menginginkan sistem yang kecil, sederhana, dan elegan.
Ia membangun sistem file hierarkis. Ia membangun konsep proses. Ia membangun shell.
Brian Kernighan, rekannya, menyebutnya \textbf{UNIX} (pelesetan dari Multics—\textit{Uni} vs \textit{Multi}).

UNIX bukanlah proyek resmi. Ia adalah proyek "bawah tanah" (`skunkworks`).
Tanpa mereka sadari, mereka sedang membangun sistem operasi paling berpengaruh dalam sejarah. Filosofi UNIX—\textit{Do one thing and do it well}, \textit{Everything is a file}, \textit{Shell pipes}—menjadi kitab suci bagi Artisan sistem selama 50 tahun ke depan. Linux, macOS, Android, iOS, dan seluruh server cloud hari ini adalah keturunan langsung atau spiritual dari peretasan Thompson di PDP-7 tua itu.

\section{Refleksi Dekade: Memanusiakan Mesin}

Dekade 1960-an ditutup dengan perubahan paradigma yang total.
Di awal dekade, manusia mengantri untuk melayani mesin. Di akhir dekade, mesin mulai melayani manusia di meja mereka sendiri.

Para Artisan tahun 60-an—Licklider, Engelbart, tim ARPANET, dan peretas UNIX—mewariskan sesuatu yang lebih berharga daripada kode: mereka mewariskan \textbf{Semangat Kebebasan}.
Mereka menolak didikte oleh keterbatasan fisik mesin mainframe. Mereka menolak otoritas sentral yang menentukan kapan dan bagaimana mereka boleh menghitung.

Mereka meretas waktu (Time-Sharing), mereka meretas ruang (ARPANET), dan mereka meretas birokrasi (UNIX).
Sebagai Artisan 2026, setiap kali Anda membuka terminal, setiap kali Anda menggunakan mouse, dan setiap kali Anda terhubung ke Wi-Fi, Anda sedang menikmati buah dari pemberontakan intelektual tahun 1960-an. Tugas kita adalah menjaga semangat itu: bahwa teknologi harus selalu memperluas ("augment") potensi manusia, bukan membatasi atau menggantikannya.
