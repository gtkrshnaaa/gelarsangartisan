\chapter{The Era of Interactivity (1960 -- 1969)}

Dekade 1960-an adalah masa di mana komputer mulai berhenti menjadi sekadar "mesin hitung" yang kaku dan mulai menjadi "medium" untuk interaksi manusia. Ini adalah dekade yang melahirkan konsep yang kita sebut sebagai *personal computing*, internet, dan sistem operasi modern. Jika 1950-an adalah tentang abstraksi bahasa, maka 1960-an adalah tentang abstraksi waktu dan ruang. Sebagai Artisan, kita melihat era ini sebagai momen di mana kontrol mulai bergeser dari operator mesin ke pengguna akhir.

\section{1960: Standarisasi Logika dan Minikomputer (ALGOL \& PDP-1)}

Tahun ini membawa upaya pertama untuk menciptakan bahasa pemrograman universal yang benar-benar elegan secara matematis, sekaligus kelahiran mesin yang lebih "intim".

\begin{description}
    \item[ALGOL 60 (Algorithmic Language)] \textit{Saat pertama kali dibuat} oleh sebuah komite internasional, ALGOL 60 memperkenalkan konsep struktur blok (\textit{block structure}) dan cakupan variabel (\textit{variable scope}) yang menjadi standar bagi hampir semua bahasa pemrograman imperatif saat ini. Ia adalah bahasa yang dirancang untuk kesempurnaan deskripsi algoritma.

    \textit{Pada saat buku ini dibuat di tahun 2026}, kita melihat ALGOL bukan sebagai bahasa yang banyak digunakan secara komersial, tetapi sebagai "DNA" dari C, Java, dan Rust. Sebagai Artisan, ALGOL mengajari kita tentang pentingnya desain yang bersih. Pengaruh ALGOL tidak datang dari dominasi pasar (yang dimenangkan oleh FORTRAN dan COBOL), tetapi dari keanggunan strukturnya yang merasuki pikiran para desainer bahasa selama dekade-dekade berikutnya. Ini adalah bukti bahwa desain yang superior secara intelektual akan selalu menemukan cara untuk mempengaruh masa depan, meskipun ia tidak menjadi produk yang paling laris.

    \item[DEC PDP-1] \textit{Saat pertama kali dibuat}, PDP-1 adalah komputer pertama yang tidak memerlukan pendingin ruangan besar dan perlengkapan operator yang rumit. Ia interaktif—memiliki layar CRT dan memungkinkan pengguna untuk mengoperasikannya secara langsung. Di mesin inilah game komputer pertama, *Spacewar!*, lahir.

    \textit{Pada saat buku ini dibuat}, kita melihat PDP-1 sebagai kakek buyut dari *workstation* modern. Sebagai Artisan, PDP-1 mengajarkan kita tentang "pengaruh melalui aksesibilitas". Ketika sebuah mesin menjadi lebih mudah dijangkau oleh tangan manusia tanpa perantara, kreativitas meledak. *Spacewar!* bukan sekadar game; ia adalah bukti bahwa komputer bisa digunakan untuk kegembiraan dan eksplorasi, bukan hanya perhitungan pajak.
\end{description}

\section{1961: Waktu yang Terbagi (Time-sharing \& CTSS)}

Salah satu lompatan konseptual terbesar terjadi di MIT dengan pengembangan *Compatible Time-Sharing System* (CTSS).

\begin{description}
    \item[CTSS \& Time-sharing] \textit{Saat pertama kali dibuat} oleh Fernando Corbató, *Compatible Time-Sharing System* (CTSS) memungkinkan banyak pengguna untuk menggunakan satu komputer secara bersamaan melalui terminal jarak jauh. Inti dari sistem ini adalah *Supervisor*—sebuah program manajemen yang sangat canggih pada masanya, yang bertugas mengatur alokasi waktu CPU dan memori untuk setiap pengguna. CTSS membuktikan bahwa interaksi *real-time* bukanlah kemewahan, melainkan kebutuhan bagi produktivitas intelektual.

    \textit{Pada saat buku ini dibuat}, konsep *multi-tasking* dan *multi-user* adalah hal yang kita anggap remeh. Namun, di tahun 1961, ini adalah revolusi dalam efisiensi intelektual. Sebagai Artisan, *time-sharing* mengajarkan kita tentang demokratisasi sumber daya. Pengaruh sejati datang bukan dari memiliki segalanya, tapi dari kemampuan untuk membagi kapasitas tanpa mengorbankan pengalaman individu. Ini adalah dasar dari ekonomi awan (\textit{cloud computing}) yang kita gunakan hari ini. Keberhasilan Corbató mengingatkan kita bahwa arsitektur yang adil adalah arsitektur yang berkelanjutan.
\end{description}

\section{1962: Jaringan Antar Galaksi dan Interaksi Visual}

Tahun ini, J.C.R. Licklider merumuskan visi tentang jaringan global, sementara Ivan Sutherland membangun alat desain visual pertama.

\begin{description}
    \item[The Intergalactic Computer Network] \textit{Saat pertama kali dibuat} sebagai sebuah memo di ARPA, visi Licklider tentang jaringan komputer yang saling terhubung secara global dianggap sebagai fantasi sains. Ia membayangkan sebuah dunia di mana informasi dan program dapat diakses secara instan dari mana saja.

    \textit{Pada saat buku ini dibuat}, visi Licklider telah terwujud sebagai Internet. Sebagai Artisan, Licklider mengajari kita tentang kekuatan visi strategis. Ia tidak membangun internet dengan tangannya sendiri, tapi ia membangun "imajinasi kolektif" yang menggerakkan pendanaan dan riset selama dekade berikutnya. Inilah *The Art of Influence* di level tertinggi: menanamkan benih pemikiran yang akan tumbuh menjadi pohon dunia.

    \item[Sketchpad (Ivan Sutherland)] \textit{Saat pertama kali dibuat} sebagai tesis doktoralnya, Sketchpad adalah program pertama yang memungkinkan manusia berinteraksi dengan komputer melalui grafik dan pena cahaya (\textit{light pen}). Ia memperkenalkan konsep objek (\textit{objects}) dan properti (\textit{properties}) dalam grafik komputer.

    \textit{Pada saat buku ini dibuat}, setiap CAD dan perangkat lunak desain grafis adalah keturunan langsung dari Sketchpad. Sebagai Artisan, Sutherland mengajarkan kita bahwa antarmuka visual adalah jembatan paling kuat untuk mengekspresikan niat manusia ke dalam mesin. Keindahan teknis Sketchpad terletak pada kemampuannya untuk mengubah abstraksi koordinat menjadi bentuk yang bisa kita manipulasi secara intuitif.
\end{description}

\section{1963: Bahasa Universal dan Alat Navigasi (ASCII \& Mouse)}

Dunia komputasi mulai menstandarisasi pertukaran data, sementara alat navigasi baru mulai dipatenkan.

\begin{description}
    \item[ASCII (American Standard Code for Information Interchange)] \textit{Saat pertama kali dibuat}, ASCII bertujuan untuk mengakhiri kekacauan kode karakter yang berbeda-beda di setiap produsen mesin. Ia memberikan kode numerik standar untuk huruf, angka, dan simbol kontrol.

    \textit{Pada saat buku ini dibuat}, ASCII tetap menjadi standar dasar pertukaran teks di seluruh dunia digital. Sebagai Artisan, kita belajar bahwa standarisasi adalah bentuk pengaruh yang paling stabil. Dengan menyetujui "bahasa pertukaran" yang sederhana, kita memungkinkan sistem-sistem yang berbeda untuk saling memahami. ASCII adalah fondasi dari komunikasi global kita.

    \item[The Mouse Patent (Douglas Engelbart)] \textit{Saat pertama kali dibuat} di SRI, prototipe mouse pertama adalah kotak kayu kecil dengan dua roda logam. Patennya menyebutnya sebagai "Indikator Posisi X-Y untuk Sistem Tampilan".

    \textit{Pada saat buku ini dibuat}, mouse telah menjadi bagian integral dari cara kita berinteraksi dengan komputer selama hampir 40 tahun (sebelum layar sentuh mengambil alih sebagian perannya). Sebagai Artisan, Engelbart mengajari kita tentang desain yang berorientasi pada manusia. Ia melihat mouse bukan sebagai perangkat tambahan, tetapi sebagai ekstensi dari niat manusia. Pengaruhnya bertahan karena ia menyentuh cara fisik manusia berinteraksi dengan dunia digital.
\end{description}

\section{1964: Standarisasi Arsitektur dan Bahasa Rakyat (IBM S/360 \& BASIC)}

Tahun ini menandai kelahiran keluarga komputer pertama yang memiliki kompatibilitas perangkat lunak, serta bahasa yang membawa pemrograman ke massa.

\begin{description}
    \item[IBM System/360] \textit{Saat pertama kali dibuat}, S/360 adalah taruhan sebesar \$5 miliar dari IBM untuk menciptakan satu arsitektur yang bisa mencakup segala hal, mulai dari kebutuhan bisnis hingga sains. Ia menggunakan teknologi *Solid Logic Technology* (SLT)—sebuah perantara antara sirkuit diskrit dan biner terintegrasi penuh. Yang paling revolusioner adalah pengembangan OS/360, sistem operasi yang sangat ambisius sehingga melahirkan buku klasik Fred Brooks, *The Mythical Man-Month*. Brooks mengajarkan kita bahwa menambah tenaga manusia ke proyek yang terlambat justru akan membuatnya semakin terlambat.

    \textit{Pada saat buku ini dibuat}, konsep kompatibilitas mundur (\textit{backward compatibility}) dan satu arsitektur yang konsisten adalah apa yang membuat sistem seperti x86 bertahan begitu lama. Sebagai Artisan, S/360 mengajarkan kita tentang pentingnya integritas arsitektur. Pengaruh yang masif hanya bisa dicapai bila kita membangun fondasi yang konsisten di berbagai kapasitas. IBM mengajari kita bahwa keteraturan adalah kunci dari dominasi industri, namun kompleksitas manajemen yang berlebihan adalah risiko terbesar kita.

    \item[BASIC (Beginner's All-purpose Symbolic Instruction Code)] \textit{Saat pertama kali dibuat} di Dartmouth College oleh John Kemeny dan Thomas Kurtz, BASIC dirancang agar mahasiswa non-tekno bisa menggunakan komputer. Ia sederhana, interaktif, dan mudah dipelajari.

    \textit{Pada saat buku ini dibuat}, kita melihat BASIC sebagai gerbang masuk bagi jutaan programmer di seluruh dunia selama dekade 1970-an dan 80-an. Sebagai Artisan, BASIC mengajari kita tentang kekuatan kesederhanaan. Terkadang, pengaruh terbesar bukan datang dari bahasa yang paling kuat, tapi dari bahasa yang paling mudah dipahami. BASIC meruntuhkan dinding antara "elit komputer" dan masyarakat umum, memulai revolusi literasi digital yang kita nikmati hari ini.
\end{description}

\section{1965: Hukum Moore dan Komputasi untuk Semua (PDP-8)}

Laju pertumbuhan teknologi mendapatkan hukumnya, sementara komputer pertama yang benar-benar bisa "dibeli" oleh laboratorium kecil mulai dipasarkan.

\begin{description}
    \item[Moore's Law (Gordon Moore)] \textit{Saat pertama kali dibuat} sebagai sebuah pengamatan empiris dalam majalah *Electronics*, Gordon Moore memprediksi bahwa jumlah komponen dalam sirkuit terintegrasi akan berlipat ganda setiap tahun (kemudian direvisi menjadi dua tahun). Ini bukan sekadar prediksi teknis, tapi menjadi ramalan yang mengatur seluruh strategi industri semikonduktor.

    \textit{Pada saat buku ini dibuat}, Hukum Moore telah menjadi mesin penggerak peradaban digital selama 60 tahun. Sebagai Artisan, kita harus menghargai disiplin pertumbuhan ini. Gordon Moore mengajari kita tentang "skalabilitas eksponensial". Kita harus membangun sistem yang tidak hanya bekerja untuk hari ini, tapi siap untuk menghadapi lonjakan daya yang sudah diprediksi di masa depan. Pengaruh adalah tentang memahami ritme kemajuan dan menari bersamanya.

    \item[DEC PDP-8] \textit{Saat pertama kali dibuat}, PDP-8 adalah minikomputer pertama yang sukses secara komersial massal. Ia seukuran lemari es kecil dan harganya cukup terjangkau sehingga departemen universitas atau bisnis kecil bisa memilikinya sendiri tanpa harus berbagi dengan pusat data besar.

    \textit{Pada saat buku ini dibuat}, kita melihat PDP-8 sebagai awal dari desentralisasi komputasi. Sebagai Artisan, PDP-8 mengajarkan kita tentang demokratisasi alat. Ketika alat yang kuat jatuh ke tangan lebih banyak orang, inovasi tidak lagi menjadi monopoli korporasi besar. Pengaruh sejati menyebar melalui fragmentasi dan aksesibilitas.
\end{description}

\section{1966: Komunikasi Paket dan Akar Internet}

Di saat dunia fisik sedang sibuk dengan gejolak sosial, dunia digital mulai merumuskan cara berkomunikasi yang tahan banting.

\begin{description}
    \item[Packet Switching (Donald Davies \& Paul Baran)] \textit{Saat pertama kali dibuat} secara independen di Inggris dan AS, konsep pensakelar paket (\textit{packet switching}) mengusulkan untuk membagi pesan menjadi potongan-potongan kecil yang dapat mengambil jalur berbeda menuju tujuan. Ini adalah antitesis dari jaringan telepon yang kaku.

    \textit{Pada saat buku ini dibuat}, setiap paket data yang Anda terima saat membaca buku ini adalah bukti kejeniusan Baran dan Davies. Sebagai Artisan, kita belajar tentang ketahanan (\textit{resiliency}). Dengan merancang sistem yang bisa beradaptasi dengan kerusakan dan perubahan jalur, kita menciptakan struktur yang abadi. Pengaruh tidak datang dari kekakuan sirkuit, tapi dari kelenturan arsitektur yang bisa menyesuaikan diri dengan kondisi medan.
\end{description}

\section{1967: Hypertext dan Smalltalk Roots}

Benih-benih cara kita mengelola pengetahuan dan memprogram secara berorientasi objek mulai tumbuh.

\begin{description}
    \item[Hypertext (Andries van Dam \& Ted Nelson)] \textit{Saat pertama kali dibuat} di Brown University, sistem *Hypertext Editing System* adalah salah satu implementasi pertama dari ide Ted Nelson tentang teks yang saling terhubung secara non-linear.

    \textit{Pada saat buku ini dibuat}, web yang kita jelajahi adalah realisasi dari Hypertext. Sebagai Artisan, kita belajar bahwa struktur informasi menentukan cara kita berpikir. Dengan memberikan pengaruh pada bagaimana data saling terhubung, kita memberikan pengaruh pada bagaimana manusia memahami pengetahuan. Sejarah IT adalah sejarah penghancuran linearitas demi konektivitas.
\end{description}

\section{1968: Ibu dari Segala Demo (The Mother of All Demos)}

Tahun ini mencatat momen paling berpengaruh dalam sejarah antarmuka manusia-komputer yang dilakukan oleh Douglas Engelbart.

\begin{description}
    \item[The Mother of All Demos] \textit{Saat pertama kali dibuat} sebagai demonstrasi selama 90 menit di San Francisco, Engelbart menunjukkan untuk pertama kalinya: *mouse*, *windows*, *hypertext*, pemrosesan kata grafis, dan konferensi video. Dunia saat itu belum siap untuk memahami apa yang mereka lihat.

    \textit{Pada saat buku ini dibuat}, kita menyadari bahwa Engelbart telah "mengimpor masa depan" ke tahun 1968. Sebagai Artisan, momen ini adalah lambang dari *The Art of Influence* yang berani. Ia tidak hanya membangun fitur; ia membangun visi tentang bagaimana teknologi bisa meningkatkan kecerdasan manusia (\textit{Augmenting Human Intellect}). Pelajarannya adalah: jangan takut untuk menjadi "terlalu maju" dari zaman Anda. Jika visi Anda benar, dunia akan mengejarnya, meskipun butuh waktu 20 tahun.
\end{description}

\section{1969: Bulan, UNIX, dan Lahirnya ARPANET}

Dekade ini ditutup dengan tiga pencapaian manusia yang paling monumental dalam sejarah teknologi.

\begin{description}
    \item[Apollo 11 Guidance Computer (AGC)] \textit{Saat pertama kali dibuat} oleh MIT Instrumentation Lab, AGC adalah salah satu komputer pertama yang menggunakan sirkuit terintegrasi silikon. Ia harus sangat handal dan ringan untuk bisa membawa manusia ke bulan dan kembali dengan selamat.

    \textit{Pada saat buku ini dibuat}, kita melihat AGC sebagai puncak dari keandalan sistem kritis (\textit{mission-critical systems}). Sebagai Artisan, AGC mengajari kita tentang tanggung jawab. Di tahun 2026, kita mungkin tidak selalu mengirim manusia ke bulan, tapi kode kita mengelola kehidupan, uang, dan kesehatan orang lain. Kedisiplinan AGC adalah standar emas bagi setiap pengrajin kode.

    \item[The Birth of UNIX (Ken Thompson \& Dennis Ritchie)] \textit{Saat pertama kali dibuat} di sebuah mesin PDP-7 bekas di Bell Labs, UNIX lahir dari keinginan untuk memiliki sistem operasi yang sederhana, elegan, dan bisa dipindahkan (\textit{portable}). Thompson dan Ritchie tidak mencari kompleksitas, mereka mencari kejelasan.

    \textit{Pada saat buku ini dibuat}, setiap server Linux, Mac, dan hampir semua infrastruktur internet berjalan di atas filosofi UNIX. Sebagai Artisan, UNIX adalah kitab suci kita. Pelajarannya abadi: "Lakukan satu hal, dan lakukan dengan baik." Pengaruh UNIX datang dari kesederhanaannya yang mematikan. Ia mengarahkan dunia sistem operasi bukan dengan paksa, tapi dengan menjadi standar yang terlalu benar untuk diabaikan.

    \item[ARPANET (The First Message)] \textit{Saat pertama kali dibuat}, ARPANET menghubungkan UCLA dan SRI. Pesan pertama yang dikirim adalah "LO" (dari kata "LOGIN", sebelum sistem tersebut hancur). Ini adalah momen "Big Bang" bagi internet.

    \textit{Pada saat buku ini dibuat}, kita hidup di dalam ARPANET yang telah tumbuh menjadi entitas global yang menyatukan seluruh umat manusia. Sebagai Artisan, kita belajar bahwa permulaan yang kecil dan bahkan "gagal" di awal (seperti pesan "LO" tersebut) bisa menjadi awal dari sesuatu yang mengubah galaksi. Pengaruh sejati membutuhkan kesabaran untuk tumbuh dari beberapa simpul menjadi jaring-jaring yang tak terbatas.
\end{description}

\section{Atmosfer Era: Revolusi di Setiap Bit}

Dekade 1960-an adalah periode perlawanan terhadap kemapanan. Budaya tandingan (counter-culture) memengaruhi teknologi secara mendalam. Komputer yang tadinya adalah alat otoritas (pemerintah dan korporasi besar) mulai "direbut" oleh para peretas idealis yang ingin membebaskan informasi.

\textit{Saat pertama kali dibuat}, suasana ini melahirkan etika peretas (*hacker ethic*) pertama di lab-lab MIT. Ada rasa urgensi untuk membuat sistem yang terbuka dan bisa diakses secara kolektif. Ini adalah dekade di mana "personalitas" mulai masuk ke dalam komputasi. Kita tidak lagi hanya menjalankan tugas, kita sedang berdialog dengan mesin.

\textit{Pada saat buku ini dibuat di tahun 2026}, kita melihat gema yang sama dalam gerakan perangkat lunak sumber terbuka (*open source*) dan desentralisasi web. Sebagai Artisan, kita harus memahami bahwa teknologi selalu membawa muatan budaya. Pengaruh kita sebagai Artisan bukan hanya teknis, tapi moral. 1960-an mengajari kita bahwa inovasi yang paling bertahan lama adalah inovasi yang memberikan kekuatan kepada individu, bukan hanya kepada institusi.

\section{Disiplin Sang Artisan: Kejelasan di Atas Kompleksitas}

Pelajaran terpenting dari dekade ini bagi seorang Artisan adalah "Filosofi UNIX": Kejelasan adalah segalanya.

\textit{Saat pertama kali dibuat}, di tengah tren sistem operasi raksasa yang gemuk dan kompleks (seperti Multics), Ken Thompson dan Dennis Ritchie memilih jalur asketik. Mereka membuang semua yang tidak perlu. Disiplin untuk berkata "tidak" pada fitur tambahan demi menjaga kesucian desain adalah apa yang membuat UNIX abadi.

\textit{Pada saat buku ini dibuat}, kita seringkali tergoda untuk menambahkan lapisan-lapisan abstraksi yang tidak perlu atau mengikuti tren *framework* yang berubah setiap musim. Sebagai Artisan, kita harus kembali ke disiplin 1960-an. Pengaruh yang kokoh dibangun di atas fondasi yang sederhana dan bisa diprediksi. "Keep It Simple, Stupid" bukan sekadar slogan, tapi jalan hidup Artisan untuk memastikan karya kita tetap bisa dipahami dan dikembangkan oleh generasi berikutnya. Kejelasan adalah bentuk tertinggi dari rasa hormat kita kepada sesama Artisan.

\section{Refleksi Dekade: Menuju Cakrawala Baru}

Dekade 1960-an ditutup dengan langkah kaki di bulan dan simpul pertama internet. Manusia telah membuktikan bahwa mereka bisa melampaui batas fisik planet ini melalui bantuan mesin logika.

\begin{description}
    \item[Warisan Sang Artisan] \textit{Saat pertama kali dibuat}, dekade ini memberikan kita Internet (ARPANET), UNIX, Mouse, Grafik Komputer, dan Moore's Law. Ini adalah "Zaman Keemasan" inovasi konseptual.

    \textit{Pada saat buku ini dibuat}, kita menyadari bahwa setiap klik mouse, setiap paket data, dan setiap perintah terminal kita hari ini adalah warisan dari mimpi-mimpi liar di tahun 1960-an. Sebagai Artisan, kita tidak hanya mewarisi teknologi mereka, tapi juga semangat pemberontakan mereka terhadap keterbatasan. Kita menggunakan pengaruh kita untuk terus mendorong batas-batasan apa yang mungkin, sambil tetap setia pada keanggunan logika yang telah mereka letakkan. Kita adalah anak-anak revolusi 1960-an, yang sedang menulis bab baru di era AI.
\end{description}
