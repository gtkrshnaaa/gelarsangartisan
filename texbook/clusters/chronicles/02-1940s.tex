\chapter{The Era of Giants (1940 -- 1949)}

Setelah teori diletakkan di dekade sebelumnya, Perang Dunia II mempercepat kelahiran mesin-mesin fisik. Ini adalah masa di mana "bug" pertama ditemukan dan arsitektur yang kita gunakan hari ini didefinisikan secara resmi.

\section{1941: Dunia Digital Pertama}

\begin{description}
    \item[Z3 (Konrad Zuse)] \textit{Saat pertama kali dibuat}, Z3 adalah komputer digital pertama di dunia yang dapat diprogram sepenuhnya dan otomatis. Menggunakan 2.000 relay, mesin ini bekerja dengan kecepatan clock antara 5 hingga 10 Hertz.

    \textit{Pada saat buku ini dibuat}, kecepatan clock itu terdengar lucu—ponsel di saku kita miliaran kali lebih cepat. Namun, Z3 adalah bukti bahwa presisi relay bisa menggantikan roda gigi mekanis. Tanpa langkah berani Zuse ini, jalur menuju transistor mungkin tidak akan pernah terbuka.
\end{description}

\section{1945: Von Neumann \& ENIAC}

\begin{description}
    \item[The First Draft (John von Neumann)] \textit{Saat pertama kali dibuat} melalui laporan "First Draft of a Report on the EDVAC", Von Neumann merinci arsitektur komputer (CPU, Memori, Unit Input/Output) yang kita gunakan hingga tahun 2026 ini. 

    \textit{Pada saat buku ini dibuat}, meskipun kita sudah memiliki prosesor multi-core dengan miliaran transistor, fondasi dasarnya tetap "Von Neumann Architecture". Menakjubkan melihat bagaimana sebuah draf laporan dari tahun 1945 masih mengendalikan cara kita membangun perangkat lunak modern.
\end{description}

\section{1947: Lahirnya Kristal Ajaib}

\begin{description}
    \item[The Transistor (Bell Labs)] \textit{Saat pertama kali dibuat}, transistor pertama di Bell Labs berukuran cukup besar untuk digenggam. Ia diciptakan untuk menggantikan tabung vakum yang panas, boros energi, dan sering pecah.

    \textit{Pada saat buku ini dibuat}, kita sedang memasukkan puluhan miliar transistor ke dalam area seukuran kuku jari. Transistor adalah "batu bata" dari peradaban digital kita. Sebagai artisan, kita harus menghormati komponen kecil ini yang memungkinkan semua abstraksi perangkat lunak yang kita cintai.
\end{description}
