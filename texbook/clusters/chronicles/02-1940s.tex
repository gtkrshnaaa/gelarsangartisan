\chapter{The Era of Giants (1940 -- 1949)}

Setelah teori logika diletakkan dalam keheningan tahun 1930-an, dekade 1940-an datang dengan ledakan yang memekakkan telinga. Perang Dunia II menjadi akselerator brutal bagi kelahiran mesin-mesin fisik. Ini adalah masa di mana komputasi ditarik paksa dari ruang seminar universitas yang tenang dan dilemparkan ke tengah lumpur pertempuran hidup dan mati.

Bagi seorang Artisan, dekade ini mengajarkan pelajaran yang paling mendalam tentang \textit{urgensi}. Abstraksi tidak lagi cukup; ia harus bermanifestasi menjadi aksi. Di era inilah "bug" pertama ditemukan dalam arti harfiah, dan arsitektur yang kita gunakan hingga milenium ketiga didefinisikan secara resmi di tengah kepulan asap mesiu.

\section{1940 -- 1943: Mesin Perang Spesialis}

Di Inggris Tengah, sebuah rumah perkebunan bergaya Victoria bernama Bletchley Park menjadi pusat dari upaya intelektual paling rahasia dalam sejarah. Di sini, Alan Turing dan rekan-rekannya tidak sedang menulis makalah; mereka sedang berpacu melawan waktu untuk mematahkan enkripsi Enigma Jerman.

Dari keputusasaan inilah lahir \textbf{The Bombe}. Ini bukanlah komputer yang "elegan" seperti yang dibayangkan Turing di tahun 1936. Ini adalah mesin elektromekanis raksasa yang berisik, penuh dengan roda gigi berputar dan kabel yang semrawut. Bombe tidak dibangun untuk melakukan apa saja; ia dibangun untuk melakukan \textit{satu} hal: mensimulasikan rotor Enigma untuk menemukan kunci harian.

Di Berlin, Konrad Zuse bekerja dalam isolasi yang berbeda. Di tengah reruntuhan bom Sekutu, ia menyelesaikan \textbf{Z3} pada tahun 1941—komputer program-terkontrol pertama yang beroperasi penuh. Menggunakan ribuan relay bekas telepon, Z3 adalah bukti keteguhan hati seorang insinyur tunggal. Ironisnya, Z3 hancur oleh serangan udara pada tahun 1943, mengajarkan kita bahwa perangkat keras itu fana, namun logika yang mendasarinya abadi.

Puncak dari era spesialis ini adalah \textbf{Colossus} (1943) di Inggris. Dibangun oleh Tommy Flowers menggunakan ribuan tabung vakum, Colossus adalah raksasa elektronik pertama yang dapat diprogram. Ia membaca pita kertas dengan kecepatan optik 5.000 karakter per detik untuk memecahkan kode Lorenz yang jauh lebih rumit daripada Enigma. Kemenangan Colossus bukan hanya teknis, tapi strategis: ia memperpendek perang selama berbulan-bulan, menyelamatkan jutaan nyawa. Ini adalah momen di mana \textit{Information Superiority} resmi menjadi senjata perang yang lebih mematikan daripada artileri.

\section{1944 -- 1945: Penderitaan yang Melahirkan Arsitektur}

Di seberang Atlantik, Amerika Serikat memasuki gelanggang dengan sumber daya yang jauh lebih besar. Di Harvard, Howard Aiken mewujudkan mimpi Babbage dengan \textbf{Harvard Mark I} (1944). Mesin sepanjang 15 meter ini adalah "tarian baja" yang digerakkan oleh poros mekanis dan motor listrik.

Di sinilah \textbf{Grace Hopper}—salah satu programmer wanita pertama—belajar untuk "berbicara" dengan mesin. Ia harus memahami ritme mekanis Mark I untuk memberinya makan pita instruksi. Dan di sinilah, di dalam relay panel F, seekor ngengat malang terjepit dan mati, melahirkan istilah legendaris: \textbf{"Bug"}. Insiden kecil ini adalah pengingat abadi bagi kita: bahwa dunia digital yang abstrak selalu rentan terhadap kekacauan dunia fisik yang kotor.

Namun, lompatan terbesar terjadi di Universitas Pennsylvania dengan \textbf{ENIAC} (1945). Berbeda dengan Mark I yang lambat, ENIAC adalah monster elektronik dengan 18.000 tabung vakum. Ia bisa menghitung lintasan balistik ribuan kali lebih cepat daripada manusia.

Tapi ENIAC memiliki cacat fatal: ia sulit diprogram. Untuk mengganti tugas dari menghitung rudal ke menghitung reaksi nuklir, tim programmer wanita jenius—Jean Bartik, Kay McNulty, dan lainnya—harus merangkak di dalam mesin, mencabut dan memasang kembali ribuan kabel secara fisik selama berhari-hari.

Dari penderitaan mengaturs ulang kabel inilah lahir wawasan terbesar abad ke-20. \textbf{John von Neumann}, melihat kesulitan ini, merumuskan konsep \textbf{Stored-Program} dalam \textit{First Draft of a Report on the EDVAC}. Idenya radikal namun sederhana: Mengapa kita harus mengutak-atik perangkat keras untuk mengubah program? Mengapa tidak menyimpan instruksi program \textit{di dalam} memori yang sama dengan data?

Arsitektur Von Neumann ini memisahkan "jiwa" (software) dari "tubuh" (hardware) untuk selamanya. Di tahun 2026, setiap kali kita mengunduh aplikasi baru tanpa harus menyolder ulang HP kita, kita sedang menikmati buah dari wawasan Von Neumann yang lahir dari kabel-kabel kusut ENIAC.

\section{1946 -- 1947: Revolusi Fisika Material}

Setelah perang usai, dunia mulai melihat potensi damai dari mesin-mesin ini. Namun, tabung vakum yang menjadi jantung ENIAC dan Colossus adalah komponen yang buruk: panas, boros energi, dan sering meledak.

Di Bell Labs, tiga ilmuwan—Shockley, Bardeen, dan Brattain—sedang mencari alternatif. Pada akhir 1947, mereka menyatukan kontak emas pada sepotong kristal germanium dan menemukan efek amplifikasi. Inilah kelahiran \textbf{Transistor}.

Bagi Artisan, ini adalah momen "The Magic Crystal". Transistor adalah sakelar yang tidak bergerak. Ia mengontrol aliran elektron bukan dengan mekanika, tapi dengan fisika kuantum zat padat. Penemuan ini akan mengecilkan raksasa seukuran ruangan menjadi kepingan yang muat di saku, memungkinkan demokratisasi teknologi yang kita nikmati hari ini.

\section{1948 -- 1949: Pembuktian Konsep}

Dekade ini ditutup dengan perlombaan untuk membuktikan teori Von Neumann. Di Manchester, \textbf{The Baby} (1948) menjadi komputer pertama yang menjalankan program dari memori elektronik. Program pertamanya sederhana—mencari faktor bilangan tertinggi—tetapi implikasinya seismik: perangkat lunak telah lahir sebagai entitas yang cair dan mudah diubah.

Setahun kemudian, \textbf{EDSAC} di Cambridge mulai beroperasi sebagai komputer praktis pertama. Maurice Wilkes, arsiteknya, menyadari bahwa ia menghabiskan lebih banyak waktu memperbaiki program daripada menulisnya. Dari sinilah lahir konsep \textbf{Subrutin} dan \textbf{Library}. Wilkes mulai menyimpan potongan kode yang sering dipakai di dalam "rak perpustakaan" agar tidak perlu ditulis ulang.

\section{Refleksi Dekade: Dari Kabel ke Kode}

Jika 1930-an adalah tentang \textit{Mimpi}, maka 1940-an adalah tentang \textit{Konstruksi}. Kita memulai dekade dengan mesin yang harus dibangun ulang secara fisik untuk setiap tugas baru, dan mengakhirinya dengan mesin yang bisa berubah fungsi hanya dengan memuat pita kertas baru.

Sebagai Artisan modern, kita berhutang budi pada dekade ini untuk kemudahan yang kita miliki. Kita tidak lagi perlu tahu cara kerja transistor secara intim atau menyambung kabel untuk membuat \textit{looping}. Namun, bahayanya adalah kita menjadi terlalu berjarak dari realitas fisik mesin kita.

Pelajaran dari Bletchley Park dan ENIAC adalah bahwa \textbf{performance comes from understanding the hardware}. Para wanita yang memprogram ENIAC tahu persis berapa milidetik yang dibutuhkan sinyal untuk merambat dari satu panel ke panel lain. Di tahun 2026, meskipun kita bekerja dengan \textit{cloud} dan \textit{serverless}, Artisan terbaik adalah mereka yang masih bisa "mendengar" detak mesin di balik lapisan abstraksi—mereka yang tahu bahwa di ujung sana, masih ada transistor yang beralih state, panas yang dihasilkan, dan batas fisik yang harus dihormati.
