\chapter{The Chronicles of Tech}
Sejarah teknologi bukan sekadar daftar penemuan. Ini adalah evolusi pemikiran manusia yang dituangkan ke dalam silikon dan listrik. Sebagai seorang \textit{artisan}, saya melihat setiap era sebagai pergeseran cara kita berinteraksi dengan kemungkinan.

\section{Era Fondasi (1940 -- 1959)}

Di sinilah segalanya dimulai. Ketika komputer masih sebesar ruangan dan diprogram menggunakan kabel fisik.

\begin{description}
    \item[1941] \textbf{Z3}: \textit{Saat pertama kali dibuat}, ini adalah komputer digital pertama yang dapat diprogram sepenuhnya, lahir di Jerman oleh Konrad Zuse. \textit{Pada saat buku ini ditulis di 2026}, kita melihat warisan logika biner ini tetap menjadi fondasi dasar mesin-mesin kuantum tercanggih kita.
    \item[1945] \textbf{ENIAC}: Menjadi sorotan dunia sebagai mesin kalkulasi raksasa. Di tahun yang sama, John von Neumann menulis laporan tentang EDVAC, mendefinisikan arsitektur komputer yang kita gunakan hingga hari ini.
    \item[1947] \textbf{Transistor}: Revolusi sejati. Bell Labs mengganti tabung vakum yang panas dan rapuh dengan silikon yang efisien.
    \item[1953] \textbf{COBOL \& Grace Hopper}: Munculnya bahasa yang lebih dekat dengan bahasa manusia, membawa teknologi keluar dari laboratorium murni.
\end{description}

\section{Era Abstraksi (1960 -- 1979)}

Teknologi mulai mengecil dan menyebar. Ini adalah era di mana \textit{Artisan} mulai mendapatkan alat yang lebih presisi.

\begin{description}
    \item[1969] \textbf{UNIX}: Ken Thompson dan Dennis Ritchie melahirkan sistem operasi yang menjadi fondasi hampir semua hal modern hari ini.
    \item[1972] \textbf{C Language}: Sebuah alat musik yang sangat kuat. Fleksibel, cepat, dan menjadi bahasa ibu bagi banyak sistem operasi.
    \item[1975] \textbf{Microsoft}: Bill Gates dan Paul Allen mulai meracik masa depan perangkat lunak.
    \item[1976] \textbf{Apple I}: Wozniak merancang papan yang akan mengubah cara individu memiliki komputer.
\end{description}

\section{Era Konsumerisme \& GUI (1980 -- 1999)}

Komputer masuk ke rumah-rumah. Grafis menggantikan teks mentah.

\begin{description}
    \item[1984] \textbf{Macintosh}: Memperkenalkan GUI kepada massa. Mouse menjadi tangan digital kita.
    \item[1991] \textbf{Linux}: Linus Torvalds merilis kernel yang akan menjalankan dunia, dari server hingga \textit{embedded systems}.
    \item[1995] \textbf{Java \& JavaScript}: Web mulai bernafas. Interaktivitas bukan lagi mimpi.
    \item[1998] \textbf{Google}: Cara kita mencari pengetahuan berubah selamanya.
\end{description}

\section{Era Konektivitas \& AI (2000 -- 2026)}

Kita sampai di masa sekarang. Masa di mana teknologi menjadi ekstensi dari diri kita sendiri.

\begin{description}
    \item[2007] \textbf{iPhone}: Dunia di telapak tangan. Aplikasi menjadi ekosistem baru.
    \item[2015] \textbf{TensorFlow}: Google membuka gerbang kecerdasan buatan untuk semua orang.
    \item[2022] \textbf{ChatGPT}: Ledakan LLM. Komputer mulai "berbicara" dan "memahami".
    \item[2025-2026] \textbf{The Agentic Era}: AI bukan lagi sekadar \textit{chatbox}. Ia adalah \textit{multi-agent systems} yang bekerja secara otonom. Di tahun 2026 ini, saat saya menulis buku ini, kita melihat \textit{coding} telah berevolusi dari sekadar mengetik sintaks menjadi kolaborasi tingkat tinggi dengan agen cerdas.
\end{description}
