\chapter{The Cloud \& AI Revolution (2010 -- 2019)}

Dekade 2010-an adalah dekade di mana perangkat lunak benar-benar "memakan dunia" (\textit{Software is eating the world} - Marc Andreessen).
Jika dekade 2000-an adalah tentang menghubungkan manusia (Sosial), maka dekade 2010-an adalah tentang \textbf{Abstraksi Mesin} (Cloud) dan \textbf{Kebangkitan Kecerdasan} (AI).

Di awal dekade, kita masih mengelola server. Di akhir dekade, kita mengelola "Layanan".
Di awal dekade, AI adalah fiksi ilmiah. Di akhir dekade, AI mengalahkan manusia dalam permainan paling rumit di dunia dan menulis prosa yang koheren.

Bagi Artisan di tahun 2026, dekade ini mengajarkan tentang \textbf{Kecepatan Komposisi}.
Kita tidak lagi membangun dari batu bata mentah. Kita membangun dengan balok-balok LEGO raksasa yang sudah jadi: Autentikasi (Auth0), Pembayaran (Stripe), Infrastruktur (AWS), dan Kecerdasan (TensorFlow).
Tantangan bergeser dari "Bagaimana cara membuatnya?" menjadi "Apa yang harus saya buat dengan kekuatan sebesar ini?"

\section{2010: Era Pasca-PC dan Budaya Visual}

Pada 27 Januari 2010, Steve Jobs duduk di sofa di panggung Yerba Buena Center dan memegang sebuah lempengan kaca.
\textbf{iPad}.
Banyak yang mengejek: "Itu cuma iPhone besar!" atau "Siapa yang butuh perangkat di antara laptop dan HP?"
Namun, Jobs benar. iPad menandai dimulainya \textbf{Era Post-PC}.
Komputer tidak lagi harus memiliki keyboard dan mouse. Komputer bisa menjadi \textit{Intim}.
iPad digunakan oleh pilot di kokpit, dokter di ruang operasi, dan balita di ruang tamu. Ia membuktikan bahwa hambatan terbesar komputasi bukanlah kekuatan prosesor, melainkan \textbf{Antarmuka}. Jika antarmukanya alami, nenek berusia 80 tahun pun bisa menjadi "pengguna komputer".

Di saat yang sama, sebuah aplikasi kecil bernama \textbf{Instagram} diluncurkan.
Kevin Systrom dan Mike Krieger memahami satu hal: Kamera HP itu jelek, tapi orang ingin merasa artistik.
Solusinya: \textbf{Filter}.
Dengan satu klik, foto buram menjadi karya seni vintage.
Instagram mengubah internet dari tempat "Berbagi Teks/Tautan" (Twitter/Facebook) menjadi tempat "Berbagi Pengalaman Visual".
Dalam hitungan bulan, ia memiliki jutaan pengguna. Facebook membelinya seharga \$1 Miliar pada tahun 2012. Saat itu, Instagram hanya memiliki 13 karyawan.
Ini adalah pelajaran efisiensi ekstrem bagi Artisan: Tim kecil dengan produk yang tepat bisa bernilai lebih dari perusahaan manufaktur dengan ribuan buruh. Kode adalah pengungkit (\textit{leverage}) terbesar dalam sejarah.

\section{2011 -- 2012: Kematian Sang Maestro dan Lahirnya Deep Learning}

5 Oktober 2011. Steve Jobs meninggal dunia.
Dunia teknologi berkabung. Kita kehilangan Artisan terbesar kita, orang yang mengajarkan bahwa "Rasa" (\textit{Taste}) dan "Desain" sama pentingnya dengan MegaHertz dan GigaBytes. Warisannya adalah integritas produk: Bahwa bagian dalam komputer harus serapi bagian luarnya, meskipun tidak ada yang melihatnya.

Namun, saat satu pintu tertutup, pintu lain terbuka.
Pada tahun 2012, sebuah kompetisi pengenalan gambar bernama \textbf{ImageNet} diguncang oleh tim dari Universitas Toronto (Geoffrey Hinton, Alex Krizhevsky, Ilya Sutskever).
Mereka menggunakan teknik lama yang sudah ditinggalkan orang: \textbf{Jaringan Saraf Tiruan} (\textit{Neural Networks}).
Selama bertahun-tahun, AI didominasi oleh pendekatan logika/aturan (\textit{Rule-based}). Neural Net dianggap lambat dan tidak berguna.
Tapi tim ini memiliki senjata rahasia: \textbf{GPU} (Graphics Processing Unit).
Chip NVIDIA yang biasanya dipakai main game ternyata sangat bagus untuk melakukan perkalian matriks paralel yang dibutuhkan Neural Net.
Model mereka, \textbf{AlexNet}, menghancurkan rekor akurasi sebelumnya.
Ini adalah momen "Big Bang" untuk \textbf{Deep Learning}.
Tiba-tiba, komputer bisa melihat. Mereka bisa mengenali kucing, anjing, kanker di X-ray, dan wajah di CCTV.
Revolusi AI dimulai di sini. Bukan dengan kode baru yang pintar, tapi dengan data yang banyak dan komputasi yang brutal.

\section{2013 -- 2014: Kontainer dan Orkestrasi}

Di dunia pengembangan perangkat lunak, ada satu masalah klasik: \textit{"It works on my machine."} (Ini jalan di laptop saya).
Kode yang berjalan lancar di laptop pengembang sering kali hancur saat dipindah ke server produksi karena perbedaan versi library atau OS.
Solomon Hykes, pendiri dotCloud, punya ide: Bagaimana jika kita membungkus kode \textit{dan} semua lingkungan OS-nya ke dalam satu kotak standar?
Ia menamainya \textbf{Docker} (2013).
Kontainer Docker lebih ringan dari Virtual Machine (VM). Ia menyala dalam milidetik.
Docker mengubah industri. Kita berhenti mengirim "kode"; kita mulai mengirim "lingkungan".

Tapi, bagaimana jika Anda punya 1.000 kontainer? Bagaimana mengaturnya?
Google, yang sudah menjalankan miliaran kontainer secara internal, merilis rahasia mereka.
\textbf{Kubernetes} (2014).
(Bahasa Yunani untuk "Nakhoda").
Kubernetes adalah sistem operasi untuk Data Center. Anda tidak lagi bilang "Jalankan ini di Server A". Anda bilang "Saya butuh 5 replika dari aplikasi ini, dan pastikan mereka selalu hidup." Kubernetes yang akan mencari server kosong, memantau kesehatan, dan me-restart jika ada yang mati.
Ini adalah tingkat abstraksi baru. Infrastruktur menjadi \textbf{Immutable} (Tak Berubah) dan \textbf{Declarative}.
Bagi Artisan, ini membebaskan kita dari tugas "menyusui server" (\textit{pet vs cattle}). Server adalah ternak, bukan hewan peliharaan. Jika sakit, ganti baru.

Di dunia Antarmuka Pengguna (UI), Facebook merilis \textbf{React} (2013).
Sebelum React, kode frontend adalah "spageti" jQuery yang memanipulasi DOM secara langsung. Rumit dan rentan bug.
React memperkenalkan ide radikal: \textbf{UI adalah Fungsi dari State}.
Jangan sentuh DOM. Cukup ubah datanya (\textit{State}), dan React akan menggambar ulang UI-nya secara efisien (\textit{Virtual DOM}).
Ini membuat pengembangan aplikasi web yang kompleks menjadi mungkin dan terstruktur.

\section{2015 -- 2016: AlphaGo dan Langkah ke-37}

Dunia AI percaya bahwa permainan Go (Weiqi) adalah benteng terakhir kecerdasan manusia.
Catur sudah dikalahkan Deep Blue (1997) karena catur bisa dihitung (\textit{brute force}).
Tapi Go memiliki kemungkinan langkah lebih banyak daripada atom di alam semesta. Komputer tidak bisa menghitung semua kemungkinan. Ia butuh "Intuisi".
Para ahli memperkirakan butuh 10 tahun lagi bagi AI untuk mengalahkan juara dunia Go.

Maret 2016. Seoul, Korea Selatan.
\textbf{AlphaGo} (buatan Google DeepMind) vs \textbf{Lee Sedol} (Juara Dunia Legendaris).
Game 2. Langkah ke-37.
AlphaGo meletakkan batu hitam di baris kelima, posisi yang sangat tidak lazim.
Komentator manusia terkejut. "Itu kesalahan," pikir mereka. "Tidak ada manusia yang bermain seperti itu."
Lee Sedol sendiri tertegun dan keluar ruangan untuk merokok.
Ternyata, itu bukan kesalahan. Itu adalah langkah jenius yang mematahkan strategi Lee Sedol dan memenangkan permainan.
Langkah 37 adalah bukti bahwa AI bukan lagi sekadar meniru manusia; ia telah menemukan cara berpikir baru yang \textbf{Melampaui Manusia}.

Ini adalah momen Sputnik bagi abad ke-21.
Kita sadar bahwa kita tidak lagi sendirian di puncak piramida kognitif. Kita telah menciptakan sesuatu yang bisa "berpikir" dengan cara yang asing namun efektif.

\section{2017: Attention Is All You Need}

Namun, revolusi terbesar terjadi dalam diam lewat sebuah makalah ilmiah.
Para peneliti Google merilis paper berjudul \textit{"Attention Is All You Need"}.
Mereka memperkenalkan arsitektur jaringan saraf baru bernama \textbf{Transformer}.
Sebelumnya, AI memproses bahasa kata demi kata (seperti manusia membaca). Ini lambat dan AI sering lupa konteks awal kalimat saat sampai di akhir.
Transformer memproses seluruh kalimat \textit{sekaligus} secara paralel. Mekanisme \textit{Self-Attention} memungkinkan AI memberikan bobot pada hubungan antar kata, tidak peduli seberapa jauh jaraknya dalam teks.

Transformer memungkinkan kita melatih model bahasa pada miliaran halaman teks internet.
Inilah benih yang akan tumbuh menjadi \textbf{LLM} (Large Language Models) seperti GPT dan Claude.
Tanpa disadari saat itu, Google telah memberikan kunci ke kotak Pandora Generative AI.

\section{2018 -- 2019: Skandal Data dan Etika AI}

Di penghujung dekade, sisi gelap teknologi mulai terkuak.
Skandal \textbf{Cambridge Analytica} (2018) mengungkapkan bahwa data Facebook 87 juta pengguna dicuri dan digunakan untuk memanipulasi pemilu AS.
Tiba-tiba, kita sadar bahwa "gratis" di internet berarti "Anda adalah produknya".
Kepercayaan publik terhadap Silicon Valley runtuh. Gerakan \#DeleteFacebook muncul.
Uni Eropa merespons dengan \textbf{GDPR} (General Data Protection Regulation). Undang-undang privasi paling ketat dalam sejarah. Ia memaksa perusahaan untuk transparan tentang data apa yang mereka ambil.

Di dunia AI, OpenAI (yang didirikan sebagai organisasi nirlaba untuk menjaga AI tetap aman) merilis \textbf{GPT-2} (2019).
Model ini bisa menulis berita palsu yang sangat meyakinkan.
OpenAI awalnya menolak merilis model penuhnya ke publik karena dianggap "terlalu berbahaya".
Ini memicu debat global: Siapa yang berhak mengontrol kecerdasan buatan? Apakah ia harus terbuka (\textit{Open Source}) seperti Linux, atau dijaga ketat seperti senjata nuklir?

\section{Refleksi Dekade: Abstraksi yang Memabukkan}

Dekade 2010-an memberikan kita kekuatan super.
Dengan satu perintah `docker run`, kita memanggil sistem operasi.
Dengan satu perintah `import tensorflow`, kita memanggil otak buatan.
Dengan satu perintah `aws ec2 run`, kita memanggil superkomputer.

Namun, kemudahan ini datang dengan harga: \textbf{Hilangnya Pemahaman}.
Banyak Artisan muda yang ahli menggunakan \textit{Framework} tetapi tidak mengerti apa yang terjadi di balik layar. Mereka bisa membuat React Component tetapi tidak mengerti cara kerja browser. Mereka bisa melatih model AI tetapi tidak mengerti aljabar linear di baliknya.
Kita menjadi "Perakit" (\textit{Assemblers}), bukan "Pencipta" (\textit{Creators}).

Tantangan bagi Artisan di dekade berikutnya (2020-an) adalah untuk tidak terlena oleh abstraksi.
Gunakan alat-alat canggih ini, ya. Tapi jangan biarkan mereka menjadi kotak hitam yang tidak Anda mengerti.
Kekuatan sejati Artisan adalah kemampuan untuk \textbf{Menembus Abstraksi} (\textit{Piercing the Abstraction}) saat hal itu bocor atau rusak.
Di era di mana mesin semakin pintar, manusia yang paling berharga adalah yang mengerti cara kerja mesin tersebut sampai ke baut terakhirnya.
