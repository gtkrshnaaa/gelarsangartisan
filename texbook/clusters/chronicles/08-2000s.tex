\chapter{The Mobile \& Social Era (2000 -- 2009)}

Jika 1990-an adalah tentang membangun jalan raya informasi, maka 2000-an adalah tentang bagaimana kita mulai hidup di dalamnya. Ini adalah dekade di mana teknologi berpindah dari meja kerja ke saku celana, dan dari dokumen statis menjadi percakapan global yang tak pernah putus. Sebagai Artisan, kita melihat 2000-an sebagai era "Kematangan Infrastruktur". Gelembung spekulasi meledak, namun dari abunya muncul fondasi yang jauh lebih kokoh.

\section{2000: Pecahnya Gelembung dan Kelahiran C\#}

Tahun di mana realitas ekonomi menghantam fantasi digital, namun inovasi bahasa pemrograman tetap melaju kencang.

\begin{description}
    \item[The Dot-com Burst] \textit{Saat pertama kali dibuat}, pasar saham teknologi jatuh secara dramatis. Banyak perusahaan web yang tidak memiliki model bisnis nyata gulung tikar.
    
    \textit{Pada saat buku ini dibuat di tahun 2026}, kita melihat peristiwa ini sebagai "Pembersihan Hutan". Hanya perusahaan yang memiliki pondasi teknis dan nilai nyata (seperti Amazon dan Google) yang bertahan. Sebagai Artisan, peristiwa ini mengajari kita tentang "Nilai Intrinsik". Jangan membangun di atas pasir spekulasi; bangunlah di atas batu efisiensi dan kegunaan nyata. \textit{The Art of Influence} di sini adalah tentang ketahanan (\textit{resilience}) di tengah badai.

    \item[C\# (C-Sharp) \& .NET Preview] \textit{Saat pertama kali dibuat} oleh Anders Hejlsberg di Microsoft, C\# adalah jawaban Microsoft terhadap dominasi Java. Di balik layar, ia dirancang untuk menggabungkan kekuatan komputasi C++ dengan kemudahan pengembangan Visual Basic. Ia memperkenalkan konsep \textit{Managed Code} ke ekosistem Windows secara masif. Di balik layar, kode C\# dikompilasi menjadi \textit{Intermediate Language} (IL) yang kemudian dijalankan oleh \textit{Common Language Runtime} (CLR) menggunakan \textit{Just-In-Time} (JIT) compilation.

    \textit{Pada saat buku ini dibuat}, C\# adalah salah satu bahasa paling elegan dengan dukungan *cross-platform* (melalui .NET Core). Sebagai Artisan, C\# mengajari kita tentang "Sintaksis sebagai Puisi". Di balik layar, fitur-fitur seperti *Properties*, *Delegates*, dan nantinya *LINQ* adalah alat Artisan untuk menyederhanakan logika data yang kompleks menjadi ekspresi yang jernih dan deklaratif. Pengaruh C\# datang dari kemampuannya untuk menyeimbangkan antara performa rendah (*low-level performance*) dan produktivitas tinggi (*high-level productivity*).
\end{description}

\section{2001: Stabilitas Desktop dan Pengetahuan Kolektif}

Tahun yang menyatukan dunia sistem operasi Apple dan Microsoft ke dalam pondasi yang lebih modern dan stabil.

\begin{description}
    \item[Windows XP (The NT Fusion)] \textit{Saat pertama kali dibuat}, XP adalah langkah berani Microsoft untuk akhirnya mengakhiri lini Windows 9x/Me yang berantakan dan menyatukan semua pengguna ke mesin berbasis NT (\textit{New Technology}). Di balik layar, ini berarti stabilitas yang jauh lebih tinggi melalui pemisahan memori yang lebih ketat dan manajemen driver yang lebih baik.

    \textit{Pada saat buku ini dibuat}, XP dikenang sebagai sistem operasi paling ikonik dalam sejarah. Sebagai Artisan, XP mengajari kita tentang "Penyatuan Visi". Di balik layar, kernel NT 5.1 adalah mahakarya rekayasa yang harus mendukung jajaran perangkat keras yang sangat luas. Pelajaran bagi Artisan: stabilitas adalah bentuk tertinggi dari rasa hormat kepada pengguna.

    \item[Mac OS X 10.0 (Cheetah)] \textit{Saat pertama kali dibuat}, ini adalah kelahiran kembali Apple di bawah Steve Jobs setelah penggabungan dengan NeXT. Di balik layar, ia dibangun di atas fondasi Unix yang disebut *Darwin*. Ini adalah pergeseran dari sistem operasi proprietari yang rapuh ke kekuatan open-source (FreeBSD/Mach kernel). Ia memperkenalkan antarmuka *Aqua* yang memukau mata, namun di bawahnya terdapat terminal bash yang siap tempur.

    \textit{Pada saat buku ini dibuat}, kita menyadari bahwa OS X adalah alasan mengapa banyak pengembang (Artisan) jatuh cinta pada Mac. Di balik layar, ia menawarkan stabilitas kernel \textit{Preemptive Multitasking} dan manajemen memori terproteksi. Inilah \textit{The Art of Influence}: menyembunyikan kompleksitas mesin perang kelas industrial di balik keindahan seni visual yang elegan. Pelajaran Artisan: kecantikan antarmuka harus selalu didukung oleh fondasi logika yang tak tergoyahkan.

    \item[Wikipedia Birth] \textit{Saat pertama kali dibuat} oleh Jimmy Wales dan Larry Sanger, Wikipedia menggunakan konsep "Wiki" (cepat dalam bahasa Hawaii) untuk memungkinkan siapa pun mengedit ensiklopedia.

    \textit{Pada saat buku ini dibuat}, Wikipedia adalah tulang punggung kebenaran publik. Sebagai Artisan, Wikipedia mengajari kita tentang "Kekuatan Algoritma Sosial". Di balik layar, ia bukan tentang kode yang rumit, tapi tentang protokol interaksi manusia yang didesain untuk menyaring kesalahan secara kolektif.
\end{description}

\section{2002: Era Jaringan dan Masa Depan Framework}

Komputer mulai belajar untuk tidak hanya diam, tapi terus berbicara satu sama lain dalam skala yang lebih terorganisir.

\begin{description}
    \item[.NET Framework 1.0 Release] \textit{Saat pertama kali dibuat}, ini adalah reposisi besar Microsoft menjadi perusahaan layanan web. Di balik layar, ia memperkenalkan *Common Language Runtime* (CLR), mesin virtual yang memungkinkan berbagai bahasa (C\#, VB.NET, J\#) berjalan di atas rumah yang sama.

    \textit{Pada saat buku ini dibuat}, kita menghargai konsep *Interoperabilitas* yang dibawa .NET. Sebagai Artisan, kita belajar tentang pentingnya "Standar Bersama". Pengaruh besar dicapai ketika kita menyediakan panggung di mana banyak Artisan lain bisa berkarya bersama tanpa konflik bahasa.

    \item[Xbox Live Launch] \textit{Saat pertama kali dibuat}, ini adalah taruhan Microsoft bahwa masa depan konsol game bukan pada "mesin tunggal", tapi pada "layanan jaringan".
    
    \textit{Pada saat buku ini dibuat}, industri game adalah penggerak utama inovasi grafis dan jaringan. Di balik layar, pengelolaan jutaan koneksi *real-time* dengan latensi rendah adalah keajaiban rekayasa jaringan.
\end{description}

\section{2003: Demokratisasi Penerbitan dan P2P Suara}

Tahun di mana suara setiap orang mulai mendapatkan panggung digital yang mudah dan murah.

\begin{description}
    \item[WordPress 0.7 (Matt Mullenweg)] \textit{Saat pertama kali dibuat} sebagai *fork* dari b2/cafelog, WordPress bertujuan untuk memberi kemudahan bagi siapa pun untuk memiliki "suara" di web. Di balik layar, ia menggunakan PHP dan MySQL untuk meruntuhkan tembok teknis penulisan *online*.

    \textit{Pada saat buku ini dibuat}, WordPress menjalankan hampir setengah dari seluruh situs web di bumi. Sebagai Artisan, WordPress mengajari kita tentang "Kekuatan Ekstensibilitas". Di balik layar, sistem *Plug-in* dan *Theme* adalah cara paling cerdik untuk membiarkan komunitas memperpanjang hidup sebuah produk.

    \item[Skype (P2P Telephony)] \textit{Saat pertama kali dibuat}, Skype menggunakan teknologi *peer-to-peer* (serupa dengan Kazaa) untuk mengirim suara melalui internet tanpa perlu server pusat yang mahal.
    
    \textit{Pada saat buku ini dibuat}, kita melihat ini sebagai awal dari runtuhnya dominasi telekomunikasi tradisional. Sebagai Artisan, Skype mengajari kita tentang "Efisien dalam Desentralisasi". Jika kita bisa menggunakan sumber daya pengguna (bandwidth) untuk menjalankan layanan, kita telah mencapai tingkat efisiensi tertinggi.
\end{description}
\section{2004: Kelahiran Web 2.0 dan Raksasa Sosial}

Tahun di mana web berubah dari kumpulan halaman statis menjadi platform interaktif yang dinamis.

\begin{description}
    \item[The Term "Web 2.0" (Tim O'Reilly)] \textit{Saat pertama kali dibuat}, istilah ini mendefinisikan pergeseran web menjadi tempat kolaborasi pengguna. Di balik layar, teknologi kuncinya adalah AJAX (\textit{Asynchronous JavaScript and XML}). Ia memungkinkan halaman web memperbarui data tanpa harus menyegarkan (\textit{refresh}) seluruh halaman.

    \textit{Pada saat buku ini dibuat}, kita melihat AJAX sebagai dasar dari setiap aplikasi web modern (React, Vue, dll). Sebagai Artisan, AJAX mengajari kita tentang "Interaksi Tanpa Hambatan". Di balik layar, objek *XMLHttpRequest* adalah pahlawan tanpa tanda jasa yang mengubah internet dari perpustakaan menjadi aplikasi hidup. Inilah \textit{The Art of Influence}: membuat pengguna lupa bahwa mereka sedang menggunakan browser.

    \item[Facebook Birth (TheFacebook)] \textit{Saat pertama kali dibuat} oleh Mark Zuckerberg di asrama Harvard, Facebook hanyalah direktori mahasiswa. Di balik layar, ia membangun kepercayaan melalui verifikasi email kampus, memanfaatkan kekuatan jaringan sosial untuk menyebar secara eksponensial.

    \textit{Pada saat buku ini dibuat}, kita melihat dampak sosial yang masif dari algoritma umpan (*Feed*). Sebagai Artisan, Facebook mengajari kita tentang "Skalabilitas Database". Di balik layar, kemampuan untuk menangani jutaan hubungan antar entitas data adalah tantangan rekayasa yang melahirkan teknologi database modern (seperti *NoSQL* dan *Graph Databases*).
\end{description}

\section{2005: Tahun Visual dan Kelahiran Alat Sang Artisan}

Tahun di mana video menjadi warga negara kelas satu di web dan alat pengelola kode paling perkasa lahir.

\begin{description}
    \item[YouTube Launch] \textit{Saat pertama kali dibuat} oleh tiga mantan karyawan PayPal, YouTube memecahkan masalah sulit: bagaimana mengunggah dan memutar video di web secara instan tanpa perlu *plugin* yang rumit (awalnya menggunakan Flash).
    
    \textit{Pada saat buku ini dibuat}, YouTube adalah universitas visual dunia. Di balik layar, keajaiban YouTube adalah pada algoritma kompresi dan jaringan pengiriman konten (CDN). Sebagai Artisan, kita belajar tentang pentingnya "Mengurangi Gesekan" (\textit{Reducing Friction}). Jika pengguna bisa melihat video hanya dengan satu klik, Anda telah menguasai dunia.

    \item[Git Version Control (Linus Torvalds)] \textit{Saat pertama kali dibuat} karena Linus Torvalds merasa sistem kontrol versi lama (BitKeeper) sudah tidak lagi mumpuni untuk pengembangan kernel Linux, Git dirancang untuk menjadi sangat cepat, efisien, dan benar-benar terdistribusi. Di balik layar, Git memandang data sebagai kumpulan *snapshots* utuh, bukan sekadar perbedaan baris teks (*diffs*).

    \textit{Pada saat buku ini dibuat}, Git adalah keterampilan wajib bagi setiap Artisan kode. Di balik layar, penggunaan struktur data *Directed Acyclic Graph* (DAG) dan *Merkle Tree* pada Git adalah murni keindahan matematika yang diaplikasikan. Ia menjamin integritas data melalui fungsi \textit{hashing} SHA-1 yang unik untuk setiap \textit{commit}. Pelajaran bagi Artisan: bangunlah alat yang begitu kuat secara arsitektural sehingga ia menjadi standar global karena keunggulannya, bukan karena paksaan vendor.
\end{description}

\section{2006: Awan Mulai Terbentuk dan jQuery}

Tahun di mana infrastruktur berubah dari server fisik menjadi layanan yang bisa disewa dengan API.

\begin{description}
    \item[Amazon Web Services (AWS) - EC2 \& S3] \textit{Saat pertama kali dibuat}, Amazon melakukan sesuatu yang radikal: menyewakan infrastruktur internal mereka kepada publik. Di balik layar, ini adalah awal dari revolusi *Cloud Computing*. Anda tidak lagi perlu membeli server fisik; Anda hanya perlu memanggil API.

    \textit{Pada saat buku ini dibuat}, dunia berjalan di atas "Awan". Sebagai Artisan, AWS mengajari kita tentang "Komoditisasi Infrastruktur". Di balik layar, teknologi virtualisasi (seperti Xen) memungkinkan satu mesin fisik menjalankan ratusan mesin virtual secara elastis. Pengaruh besar dicapai ketika kita mengubah masalah fisik menjadi masalah perangkat lunak.

    \item[jQuery (John Resig)] \textit{Saat pertama kali dibuat}, jQuery bertujuan untuk membuat manipulasi DOM di JavaScript menjadi mudah dan konsisten di semua peramban. Slogannya: "Write Less, Do More".
    
    \textit{Pada saat buku ini dibuat}, meskipun kita menggunakan framework modern, semangat jQuery untuk abstraksi yang elegan tetap hidup. Sebagai Artisan, jQuery mengajari kita tentang "Desain API yang Manusiawi". Sebuah alat yang menyenangkan untuk digunakan akan selalu menang dalam adopsi massal.
\end{description}

\section{2007: Revolusi Saku dan Layanan Awan Pribadi}

Tahun di mana komputer berhenti menjadi benda yang kita datangi, dan mulai menjadi benda yang kita bawa ke mana saja.

\begin{description}
    \item[The iPhone (Mobile Safari)] \textit{Saat pertama kali dibuat}, iPhone bukan sekadar telepon baru; ia adalah komputer Unix yang muat di saku. Di balik layar, keajaiban iPhone bukan hanya pada layar sentuhnya, tapi pada peramban *Mobile Safari* yang membawa "web yang sesungguhnya" ke layar kecil.

    \textit{Pada saat buku ini dibuat}, perangkat seluler mengkonsumsi sebagian besar waktu manusia. Sebagai Artisan, iPhone mengajari kita tentang "Ekonomi Perhatian". Di balik layar, optimasi baterai dan performa pada perangkat kecil adalah batasan yang memaksa kita menulis kode yang lebih efisien. Inilah \textit{The Art of Influence}: mengubah perilaku manusia melalui kemudahan jangkauan.

    \item[Dropbox (Cloud Storage for All)] \textit{Saat pertama kali dibuat} oleh Drew Houston, Dropbox membuktikan bahwa sinkronisasi file itu sulit dan berharga. Di balik layar, algoritma *differential sync* mereka adalah karya seni: hanya mengunggah bagian file yang berubah.
\end{description}

\section{2008: Ekonomi Aplikasi dan Kelahiran Kepercayaan Digital}

Tahun yang melahirkan cara baru bagi para Artisan untuk menjual karya mereka dan cara baru untuk mendefinisikan "nilai".

\begin{description}
    \item[The App Store Launch] \textit{Saat pertama kali dibuat}, Apple membuka gerbang bagi para pengembang independen untuk menjual aplikasi langsung ke jutaan pengguna.

    \textit{Pada saat buku ini dibuat}, ini adalah awal dari "Ekonomi Kode". Seorang Artisan di kamar kosnya sekarang bisa memengaruhi jutaan orang di seluruh dunia. Di balik layar, sistem distribusi digital massal ini adalah pengubah permainan bagi keberlanjutan hidup sang Artisan digital.

    \item[Android (The Open Alternative)] \textit{Saat pertama kali dibuat} oleh Google sebagai jawaban atas iPhone, Android membawa semangat open-source ke dunia seluler. Di balik layar, ia menggunakan kernel Linux dan mesin virtual Java/Dalvik.

    \textit{Pada saat buku ini dibuat}, Android adalah sistem operasi paling luas di dunia. Sebagai Artisan, Android mengajari kita tentang "Pengaruh melalui Keterbukaan". Dengan memberikan sistem operasi secara cuma-cuma kepada produsen perangkat keras, Google memengaruhi seluruh ekosistem tanpa harus memproduksi perangkat itu sendiri.

    \item[Bitcoin Whitepaper (Satoshi Nakamoto)] \textit{Saat pertama kali dibuat} di tengah krisis finansial global 2008 sebagai tindakan perlawanan terhadap sistem perbankan tradisional, Bitcoin adalah proposal untuk sistem uang elektronik yang beroperasi tanpa otoritas pusat. Di balik layar, ia memperkenalkan mekanisme *Blockchain*—sebuah buku besar terdistribusi yang dijaga oleh algoritma *Proof of Work* (PoW). Ia memecahkan masalah *Double Spending* tanpa perlu perantara pihak ketiga. 
    
    \textit{Pada saat buku ini dibuat}, kita melihat ini sebagai kelahiran "Kepercayaan Matematis Digital". Sebagai Artisan, Bitcoin mengajari kita tentang "Kekuatan Konsensus Deklaratif". Di balik layar, aturan protokol adalah hukum (\textit{Code is Law}). Penggunaan rantai \textit{hash} (setiap blok merujuk pada \textit{hash} blok sebelumnya) menciptakan sejarah yang tidak bisa diubah tanpa memanipulasi seluruh kekuatan komputasi jaringan. Inilah bentuk tertinggi dari kedaulatan informasi yang terjangkar pada kriptografi, bukan pada kepercayaan institusi.

    \item[Stack Overflow Launch] \textit{Saat pertama kali dibuat} oleh Jeff Atwood dan Joel Spolsky, Stack Overflow merevolusi cara pengembang mencari solusi atas masalah teknis mereka. Di balik layar, ia menggunakan sistem gamifikasi (reputasi dan lencana) untuk mendorong partisipasi komunitas.

    \textit{Pada saat buku ini dibuat}, Stack Overflow adalah memori kolektif seluruh pengembang di dunia. Sebagai Artisan, ia mengajari kita tentang "Nilai Dokumentasi yang Hidup". Di balik layar, sistem pencarian dan kurasi oleh komunitas memastikan bahwa jawaban terbaik selalu berada di puncak. Pengaruh besarnya datang dari keterbukaan akses terhadap pengetahuan yang sebelumnya terkunci di kepala para ahli atau buku-buku mahal.
\end{description}

\section{2009: JavaScript di Server dan Bahasa Masa Depan}

Dekade ini ditutup dengan pergeseran besar dalam paradigma pemrograman yang akan menyatukan dunia frontend dan backend.

\begin{description}
    \item[Node.js Birth (Ryan Dahl)] \textit{Saat pertama kali dibuat} dan dipresentasikan di JSConf EU, Node.js memungkinkan JavaScript berjalan di luar peramban untuk pertama kalinya dengan performa tinggi. Di balik layar, ia menggunakan mesin V8 Google (yang dikembangkan untuk Chrome) dan arsitektur *event-driven non-blocking I/O*.

    \textit{Pada saat buku ini dibuat}, Node.js adalah tulang punggung dari jutaan aplikasi real-time. Sebagai Artisan, Node.js mengajari kita tentang "Efisien dalam Menangani Volume". Di balik layar, meskipun berjalan pada satu *thread*, sistem *Event Loop* memungkinkan Node.js menangani ribuan koneksi konkuren secara asinkron tanpa beban berat *thread context switching*. Inilah keindahan rekayasa yang memaksimalkan sumber daya terbatas menjadi performa tak terbatas.

    \item[Go Language (Google)] \textit{Saat pertama kali dibuat}, Go dirancang untuk memecahkan masalah komputasi skala besar di Google. Slogannya: sederhana, cepat, dan andal.
    
    \textit{Pada saat buku ini dibuat}, Go adalah bahasa default untuk infrastruktur awan (Docker, Kubernetes). Sebagai Artisan, Go mengajari kita tentang "Kekuatan dalam Kesederhanaan". Terkadang, dengan membuang fitur-fitur yang tidak perlu, kita justru mendapatkan kekuatan yang sesungguhnya.
\end{description}

\section{Atmosfer Era: Era Konektivitas dan Kebisingan}

2000-an adalah era di mana kita berhenti menjadi penonton pasif dan mulai menjadi partisipan aktif di internet.

\textit{Saat pertama kali dibuat}, suasana ini melahirkan optimisme tentang "Global Village". Kita bisa berbicara dengan siapa saja, kapan saja. Namun, ini juga awal dari era "Kebisingan Digital" di mana perhatian kita mulai menjadi komoditas yang diperdagangkan.

\textit{Pada saat buku ini dibuat di tahun 2026}, kita melihat 2000-an sebagai titik balik dari privasi menuju transparansi yang terkadang dipaksakan. Sebagai Artisan, tugas kita adalah menjaga keseimbangan antar koneksi sosial yang bermakna dan kedaulatan pribadi kita atas data.

\section{Disiplin Sang Artisan: Membangun di Atas Raksasa}

Pelajaran terpenting dari dekade ini adalah: Jangan temukan kembali roda (\textit{Don't reinvent the wheel}), tapi pahamilah bagaimana roda itu bekerja.

\textit{Saat pertama kali dibuat}, begitu banyak framework dan pustaka yang bermunculan. Artisans di masa itu harus disiplin untuk memilih alat yang tepat dan tidak terjebak dalam tren sesaat. Mereka belajar untuk membaca kode orang lain (di GitHub) dan berkontribusi kembali.

\textit{Pada saat buku ini dibuat}, kita memiliki jutaan paket di NPM atau Crates.io. Disiplin Artisan di tahun 2026 adalah "Kurasi yang Ketat". Jangan asal pasang paket; pahamilah dependensi Anda. Pengaruh sejati datang dari kode yang minimalis namun perkasa, bukan dari tumpukan kode orang lain yang tidak Anda mengerti.

\section{Refleksi Dekade: Dunia di Genggaman}

Dekade 2000-an berakhir dengan perasaan bahwa masa depan telah tiba lebih awal.

\begin{description}
    \item[Warisan Sang Artisan] \textit{Saat pertama kali dibuat}, dekade ini memberikan kita iPhone, AWS, Git, dan Bitcoin. Ini adalah dekade yang memindahkan kekuasaan dari pusat-pusat data besar ke saku setiap individu.

    \textit{Pada saat buku ini dibuat}, kita menyadari bahwa dekade 2000-an adalah saat di mana teknologi menjadi napas kehidupan. Sebagai Artisan, kita menghargai era ini dengan cara terus membangun alat yang membebaskan, bukan yang menjerat—melalui desentralisasi, efisiensi awan, dan keintiman teknologi seluler yang bertanggung jawab.
\end{description}
