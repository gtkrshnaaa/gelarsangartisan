\chapter{The Mobile \& Social Era (2000 -- 2009)}

Dekade 2000-an diawali dengan kiamat kecil dan diakhiri dengan kelahiran kembali peradaban digital.
Pada awal dekade, kita menatap layar monitor tabung (CRT) yang berat, bekerja dengan komputer yang dirantai ke meja, dan "pergi online" adalah sebuah kegiatan yang disengaja dengan bunyi modem \textit{dial-up} yang memekakkan telinga.
Di akhir dekade, internet ada di saku kita, selalu aktif, selalu terhubung, dan kita mulai mendefinisikan diri kita berdasarkan profil digital kita di Facebook atau Twitter.

Ini adalah dekade \textbf{Infrastruktur Sosial}.
Kita berhenti melihat komputer sebagai alat hitung atau alat kerja semata. Kita mulai melihatnya sebagai alat untuk \textbf{Menghubungkan Manusia}.
Satu per satu, aspek fisik kehidupan kita mulai didigitalkan:
- Musik fisik (CD) menjadi file digital (MP3/iTunes).
- Peta fisik menjadi Google Maps.
- Ensiklopedia fisik menjadi Wikipedia.
- Album foto fisik menjadi Facebook.
- Toko fisik menjadi Amazon.

Bagi Artisan di tahun 2026, dekade ini mengajarkan tentang \textbf{Skalabilitas Manusia}.
Teknologi bukan lagi tentang seberapa cepat prosesor Anda, tetapi tentang seberapa banyak kehidupan manusia yang bisa ia tampung. Kita belajar bahwa kode yang paling berpengaruh bukanlah kode yang paling rumit secara matematis, tetapi kode yang paling memahami psikologi dan kebutuhan sosial manusia.

\section{2000 -- 2001: Ledakan Gelembung dan Konsolidasi}

Maret 2000. NASDAQ mencapai puncaknya, lalu terjun bebas.
Gelembung \textbf{Dot-com} pecah.
Triliunan dolar kekayaan kertas lenyap. Perusahaan seperti \textbf{Pets.com} dan \textbf{Webvan} bangkrut dalam semalam. Mereka memiliki ide yang benar (e-commerce, pengiriman barang), tetapi mereka datang terlalu cepat, dengan infrastruktur yang belum siap dan model bisnis yang membakar uang.
Banyak yang mengatakan "Internet adalah mode sesaat yang sudah lewat."

Namun, kehancuran ini justru membersihkan hutan. Ia membunuh parasit dan menyisakan predator puncak.
Perusahaan yang bertahan—\textbf{Amazon}, \textbf{eBay}, \textbf{Google}—adalah mereka yang benar-benar memberikan nilai. Mereka fokus pada unit ekonomi yang sehat, bukan sekadar "eyeballs" (jumlah pengunjung).
Bagi Artisan, ini adalah pelajaran tentang \textbf{Fundamental}. Jangan membangun bisnis di atas sensasi (\textit{hype}). Bangunlah di atas masalah nyata yang dipecahkan dengan efisiensi nyata.

Di tengah puing-puing ekonomi ini, dunia sistem operasi akhirnya mencapai kedewasaan.
Pada tahun 2001, Microsoft merilis \textbf{Windows XP}.
Setelah bertahun-tahun berjuang dengan DOS yang tidak stabil (Windows 95/98/ME), Microsoft akhirnya memindahkan pengguna rumahan ke kernel \textbf{NT} (\textit{New Technology}) yang kuat. XP adalah sistem operasi yang solid, berwarna, dan tahan banting. Ia menjadi standar de-facto dunia selama lebih dari satu dekade.
Di sisi lain, Apple merilis \textbf{Mac OS X} (Cheetah).
Steve Jobs, yang kembali memimpin, melakukan langkah berani: Ia membuang sistem operasi Mac klasik dan menggantinya dengan \textbf{UNIX} (berbasis NeXTSTEP/BSD). Ia membungkus kekuatan Unix dengan antarmuka grafis yang memukau bernama \textbf{Aqua}. Tombol-tombolnya terlihat seperti permen yang ingin dijilat.
Ini adalah momen penting bagi Artisan: Mac OS X membuktikan bahwa Anda bisa memiliki \textbf{Kekuatan} (Unix terminal) dan \textbf{Keindahan} (GUI) di satu mesin. Inilah alasan mengapa Mac menjadi pilihan utama para pengembang di dekade-dekade berikutnya.

Dan pada Oktober 2001, Steve Jobs mengeluarkan benda ajaib dari sakunya: \textbf{iPod}.
"1.000 lagu di saku Anda."
Sebelum iPod, pemutar MP3 itu rumit, jelek, dan kapasitasnya kecil. iPod memiliki \textit{Scroll Wheel} yang jenius dan hard disk Toshiba 5GB yang mungil.
iPod bukan sekadar gadget; ia adalah penyelamat industri musik (melalui iTunes Store) dan penyelamat Apple. Ia mengajarkan kita bahwa \textbf{Kenyamanan Mengalahkan Kualitas}. Orang rela membayar untuk kemudahan, meskipun kualitas suaranya (MP3) lebih rendah daripada CD.

\section{2003 -- 2005: Web 2.0 dan Kebangkitan Kembali}

Setelah kehancuran dot-com, web perlahan bangkit kembali dengan wajah baru. Tim O'Reilly menyebutnya \textbf{Web 2.0}.
Web 1.0 adalah "Read-Only" (Situs berita, brosur perusahaan).
Web 2.0 adalah "Read-Write" (Blog, Wiki, Sosial Media). Pengguna bukan lagi konsumen pasif; mereka adalah pembuat konten.

Teknologi di balik revolusi ini adalah \textbf{AJAX} (\textit{Asynchronous JavaScript and XML}).
Sebelum AJAX, setiap kali Anda mengklik sesuatu di web, seluruh halaman harus dimuat ulang (refresh). Itu lambat.
Pada tahun 2004, Google merilis \textbf{Gmail} dan kemudian \textbf{Google Maps}.
Keduanya terasa seperti aplikasi desktop. Anda bisa menggeser peta tanpa reload. Email baru muncul tanpa refresh.
Ini mengubah segalanya. Web menjadi \textbf{Platform Aplikasi}.
JavaScript, bahasa yang dulu diremehkan sebagai "mainan", tiba-tiba menjadi bahasa paling penting di dunia.

Pada tahun 2004, di asrama Harvard, Mark Zuckerberg meluncurkan \textbf{TheFacebook}.
Berbeda dengan MySpace yang kacau dan penuh gambar latar belakang yang norak, Facebook bersih, biru, dan eksklusif (awalnya hanya untuk mahasiswa Harvard). Ia memetakan hubungan sosial dunia nyata ke dalam database.
Facebook mengajarkan Artisan tentang \textbf{Efek Jaringan} (\textit{Network Effect}). Nilai sebuah produk meningkat secara eksponensial seiring dengan bertambahnya jumlah pengguna.

\section{2006 -- 2007: Infrastruktur Awan dan Revolusi Saku}

Tahun 2006 adalah tahun yang paling diremehkan namun paling penting bagi infrastruktur modern.
Amazon, sebuah toko buku online, meluncurkan \textbf{AWS} (\textit{Amazon Web Services}).
Layanan pertama mereka adalah \textbf{S3} (penyimpanan) dan \textbf{EC2} (komputer sewaan).
Sebelum AWS, jika Anda ingin membuat startup, Anda butuh modal \$50.000 untuk membeli server, menyewa rak di data center, dan membayar admin sistem.
Dengan AWS, Anda hanya butuh kartu kredit. Anda bisa menyewa server seharga beberapa sen per jam. Jika gagal, matikan saja servernya. Jika sukses, nyalakan 1.000 server lagi dalam hitungan menit.
Ini adalah \textbf{Demokratisasi Infrastruktur}. AWS memungkinkan Airbnb, Netflix, dan Pinterest untuk lahir tanpa modal infrastruktur raksasa.

Namun, revolusi yang paling kasat mata terjadi pada 9 Januari 2007.
Di panggung Macworld, Steve Jobs memperkenalkan tiga produk:
"Sebuah iPod layar lebar dengan kontrol sentuh."
"Sebuah ponsel revolusioner."
"Sebuah perangkat komunikator internet penerobos."
"Ini bukan tiga perangkat terpisah. Ini satu perangkat. Dan kami menamainya \textbf{iPhone}."

iPhone menghancurkan paradigma komputasi yang ada.
Ia membuang keyboard fisik (BlackBerry) dan stylus (Palm Pilot). Ia menggunakan jari kita sebagai alat penunjuk terbaik di dunia (\textit{Multi-Touch}).
Ia membawa sistem operasi kelas desktop (OS X yang dikecilkan menjadi iOS) ke saku.
Ia membawa browser web sesungguhnya (Safari), bukan WAP yang disederhanakan.

Setahun kemudian (2008), Apple meluncurkan \textbf{App Store}.
Ini membuka gerbang emas bagi pengembang independen. Seorang remaja di kamarnya bisa membuat game \textit{Flappy Bird} dan menjadi jutawan dalam semalam.
Ekonomi Aplikasi (\textit{App Economy}) lahir. Jutaan pekerjaan baru tercipta. Artisan kode kini memiliki pasar global di ujung jari mereka.

Di sisi lain, Google tidak tinggal diam. Mereka membeli \textbf{Android} dan merilisnya sebagai sistem operasi terbuka (\textit{Open Source}) untuk melawan iPhone.
Strategi Google brilian: Berikan OS gratis kepada Samsung, HTC, Motorola, dll. Biarkan mereka membanjiri pasar dengan perangkat murah. Pastikan semua orang menggunakan Google Search dan Google Maps.
Perang iOS vs Android dimulai, membagi dunia menjadi dua kubu hijau dan biru.

\section{2008 -- 2009: Krisis dan Kriptografi}

Di penghujung dekade, dunia dihantam krisis finansial global 2008. Bank-bank besar runtuh. Pemerintah mencetak uang gila-gilaan untuk menalangi mereka (\textit{bailout}). Kepercayaan publik terhadap sistem keuangan hancur.

Pada 31 Oktober 2008, sebuah makalah muncul di milis kriptografi dari seseorang bernama \textbf{Satoshi Nakamoto}.
Judulnya: \textit{"Bitcoin: A Peer-to-Peer Electronic Cash System"}.
Satoshi mengajukan pertanyaan radikal: Bisakah kita memiliki uang digital tanpa bank sentral? Bisakah kita mempercayai matematika alih-alih mempercayai manusia?

Jawabannya adalah \textbf{Blockchain}.
Satoshi memecahkan masalah klasik ilmu komputer: \textit{The Byzantine Generals Problem}. Bagaimana membuat konsensus di jaringan yang tidak terpercaya?
Solusinya adalah \textbf{Proof of Work} (PoW). Penambang (\textit{miners}) harus menghabiskan energi listrik untuk memecahkan teka-teki matematika guna memvalidasi transaksi. Ini membuat serangan terhadap jaringan menjadi sangat mahal secara ekonomi.
Blok pertama (Genesis Block) ditambang pada 3 Januari 2009. Di dalamnya, Satoshi menyisipkan pesan dari koran The Times: \textit{"Chancellor on brink of second bailout for banks."}
Bitcoin bukan hanya teknologi; ia adalah protes politik. Ia adalah deklarasi kemerdekaan moneter.
Bagi Artisan, Blockchain mengajarkan tentang \textbf{Desentralisasi} dan \textbf{Kekekalan} (\textit{Immutability}). Bahwa kode bisa menjadi hukum (\textit{Code is Law}).

Tahun 2009 juga melihat kelahiran teknologi lain yang akan mengubah cara kita menulis kode backend: \textbf{Node.js}.
Ryan Dahl mengambil mesin JavaScript V8 dari browser Chrome dan menjalankannya di server.
Tiba-tiba, JavaScript bisa melakukan segalanya: Frontend dan Backend.
Konsep \textbf{Asynchronous I/O} (\textit{Non-blocking}) memungkinkan Node.js menangani ribuan koneksi bersamaan dengan sangat ringan. Ini sempurna untuk aplikasi \textit{real-time} seperti chat dan game.
Mimpi "Satu Bahasa untuk Segalanya" (Universal JavaScript) mulai terwujud.

\section{Refleksi Dekade: Hidup dalam Aliran}

Dekade 2000-an mengubah ritme kehidupan manusia.
Dulu, kita memiliki "waktu online" dan "waktu offline".
Sekarang, kita selalu online (\textit{Always On}).
Kita bangun tidur dan hal pertama yang kita lakukan adalah mengecek notifikasi. Kita makan sambil memotret makanan untuk Instagram. Kita tersesat dan bertanya pada Waze.

Bagi Artisan, ini adalah tanggung jawab yang berat.
Kode yang kita tulis tidak lagi hanya berjalan di mesin kantor; ia berjalan di saku, di tempat tidur, di meja makan. Ia memengaruhi cara orang berinteraksi, cara orang mencintai, dan cara orang memahami dunia.
Kita telah membangun \textbf{Saraf Digital Global}.
Sekarang pertanyaannya adalah: Apakah saraf ini membuat kita lebih sadar, atau hanya lebih cemas?
Tantangan dekade berikutnya (2010-an) bukan lagi soal konektivitas, tapi soal bagaimana mengelola banjir data yang telah kita ciptakan ini.
