\chapter{The Silicon Revolution (1970 -- 1979)}

Dekade 1970-an adalah saat di mana komputasi "turun ke bumi". Jika dekade sebelumnya adalah tentang gedung-gedung besar yang dipenuhi mainframe, maka 1970-an adalah tentang bagaimana kekuatan tersebut menyusut hingga muat di atas sebuah kepingan silikon kecil. Ini adalah era di mana bahasa pemrograman yang kita gunakan hari ini lahir, dan di mana konsep antarmuka grafis mulai menempa masa depan. Sebagai Artisan, kita melihat 1970-an sebagai dekade di mana "kontrol" benar-benar menjadi demokratis melalui kelahiran mikroprosesor.

\section{1970: Unix, Pascal, dan Memori Silikon}

Tahun ini menandai awal dari era Unix yang resmi serta standarisasi metode pemrograman yang lebih disiplin.

\begin{description}
    \item[Unix Official Release] \textit{Saat pertama kali dibuat} di Bell Labs setelah kegagalan Multics, Unix dirancang untuk menjadi sistem yang kecil dan bisa dipahami oleh satu orang. Ken Thompson dan Dennis Ritchie menciptakan filosofi di mana setiap alat melakukan satu hal dengan sempurna dan berkomunikasi melalui pipa teks (*pipes*).

    \textit{Pada saat buku ini dibuat di tahun 2026}, kita merayakan tahun 1970 sebagai "Epoch Time" bagi dunia sistem operasi. Sebagai Artisan, Unix 1970 mengajarkan kita tentang "kekuatan dari kesederhanaan yang disiplin". Pengaruh Unix tidak datang dari fitur-fitur yang gemuk, tapi dari keanggunan arsitekturalnya yang memungkinkan ia di-porting ke hampir semua mesin di dunia. Inilah *The Art of Influence*: menjadi fondasi yang begitu esensial sehingga dunia tidak bisa berjalan tanpanya.

    \item[Pascal (Niklaus Wirth)] \textit{Saat pertama kali dibuat}, Pascal bertujuan untuk mengajarkan pemrograman yang terstruktur dan disiplin. Wirth merancang bahasa ini agar efisien baik untuk kompilasi maupun saat dijalankan, sambil memaksa programmer untuk berpikir secara logis dan teratur melalui pengetikan yang kuat (*strong typing*).

    \textit{Pada saat buku ini dibuat}, kita melihat Pascal sebagai bahasa yang membentuk cara kita mengajarkan ilmu komputer selama dekade-dekade berikutnya. Sebagai Artisan, Pascal mengajari kita bahwa disiplin bukanlah penghalang kreativitas, melainkan penjaga kualitas. Dengan membatasi kebebasan yang ceroboh, kita justru mampu membangun sistem yang lebih besar dan lebih andal.

    \item[Intel 1103 (DRAM)] \textit{Saat pertama kali dibuat}, ini adalah kepingan memori akses acak dinamis (DRAM) pertama yang dipasarkan secara massal. Ia mulai menggantikan memori inti magnetik yang besar dan mahal.

    \textit{Pada saat buku ini dibuat}, DRAM adalah jantung dari setiap perangkat digital kita. Sebagai Artisan, transisi di tahun 1970 ini mengingatkan kita bahwa setiap karya besar membutuhkan material yang efisien. Silikon memungkinkan kita untuk menyimpan lebih banyak pikiran (data) dalam ruang yang lebih kecil.
\end{description}

\section{1971: Ledakan Mikroprosesor dan Komunikasi Pribadi}

Tahun di mana komputer benar-benar menjadi mikro, mengubah lintasan sejarah manusia selamanya.

\begin{description}
    \item[Intel 4004 (The First Microprocessor)] \textit{Saat pertama kali dibuat} oleh Marcian "Ted" Hoff, Federico Faggin, dan Masatoshi Shima, Intel 4004 adalah CPU lengkap pertama yang diletakkan di atas satu kepingan silikon. Ukurannya hanya sebesar kuku manusia, namun memiliki kekuatan yang sama dengan ENIAC yang memenuhi ruangan di tahun 1946.

    \textit{Pada saat buku ini dibuat}, setiap chip di tahun 2026 adalah "keturunan" dari 4004. Sebagai Artisan, 4004 mengajarkan kita tentang "kondensasi kekuatan". Pengaruh terbesar seringkali datang dalam bentuk yang paling kecil. Dengan mengecilkan fisik, kita justru memperluas jangkauan ide. Faggin dan timnya mengajari kita bahwa untuk mengarahkan arus dunia, terkadang kita harus memampatkan kompleksitas menjadi satu titik fokus yang tajam.

    \item[The First Email (Ray Tomlinson)] \textit{Saat pertama kali dibuat}, Tomlinson mengirim pesan antara dua komputer yang berada berdampingan, menggunakan simbol "@" untuk memisahkan nama pengguna dari nama mesin.

    \textit{Pada saat buku ini dibuat}, email telah menjadi urat nadi komunikasi global. Sebagai Artisan, keputusan sederhana Tomlinson menggunakan simbol "@" adalah contoh klasik dari "pengaruh desain yang abadi". Terkadang, solusi yang paling sederhana untuk masalah pengalamatan adalah solusi yang akan digunakan oleh milyaran orang puluhan tahun kemudian.
\end{description}

\section{1972: Lahirnya Bahasa C dan Hiburan Digital}

Ini adalah tahun di mana Dennis Ritchie memberikan Artisan alat yang paling tajam dalam gudang senjata mereka.

\begin{description}
    \item[The C Programming Language] \textit{Saat pertama kali dibuat} di Bell Labs untuk menulis ulang Unix, bahasa C dirancang untuk menjadi bahasa tingkat menengah yang memberikan kekuatan manipulasi perangkat keras seperti assembly, namun dengan keanggunan struktur bahasa tingkat tinggi.

    \textit{Pada saat buku ini dibuat}, C tetap menjadi "bahasa para dewa". Hampir semua sistem operasi modern, dari Linux hingga Windows, ditulis dalam C atau keturunannya. Sebagai Artisan, C adalah simbol dari "kontrol tanpa kompromi". Ia menuntut disiplin yang sangat tinggi dari penggunanya karena ia tidak memiliki jaring pengaman otomatis. C mengajari kita bahwa pengaruh yang sejati datang dari pemahaman mendalam tentang bagaimana mesin bekerja. Kita tidak mengarahkan dunia dengan menjauh dari mesin, tapi dengan menjadi satu dengannya melalui bahasa yang jujur.

    \item[Pong (Atari)] \textit{Saat pertama kali dibuat} oleh Nolan Bushnell dan Al Alcorn, Pong adalah video game komersial pertama yang sukses besar. Ia membawa komputasi ke dalam ruang tamu dan tempat hiburan sebagai alat bermain.

    \textit{Pada saat buku ini dibuat}, industri game telah melampaui industri film dalam hal pendapatan. Sebagai Artisan, Pong mengajarkan kita bahwa teknologi mencapai puncaknya ketika ia bisa menyentuh sisi manusiawi kita—keinginan untuk bermain dan bersosialisasi. Pengaruh tidak selalu harus berupa hal yang serius dan berat; kegembiraan adalah salah satu cara paling efektif untuk menyebarkan inovasi.
\end{description}

\section{1973: Fondasi GUI dan Jaringan Modern}

Tahun ini membawa kita ke pusat penelitian Xerox PARC, di mana masa depan sedang dibangun jauh sebelum dunia siap menerimanya.

\begin{description}
    \item[Xerox Alto (The First GUI Computer)] \textit{Saat pertama kali dibuat}, Alto adalah komputer pertama yang menggunakan antarmuka grafis (GUI), mouse, dan sistem WYSIWYG (*What You See Is What You Get*). Ini adalah komputer yang dirancang untuk satu orang, dengan layar vertikal yang menyerupai selembar kertas.

    \textit{Pada saat buku ini dibuat}, setiap smartphone dan laptop kita adalah warisan dari Alto. Sebagai Artisan, Alto mengajarkan kita tentang "visi yang mendahului zaman". Xerox PARC membangun masa depan yang tidak bisa mereka jual secara massal saat itu, namun mereka memberikan cetak biru bagi setiap perusahaan teknologi di dekade-dekade mendatang. Pengaruh sejati seringkali tidak terlihat dalam angka penjualan saat ini, tetapi dalam seberapa banyak ide Anda dicuri dan diadaptasi oleh orang lain untuk merubah dunia.

    \item[Ethernet (Robert Metcalfe)] \textit{Saat pertama kali dibuat} di Xerox PARC, Ethernet adalah cara untuk menghubungkan komputer-komputer di dalam satu kantor agar bisa berbagi printer dan data dengan kecepatan tinggi.

    \textit{Pada saat buku ini dibuat}, Ethernet tetap menjadi standar untuk jaringan area lokal (LAN). Sebagai Artisan, Metcalfe mengajari kita tentang "hukum jaringan" (\textit{Metcalfe's Law}): nilai sebuah jaringan meningkat secara eksponensial dengan jumlah penggunanya. Pengaruh kita sebagai Artisan akan berlipat ganda ketika kita membangun sistem yang menghubungkan orang lain. Kerjasama adalah bentuk tertinggi dari rekayasa sosial.
\end{description}

\section{1974: Ledakan PC Pertama dan Bahasa Query (Altair \& SQL)}

Tahun ini membawa komputer pertama yang bisa dirakit sendiri di rumah, memberikan kekuatan komputasi ke tangan individu.

\begin{description}
    \item[MITS Altair 8800] \textit{Saat pertama kali dibuat} oleh Ed Roberts sebagai kit yang harus dirakit sendiri, Altair 8800 tidak memiliki keyboard, layar, atau printer. Ia hanya memiliki sakelar dan lampu di panel depan. Namun, ia adalah komputer pribadi pertama yang layak secara ekonomi, menggunakan chip Intel 8080.

    \textit{Pada saat buku ini dibuat}, Altair 8800 dihormati sebagai percikan yang menyalakan api revolusi PC. Sebagai Artisan, Altair mengajarkan kita bahwa "kesederhanaan mentah" bisa menjadi katalisator bagi ekosistem yang masif. Tanpa Altair, kita mungkin tidak akan memiliki antarmuka yang kita gunakan hari ini. Pengaruh Roberts datang dari keberaniannya untuk menciptakan pasar yang belum pernah ada sebelumnya.

    \item[SQL (Structured Query Language)] \textit{Saat pertama kali dibuat} oleh Donald Chamberlin dan Raymond Boyce di IBM (awalnya dinamakan SEQUEL), SQL dirancang untuk memanipulasi dan mengambil data dari database relasional berdasarkan model Edgar F. Codd.

    \textit{Pada saat buku ini dibuat}, SQL tetap menjadi bahasa standar untuk berinteraksi dengan data di seluruh dunia. Sebagai Artisan, SQL mengajari kita tentang "abstraksi deklaratif". Kita tidak peduli *bagaimana* mesin mengambil data, kita hanya peduli *apa* yang kita inginkan. Inilah bentuk pengaruh yang elegan: memberikan bahasa yang memungkinkan manusia berbicara dengan data secara intuitif.
\end{description}

\section{1975: Kelahiran Microsoft dan Budaya Komunitas}

Komputer pribadi mulai memiliki perangkat lunak yang bisa diandalkan, dan komunitas peminat mulai terorganisir.

\begin{description}
    \item[The Founding of Microsoft] \textit{Saat pertama kali dibuat} oleh Bill Gates dan Paul Allen (awalnya "Micro-Soft"), fokus utama mereka adalah menciptakan interpreter BASIC untuk Altair 8800. Mereka menyadari bahwa perangkat keras tanpa perangkat lunak hanyalah besi mati.

    \textit{Pada saat buku ini dibuat}, Microsoft telah menjadi salah satu perusahaan paling berpengaruh dalam sejarah. Sebagai Artisan, Gates mengajari kita tentang "nilai strategis dari perangkat lunak". Pengaruh yang abadi seringkali tidak terletak pada besi yang kita sentuh, tapi pada logika yang menggerakkannya. Keberanian Microsoft untuk memfokuskan diri hanya pada perangkat lunak merubah peta kekuatan industri teknologi selamanya.

    \item[Homebrew Computer Club] \textit{Saat pertama kali dibuat} di Menlo Park, California, klub ini adalah tempat berkumpulnya para penggemar elektronik yang berbagi ide, desain, dan kode. Di sinilah Steve Wozniak dan Steve Jobs mulai memperkenalkan ide-ide mereka.

    \textit{Pada saat buku ini dibuat}, kita melihat komunitas ini sebagai embrio dari Silicon Valley. Sebagai Artisan, Homebrew mengajari kita tentang kekuatan "kolaborasi terbuka". Pengaruh terbesar seringkali lahir dari diskusi santai di antara para pengrajin yang memiliki gairah yang sama. Berbagi ilmu adalah cara Artisan untuk mempercepat putaran roda kemajuan.
\end{description}

\section{1976: Kebangkitan Apple dan Supercomputing}

Tahun di mana estetika mulai masuk ke dalam desain komputer, dan kekuatan perhitungan mencapai level super.

\begin{description}
    \item[Apple I \& The Birth of Apple Computer] \textit{Saat pertama kali dibuat} oleh Steve Wozniak sebagai sebuah papan sirkuit tunggal yang dirakit dengan tangan, Apple I memperkenalkan konsep komputer yang bisa dihubungkan langsung ke TV dan keyboard. Steve Jobs menyadari potensi komersialnya dan mendirikan Apple Computer Company.

    \textit{Pada saat buku ini dibuat}, Apple adalah simbol dari perpaduan antara teknologi dan seni. Sebagai Artisan, Wozniak mengajari kita tentang "kejeniusan teknik", sementara Jobs mengajari kita tentang "pengemasan visi". Pengaruh Apple I terletak pada bagaimana ia mulai mengubah komputer dari "alat hobi" menjadi "produk konsumen". Inilah *The Art of Influence*: membuat teknologi menjadi sesuatu yang diinginkan, bukan hanya dibutuhkan.

    \item[Cray-1 (Supercomputer)] \textit{Saat pertama kali dibuat} oleh Seymour Cray, Cray-1 adalah komputer tercepat di dunia dengan desain unik berbentuk huruf "C" untuk meminimalkan panjang kabel dan meningkatkan kecepatan.

    \textit{Pada saat buku ini dibuat}, Cray-1 dianggap sebagai mahakarya desain perangkat keras. Sebagai Artisan, Seymour Cray mengajari kita tentang "inovasi tanpa kompromi". Ia menantang batas-batas fisika untuk mencapai performa yang mustahil. Baginya, desain bukan hanya soal tampilan, tapi soal fungsi yang sangat optimal. Inilah integritas seorang Artisan supercomputer.
\end{description}

\section{1977: Trinitas Komputer Pribadi (Apple II, TRS-80, \& PET)}

Tahun ini adalah tahun di mana komputer pribadi benar-benar menjadi barang konsumsi massal yang rapi dan siap pakai.

\begin{description}
    \item[The 1977 Trinity] \textit{Saat pertama kali dibuat}, Apple II, TRS-80, dan Commodore PET dirilis sebagai komputer yang sudah dirakit lengkap, memiliki casing plastik (bukan lagi besi kasar atau kayu), dan bisa langsung digunakan setelah dikeluarkan dari kotak.

    \textit{Pada saat buku ini dibuat}, 1977 diakui sebagai tahun ledakan komputasi rumah tangga. Sebagai Artisan, "Trinitas 1977" mengajari kita tentang pentingnya "pengalaman pengguna" (*user experience*). Pengaruh sejati dicapai ketika teknologi tidak lagi menantang pengguna untuk merakitnya, tapi mengundang pengguna untuk berkreasi dengannya. Apple II, khususnya, dengan casing plastiknya yang ramah, merubah persepsi komputer dari alat militer menjadi alat rumah tangga.

    \item[The Apple II's Color Graphics] \textit{Saat pertama kali dibuat}, Apple II adalah salah satu komputer pribadi pertama yang mampu menampilkan grafik berwarna. Wozniak menggunakan trik sirkuit yang jenius untuk menghasilkan warna tanpa chip khusus yang mahal.

    \textit{Pada saat buku ini dibuat}, kita memahami bahwa warna dan visual adalah bahasa universal. Sebagai Artisan, Wozniak mengajari kita bahwa keterbatasan perangkat keras bisa diatasi dengan kejeniusan perangkat lunak dan desain sirkuit. Kita mengarahkan dunia dengan menciptakan keindahan dalam keterbatasan.
\end{description}

\section{1978: Arsitektur x86 dan Revolusi VAX}

Fondasi dari arsitektur PC modern dan minikomputer tingkat lanjut mulai ditetapkan.

\begin{description}
    \item[Intel 8086] \textit{Saat pertama kali dibuat}, prosesor 16-bit ini memperkenalkan arsitektur x86. Meskipun awalnya tidak dianggap sebagai revolusi besar, ia terpilih oleh IBM untuk PC mereka di tahun 1981.

    \textit{Pada saat buku ini dibuat}, x86 tetap menjadi arsitektur dominan di dunia PC dan server. Sebagai Artisan, 8086 mengajarkan kita tentang "momentum sejarah". Terkadang, menjadi yang "cukup baik" dan berada di tempat yang tepat pada saat yang tepat (desain yang diadopsi IBM) memberikan pengaruh yang lebih luas daripada menjadi yang paling canggih secara teoritis. Arsitektur adalah tentang ekosistem.

    \item[VAX-11/780] \textit{Saat pertama kali dibuat} oleh DEC, VAX adalah minikomputer 32-bit yang sangat populer dan menjadi standar emas di universitas dan pusat riset. Ia dikenal karena set instruksinya yang kaya (CISC).

    \textit{Pada saat buku ini dibuat}, VAX dikenang sebagai mesin impian bagi para programmer sistem. Sebagai Artisan, VAX mengajari kita tentang "kemewahan fungsional". Memberikan set alat yang lengkap kepada pengguna memungkinkan mereka untuk fokus pada pembangunan aplikasi yang kompleks tanpa harus memikirkan batasan instruksi tingkat rendah.
\end{description}

\section{1979: Spreadsheet, Database, dan Mikroprosesor 16/32-bit}

Dekade ini ditutup dengan perangkat lunak yang akan mendefinisikan produktivitas bisnis selama puluhan tahun.

\begin{description}
    \item[Visicalc (First Spreadsheet)] \textit{Saat pertama kali dibuat} oleh Dan Bricklin dan Bob Frankston untuk Apple II, Visicalc merubah komputer dari mainan hobi menjadi alat bisnis yang esensial. Seorang akuntan bisa melakukan perhitungan dalam menit yang sebelumnya butuh waktu berhari-hari.

    \textit{Pada saat buku ini dibuat}, kita melihat Visicalc sebagai "Killer App" pertama. Sebagai Artisan, Visicalc mengajari kita bahwa sebuah teknologi akan meledak jika ia bisa memecahkan masalah praktis yang paling membosankan bagi manusia. Pengaruh datang dari efisiensi yang nyata bagi kehidupan sehari-hari.

    \item[Oracle (V2)] \textit{Saat pertama kali dibuat} oleh Larry Ellison dan timnya, ini adalah implementas komersial pertama dari database relasional SQL milik IBM.

    \textit{Pada saat buku ini dibuat}, manajemen data relasional adalah tulang punggung internet. Sebagai Artisan, Ellison mengajari kita tentang "agresivitas visoner". Ia mengambil ide riset (SQL IBM) dan menjadikannya produk yang menguasai pasar. Pengaruh seringkali datang kepada mereka yang berani merealisasikan ide orang lain menjadi kenyataan fungsional.

    \item[Motorola 68000] \textit{Saat pertama kali dibuat}, prosesor 16/32-bit ini begitu canggih sehingga ia menjadi jantung dari Apple Macintosh, workstation Sun, dan banyak komputer revolusioner lainnya.

    \textit{Pada saat buku ini dibuat}, 68000 dianggap sebagai mahakarya desain mikroprosesor dengan set instruksi yang sangat ortogonal dan elegan. Sebagai Artisan, 68000 mengajari kita tentang "keindahan struktural". Sebuah desain yang bersih akan selalu menarik minat para pengrajin hebat lainnya untuk membangun sesuatu di atasnya.
\end{description}

\section{Atmosfer Era: Dari Menara Gading ke Garasi}

Untuk memahami 1970-an, kita harus melihatnya sebagai era pemberontakan intelektual. Jika 1960-an adalah tentang protes sosial, 1970-an adalah tentang personifikasi kekuatan. Slogan "Computer Power to the People" bukan sekadar jargon; itu adalah misi teknis.

\textit{Saat pertama kali dibuat}, suasana ini melahirkan etos "Garasi". Komputer tidak lagi membutuhkan tim teknisi berjas putih dan ruangan berpendingin udara. Ia bisa dirakit di atas meja kayu di garasi rumah, di sela-sela waktu minum bir dan diskusi filosofis. Ada rasa urgensi untuk memiliki akses mandiri terhadap informasi.

\textit{Pada saat buku ini dibuat di tahun 2026}, kita melihat gema yang sama dalam gerakan *Individual Sovereignty* dan teknologi terdesentralisasi. Sebagai Artisan, kita harus menangkap semangat "kemandirian" dari tahun 1970-an. Pengaruh sejati tidak datang dari bergantung pada infrastruktur raksasa, tapi dari kemampuan untuk membangun dan menguasai alat kita sendiri. 1970-an mengajari kita bahwa revolusi yang sesungguhnya terjadi ketika alat yang paling kuat di dunia jatuh ke tangan orang-orang yang paling bebas di dunia.

\section{Disiplin Sang Artisan: Keaslian di Tengah Abstraksi}

Pelajaran terbesar bagi seorang Artisan dari dekade ini adalah pentingnya menjaga hubungan dengan "besi" (hardware) sambil kita terus naik ke tingkat abstraksi yang lebih tinggi.

\textit{Saat pertama kali dibuat}, transisi dari Assembly ke C adalah momen disiplin. Para Artisan harus belajar untuk tetap memiliki insting tentang penggunaan memori dan siklus CPU, meskipun bahasa pemrograman mulai menyembunyikan detail tersebut. Mereka tidak boleh menjadi malas.

\textit{Pada saat buku ini dibuat}, di era sistem yang sangat terabstraksi seperti *Cloud* dan *Framework* raksasa, kita menghadapi risiko kehilangan "jiwa teknis". Pelajaran dari para pengrajin tahun 1970-an adalah: gunakan abstraksi untuk kecepatan, tapi jangan pernah kehilangan kemampuan untuk turun ke level bit jika sistem tersebut rusak. Disiplin kita di tahun 2026 adalah untuk menjadi "Full-Stack Artisan" yang sejati—yang memahami tumpukan teknologi mulai dari gerbang logika hingga antarmuka pengguna, karena di situlah kontrol yang sejati berada.

\section{Refleksi Dekade: Fondasi Dunia Modern}

Dekade 1970-an ditutup dengan dunia yang sudah memiliki semua elemen dasar untuk revolusi informasi yang akan datang di dekade 1980-an.

\begin{description}
    \item[Warisan Sang Artisan] \textit{Saat pertama kali dibuat}, dekade ini memberikan kita Mikroprosesor, Bahasa C, GUI, Spreadsheet, Database Relasional, dan Personal Computer. Ini adalah dekade yang meletakkan fondasi permanen bagi peradaban digital.

    \textit{Pada saat buku ini dibuat}, kita menyadari bahwa kita masih hidup di dalam "Dunia yang dibangun tahun 1970-an". Hampir semua yang kita anggap sebagai teknologi mutakhir hari ini hanyalah penyempurnaan dari visi yang lahir di dekade ini. Sebagai Artisan, kita menghormati pionir-pionir ini dengan cara terus mengasah pisau logika kita menggunakan alat-alat (seperti C dan Unix) yang mereka berikan. Kita tidak hanya menggunakan teknologi; kita sedang menjaga api kreativitas garasi tetap menyala di era korporasi raksasa.
\end{description}
