\chapter{The Silicon Revolution (1970 -- 1979)}

Mungkin tidak ada dekade dalam sejarah manusia yang mengubah nasib individu secara lebih radikal daripada tahun 1970-an.
Di awal dekade, "komputer" adalah benda mitologis yang hanya dilihat oleh segelintir ilmuwan berjas putih di balik pintu tertutup. Di akhir dekade, komputer adalah benda yang bisa Anda beli di toko elektronik, bawa pulang, dan letakkan di meja makan Anda.

Ini adalah dekade \textbf{Desentralisasi Kekuasaan}.
Kekuasaan komputasi, yang sebelumnya dimonopoli oleh pemerintah dan korporasi raksasa, tiba-tiba dihancurkan menjadi kepingan-kepingan silikon kecil dan dibagikan kepada rakyat jelata.

Dua revolusi terjadi secara paralel:
1.  \textbf{Revolusi Keras}: Penemuan mikroprosesor memampatkan mainframe seukuran gudang menjadi chip seukuran kuku jari.
2.  \textbf{Revolusi Lunak}: Penemuan bahasa C dan sistem operasi UNIX memberikan kita alat untuk mengontrol silikon tersebut dengan presisi dan portabilitas yang belum pernah ada sebelumnya.

Bagi seorang Artisan di tahun 2026, dekade ini mengajarkan tentang \textbf{Kemandirian}. Tahun 70-an adalah masa di mana para peretas garasi berhenti meminta izin dan mulai membangun masa depan mereka sendiri dengan solder dan kode assembly. Semangat \textit{Do It Yourself} (DIY) ini adalah warisan abadi yang masih kita rasakan setiap kali kita melakukan `npm init` atau merakit PC gaming.

\section{1970 -- 1971: Alam Semesta dalam Kuku Jari}

Pada akhir tahun 60-an, sebuah perusahaan kalkulator Jepang bernama Busicom datang ke Intel (yang saat itu baru berdiri dan fokus pada memori). Mereka menginginkan 12 chip khusus untuk kalkulator baru mereka.
Insinyur Intel, \textbf{Ted Hoff}, melihat pesanan ini dan berpikir: "Ini gila. Membuat 12 chip berbeda itu mahal dan tidak efisien. Mengapa kita tidak membuat \textit{satu} chip yang bisa melakukan \textit{semua} fungsi itu hanya dengan mengubah programnya?"

Ide ini—logika umum yang diprogram (\textit{general-purpose programmable logic}) dalam satu chip—melahirkan \textbf{Intel 4004} pada tahun 1971.
Ini adalah \textbf{Mikroprosesor} pertama di dunia.
Dengan 2.300 transistor, CPU 4-bit ini memiliki kekuatan komputasi yang setara dengan ENIAC (1946) yang beratnya 30 ton. Tapi 4004 hanya seukuran kuku jari kelingking.
Hukum Moore terbukti benar secara spektakuler. Biaya komputasi runtuh. Tiba-tiba, kita bisa menaruh kecerdasan digital di mana saja: di lampu lalu lintas, di mesin cuci, dan tentu saja, di komputer pribadi.

Bagi Artisan, Intel 4004 mengajarkan prinsip \textbf{Abstraksi Fisik}. Kita tidak perlu lagi merangkai ribuan kabel untuk membuat logika; kita cukup menulis instruksi perangkat lunak pada sepotong silikon standar. Perangkat keras menjadi kanvas kosong; perangkat lunak menjadi catnya.

\section{1972 -- 1973: Bahasa Para Dewa dan Masa Depan yang Hilang}

Sementara Intel mengecilkan perangkat keras, di Bell Labs, \textbf{Dennis Ritchie} sedang menyempurnakan alat untuk menguasainya.
Ritchie ingin menulis ulang sistem operasi UNIX agar bisa dipindahkan (\textit{portable}) antar mesin yang berbeda. Tapi bahasa yang ada saat itu tidak cukup baik. Assembly terlalu terikat pada mesin tertentu. B (bahasa pendahulu) terlalu lambat.
Jadi, Ritchie menciptakan \textbf{C}.

Bahasa C (1972) adalah mahakarya keseimbangan. Ia cukup rendah (\textit{low-level}) untuk mengakses memori fisik dan register mesin secara langsung, tetapi cukup tinggi (\textit{high-level}) untuk memiliki struktur data, fungsi, dan logika manusiawi.
C menjadi "Pedang Excalibur" bagi para Artisan sistem. Hingga hari ini, tahun 2026, kernel Linux, Windows, macOS, dan bahkan infrastruktur internet, semuanya ditulis dalam C (atau turunannya). C adalah \textit{Lingua Franca} komputasi. Ia mengajarkan kita bahwa \textbf{Kontrol} dan \textbf{Efisiensi} adalah nilai abadi. C tidak memegang tangan Anda; ia membiarkan Anda melakukan apa saja, termasuk menghancurkan sistem Anda sendiri (\textit{segmentation fault}). Itu adalah bahasa untuk orang dewasa.

Sementara itu, di pesisir barat, di Xerox PARC (Palo Alto Research Center), para peneliti sedang membangun mesin waktu.
Mereka menciptakan \textbf{Xerox Alto} (1973).
Komputer ini memiliki semua yang kita anggap modern lima puluh tahun kemudian:
\begin{itemize}
    \item Antarmuka Grafis (GUI) dengan jendela dan ikon.
    \item Mouse untuk menunjuk dan klik.
    \item Jaringan Ethernet untuk menghubungkan komputer.
    \item Konsep \textit{Object-Oriented Programming} (melalui Smalltalk).
    \item Editor dokumen \textit{WYSIWYG} (What You See Is What You Get).
\end{itemize}

Xerox Alto adalah komputer personal yang sempurna... yang tidak pernah dijual.
Manajemen Xerox di New York, yang bisnis utamanya adalah mesin fotokopi, tidak melihat nilai dari "mainan" ini. "Mengapa orang butuh layar grafis untuk mengetik surat?" tanya mereka.
Mereka membiarkan penemuan triliunan dolar ini berdebu di laboratorium.

Ini adalah pelajaran pahit namun penting bagi Artisan: \textbf{Inovasi Teknis Saja Tidak Cukup}. Anda membutuhkan \textbf{Visi Bisnis} untuk membawa inovasi tersebut ke dunia. Xerox Alto menjadi "Masa Depan yang Hilang", menunggu untuk ditemukan kembali oleh seseorang yang memiliki visi tersebut (Steve Jobs).

\section{1974 -- 1976: Percikan Api di Garasi}

Pada Januari 1975, majalah \textit{Popular Electronics} memajang sebuah kotak biru dengan lampu kedip-kedip di sampulnya: \textbf{Altair 8800}.
Ini adalah "komputer" pertama yang bisa dibeli oleh orang biasa (seharga \$397 dalam bentuk kit). Tidak ada keyboard. Tidak ada layar. Anda memprogramnya dengan memutar sakelar (\textit{switches}) dan membaca hasilnya lewat lampu LED.

Bagi orang awam, itu sampah. Tapi bagi kaum \textit{hacker} dan hobiis elektronik, itu adalah cawan suci.
Di Harvard, seorang mahasiswa bernama \textbf{Bill Gates} dan temannya \textbf{Paul Allen} melihat majalah itu. Mereka sadar: "Revolusi dimulai tanpa kita!"
Mereka menelepon MITS (pembuat Altair) dan berbohong: "Kami punya interpreter BASIC untuk mesin Anda."
Padahal mereka belum punya apa-apa. Dan mereka bahkan tidak punya mesin Altair.
Selama 8 minggu berikutnya, mereka melakukan \textit{coding marathon} yang legendaris, menulis kode mesin di atas kertas, mensimulasikan CPU Altair di komputer kampus.
Ketika Paul Allen terbang ke Albuquerque untuk mendemonstrasikan kode tersebut, itu adalah pertama kalinya kode itu dijalankan di mesin asli.
Dan itu berhasil.
\textbf{Microsoft} lahir. Misi mereka: "Sebuah komputer di setiap meja dan di setiap rumah, menjalankan perangkat lunak Microsoft."

Di Silicon Valley, semangat yang sama membakar \textbf{Steve Wozniak}.
Wozniak (The Woz) adalah jenius elektronik murni. Ia ingin membuat komputer sendiri karena ia tidak mampu membeli Altair. Ia merancang papan sirkuit yang jauh lebih elegan, menggunakan chip MOS 6502 yang murah. Ia menambahkan keyboard (karena ia suka mengetik) dan kemampuan untuk terhubung ke TV.
Temannya, \textbf{Steve Jobs}, melihat papan sirkuit itu. Jobs tidak mengerti sirkuit sebaik Woz, tapi ia mengerti manusia. Ia berkata: "Kita bisa menjual ini."
Pada 1 April 1976, \textbf{Apple Computer} lahir.
Produk pertama mereka, Apple I, dirakit dengan tangan di garasi orang tua Jobs.

Tahun 1976 juga menyaksikan kelahiran \textbf{Cray-1}, superkomputer paling ikonik di dunia. Didesain oleh Seymour Cray, mesin berbentuk huruf "C" ini (untuk meminimalkan panjang kabel) adalah monster komputasi vektor. Cray mengajarkan Artisan tentang \textbf{Kinerja Melalui Desain Fisik}. Ia tidak hanya memikirkan logika; ia memikirkan panas, listrik, dan kecepatan cahaya dalam kabel.

\section{1977 -- 1979: Trinitas dan Aplikasi Pembunuh}

Tahun 1977 dikenal sebagai tahun "Trinitas 1977". Tiga komputer yang sudah jadi (bukan kit) dirilis ke pasar massal:
1.  \textbf{Apple II}: Berwarna, memiliki grafis, ekspandabel, dan didesain dengan casing plastik beige yang ramah rumah tangga.
2.  \textbf{Commodore PET}: Komputer \textit{all-in-one} dengan layar monokrom terintegrasi.
3.  \textbf{TRS-80}: Dijual melalui jaringan toko Radio Shack yang masif.

Komputer telah tiba di ruang tamu. Tapi pertanyaan besarnya tetap: "Untuk apa?"
Orang membeli Apple II untuk bermain game, belajar BASIC, atau sekadar gaya. Tapi belum ada alasan \textit{ekonomi} yang kuat untuk memilikinya.

Jawabannya datang pada tahun 1979, bukan dari pembuat hardware, tapi dari seorang mahasiswa Harvard Business School bernama \textbf{Dan Bricklin}.
Saat duduk di kelas akuntansi, Bricklin melihat dosennya menghapus dan menulis ulang angka di papan tulis berulang kali karena satu kesalahan hitung di awal.
Bricklin berimajinasi: "Bagaimana jika papan tulis itu elektronik? Bagaimana jika saya ubah satu angka, dan semua angka lain yang berhubungan ikut berubah otomatis?"

Ia menciptakan \textbf{VisiCalc} (\textit{Visible Calculator}).
Ini adalah \textbf{Spreadsheet} elektronik pertama di dunia.
Dampaknya instan dan masif.
Seorang akuntan yang butuh 20 jam seminggu untuk melakukan proyeksi keuangan, kini bisa melakukannya dalam 15 menit. Ia bisa melakukan skenario "What-If" ("Bagaimana jika bunga naik 1\%?") dalam detik.
Tiba-tiba, Apple II bukan lagi mainan \$2.000. Ia adalah mesin pencetak uang. Orang-orang masuk ke toko komputer dan bertanya: "Saya minta VisiCalc, dan tolong berikan komputer apa saja yang bisa menjalankannya."

VisiCalc adalah \textbf{Killer App} pertama.
Ia mengajarkan pelajaran abadi bagi Artisan: \textbf{Perangkat Lunaklah yang Menjual Perangkat Keras}. Nilai sebuah teknologi bukan pada spesifikasinya (berapa MHz, berapa RAM), tetapi pada masalah manusia apa yang bisa ia selesaikan. VisiCalc mengubah komputer dari hobi menjadi kebutuhan bisnis mutlak.

Di akhir dekade, pada 1979, Atari merilis \textbf{Atari 400/800}, membawa chip grafis khusus (ANTIC dan CTIA) ke rumah. Ini adalah cikal bakal konsep GPU dan coprocessor. Sementara itu, di dunia game arcade, \textbf{Space Invaders} (1978) menyebabkan kelangkaan koin 100-yen di Jepang. Budaya digital mulai merasuk ke dalam budaya pop.

\section{Refleksi Dekade: Kedaulatan Individu}

Jika kita melihat kembali tahun 1970-an, kita melihat pola yang jelas: \textbf{Pemberdayaan}.
Teknologi bergerak dari tangan segelintir elit (imam besar mainframe) ke tangan individu (hacker garasi, akuntan, anak-anak).

Artisan tahun 70-an—Wozniak, Gates, Ritchie, Bricklin—adalah pahlawan pola dasar (\textit{archetypal heroes}) kita. Mereka tidak menunggu izin dari IBM. Mereka tidak menunggu dana riset pemerintah. Mereka melihat alat-alat baru (mikroprosesor) dan membangun masa depan dengan tangan mereka sendiri.

Warisan mereka bagi kita di tahun 2026 adalah \textbf{Kebebasan Mencipta}.
Kita memiliki komputer (laptop/HP) yang jutaan kali lebih kuat dari Altair atau Apple II. Kita memiliki alat (IDE, Compiler, Cloud) yang jauh lebih canggih dari BASIC atau VisiCalc.
Pertanyaannya adalah: Apakah kita memiliki \textit{semangat} yang sama dengan mereka? Apakah kita berani membangun sesuatu yang baru di "garasi" kita (kamar tidur kita), atau kita hanya menjadi konsumen pasif?
Setiap kali Anda menulis baris kode pertama untuk proyek sampingan Anda, Anda sedang menyalakan kembali api yang dinyalakan di garasi Los Altos pada tahun 1976. Jangan biarkan api itu padam.
