\chapter{The Bloom of Abstraction (1950 -- 1959)}

Jika dekade 1940-an adalah tentang membangun "tubuh" elektronik yang kasar dan panas, maka 1950-an adalah saat di mana komputasi mulai menemukan "suaranya". Ini adalah dekade di mana kita berhenti berbicara dalam biner mentah dan mulai membangun jembatan bahasa antara manusia dan mesin.

Bagi seorang Artisan, perubahan terbesar di dekade ini bukanlah pada kecepatan prosesor, melainkan pada \textit{Level of Abstraction}. Kita bergerak dari "menyolder kabel" menuju "menulis simbol". Inilah era di mana \textit{Software Engineering} mulai memisahkan diri dari \textit{Electrical Engineering}.

\section{1950 -- 1952: Masalah Babel dan Sang Penterjemah}

Dekade ini dibuka dengan kebingungan. Setiap komputer—baik itu UNIVAC, EDSAC, atau mesin \textit{custom} lainnya—memiliki bahasa mesinnya sendiri yang unik. Seorang programmer UNIVAC tidak bisa berbicara dengan mesin IBM. Dunia komputasi adalah Menara Babel yang terfragmentasi.

Di tengah kekacauan inilah \textbf{Grace Hopper} muncul dengan solusi yang radikal. Bekerja pada UNIVAC, ia menyadari bahwa manusia buruk dalam mengingat angka biner, tetapi hebat dalam mengingat kata-kata. Pada tahun 1952, ia menciptakan \textbf{A-0 System}, kompiler pertama di dunia.

Idenya sederhana namun heretik: tulis kode dalam simbol yang dimengerti manusia, dan biarkan komputer lain (kompiler) menerjemahkannya menjadi biner mesin. Rekan-rekannya mencemooh, "Komputer hanya bisa melakukan aritmatika, mereka tidak bisa menulis program!" Namun Hopper membuktikan mereka salah. A-0 adalah leluhur dari setiap bahasa pemrograman yang kita gunakan hari ini. Tanpa keberanian Hopper untuk "malas" (dalam arti positif: mengotomatisasi pekerjaan berulang), kita masih akan menulis kode dalam heksadesimal.

Sementara itu, Alan Turing mengajukan pertanyaan yang lebih filosofis: "Bisakah mesin berpikir?" Melalui \textbf{Turing Test} (1950), ia meletakkan tonggak ambisi tertinggi kita. Ia tidak memberikan spesifikasi teknis, melainkan tujuan fungsional: jika sebuah mesin bisa menipu manusia lewat percakapan teks, maka ia cerdas. Ini adalah \textit{North Star} yang masih kita kejar hingga hari ini dengan LLM.

\section{1953 -- 1955: Fondasi Material dan Kecepatan}

Di dunia fisik, revolusi material sedang terjadi. Tabung vakum yang rapuh mulai ditinggalkan. Texas Instruments, melalui Gordon Teal, memperkenalkan \textbf{Silicon Transistor} pertama pada tahun 1954.

Keputusan untuk beralih ke silikon (dari germanium) adalah \textit{The Artisan's Choice} yang mendefinisikan seluruh industri. Silikon lebih tahan panas, lebih stabil, dan bahannya (pasir) melimpah. Inilah awal "Silicon Valley" secara harfiah.

Dengan material baru ini, komputer menjadi cukup andal untuk tugas-tugas kritis. IBM meluncurkan \textbf{IBM 701} (1953) dengan \textbf{Magnetic Core Memory}. Memori inti ini—cincin ferit kecil yang ditenun dengan kawat—memberikan kita akses data yang cepat dan \textit{non-volatile}. Meskipun kita sudah lama meninggalkan cincin magnetik, prinsip \textit{Random Access} yang lahir di sini tetap menjadi standar emas arsitektur memori kita.

\section{1956 -- 1957: Bahasa Bagi Para Dewa Sains}

Dengan perangkat keras yang semakin kuat, kebutuhan akan bahasa pemrograman yang lebih ekspresif meledak. Para ilmuwan membutuhkan cara untuk menulis rumus matematika, bukan instruksi pemindahan register.

Di IBM, John Backus memimpin tim untuk menciptakan \textbf{FORTRAN} (Formula Translation). Dirilis pada tahun 1957, FORTRAN adalah bahasa tingkat tinggi pertama yang sukses secara komersial. Ia memungkinkan ilmuwan menulis `X = (Y + Z) / 2` alih-alih deretan kode mesin yang panjang.

Kritikus awal meragukan efisiensinya. "Kode hasil mesin tidak akan secepat kode tulisan tangan!" teriak mereka. Namun, Backus menciptakan \textit{optimizing compiler} yang begitu cerdas hingga kode yang dihasilkannya sering kali lebih efisien daripada buatan manusia. Pelajaran bagi Artisan: abstraksi yang baik tidak menyembunyikan kekuatan mesin; ia melipatgandakan kekuatan manusia tanpa mengorbankan performa mesin.

Pada tahun 1956, di sebuah konferensi musim panas di Dartmouth, John McCarthy, Marvin Minsky, dan kawan-kawan secara resmi melahirkan istilah \textbf{Artificial Intelligence}. Mimpi Turing kini memiliki nama, dan bidang studi baru pun lahir.

\section{1958: Penyatuan Fisik dan Logika Simbolik}

Tahun 1958 adalah tahun mukjizat ganda.

Di Texas Instruments, Jack Kilby memecahkan "Tyranny of Numbers"—masalah di mana rangkaian elektronik menjadi terlalu rumit untuk disolder tangan. Kilby menyadari bahwa semua komponen (resistor, kapasitor, transistor) bisa dibuat dari satu blok bahan semikonduktor yang sama. Ia menciptakan \textbf{Integrated Circuit (IC)} pertama.

Momen ini adalah "Big Bang" miniaturisasi. Tanpa IC, komputer selamanya akan sebesar ruangan. Kilby mengajari kita bahwa integrasi—menyatukan fungsi-fungsi terpisah ke dalam satu substrat koheren—adalah kunci skalabilitas.

Di saat yang sama, John McCarthy di MIT menciptakan \textbf{LISP} (List Processing). Berbeda dengan FORTRAN yang imperatif, LISP didasarkan pada kalkulus lambda. Ia memperkenalkan konsep-konsep "alien" seperti rekursi, \textit{garbage collection}, dan kode sebagai data. LISP menjadi bahasa bagi AI, bahasa bagi mereka yang ingin memodelkan \textit{pikiran} alih-alih memodelkan \textit{mesin}. Bagi Artisan modern, LISP adalah pengingat bahwa keindahan matematis dalam kode adalah bentuk seni tersendiri.

\section{1959: Bahasa Bisnis dan Demokratisasi}

Dekade ini ditutup dengan masuknya komputer ke dunia bisnis arus utama. \textbf{COBOL} (Common Business-Oriented Language) diciptakan sebagai bahasa standar untuk pemrosesan data, dirancang agar bisa dibaca seperti bahasa Inggris.

Meskipun sering dicemooh oleh akademisi karena sintaksisnya yang bertele-tele, COBOL berhasil melakukan apa yang tidak bisa dilakukan bahasa lain: ia menangani uang dunia dengan presisi desimal yang sempurna. Hingga tahun 2026, sistem perbankan global masih bertumpu pada pondasi COBOL yang diletakkan di tahun 1959.

Bersamaan dengan itu, \textbf{IBM 1401} diluncurkan. Komputer ini terjangkau, andal, dan menggunakan transistor sepenuhnya. Ia menjadi "Model T" bagi industri komputer, terjual lebih dari 10.000 unit. Tiba-tiba, setiap perusahaan menengah bisa memiliki "otak elektronik" mereka sendiri.

\section{Refleksi Dekade: Membangun Menara Abstraksi}

Dekade 1950-an adalah tentang meletakkan batu pertama dari menara abstraksi yang kita tinggali hari ini.

Kita memulai dekade dengan kabel ruwet dan kode mesin yang tidak terbaca, dan mengakhirinya dengan sirkuit terpadu yang rapi dan bahasa pemrograman yang manusiawi (FORTRAN, COBOL, LISP). Para pionir dekade ini—Hopper, Backus, Kilby, McCarthy—tidak hanya memecahkan masalah teknis; mereka memecahkan masalah \textit{komunikasi}.

Bagi Artisan 2026, pelajaran dari 1950-an adalah tentang \textbf{The Power of Tools}. Grace Hopper tidak menunggu pekerjaan menjadi lebih mudah; ia \textit{membuat alat} (kompiler) untuk membuatnya lebih mudah. John Backus tidak puas dengan kinerja manusia; ia \textit{membuat alat} untuk mengoptimalkan kode. Kita adalah pewaris semangat ini. Tugas kita bukan hanya menggunakan alat, tapi terus menempa alat baru yang mengangkat level abstraksi kita semakin tinggi, mendekati kecepatan pikiran murni.
