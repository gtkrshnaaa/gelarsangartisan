\chapter{The Genesis of Logic (1930 -- 1939)}

Segala sesuatu yang kita bangun hari ini—AI peracik kode, sistem otonom, hingga jaringan global—memiliki akar yang sama di dekade ini. Ini adalah era di mana komputer belum berbentuk fisik, melainkan sebuah gagasan matematis murni.

\section{1936: The Universal Machine}

\begin{description}
    \item[Alan Turing \& Turing Machine] \textit{Saat pertama kali dibuat} (atau lebih tepatnya, dirumuskan dalam makalah "On Computable Numbers"), konsep ini bukanlah sebuah mesin fisik yang bisa kamu sentuh. Alan Turing menciptakan model matematis tentang sebuah mesin yang mampu melakukan kalkulasi apa pun asalkan ada instruksinya. Ia membayangkan sebuah pita tak terbatas dengan simbol-simbol yang dibaca oleh sebuah kepala pemroses.

    \textit{Pada saat buku ini dibuat di tahun 2026}, setiap perangkat elektronik yang ada di sekitar kita—mulai dari jam tangan pintar hingga pusat data raksasa—hanyalah implementasi fisik dari ide Turing ini. Apa yang dulu dianggap sebagai abstraksi filosofis, kini menjadi udara yang kita hirup dalam ekosistem digital.
\end{description}

\section{1937: Pintu Gerbang Biner}

\begin{description}
    \item[Claude Shannon \& Information Theory] \textit{Saat pertama kali dibuat} melalui tesis masternya di MIT, Shannon menunjukkan bahwa logika biner (0 dan 1) dapat diterapkan dalam sirkuit sakelar elektrik. Ia membuktikan bahwa aljabar Boolean bisa digunakan untuk menyelesaikan masalah logika secara otomatis.

    \textit{Pada saat buku ini dibuat}, penemuan Shannon adalah alasan mengapa kita bisa mengirim triliunan bit data melintasi benua dalam hitungan milidetik. Tanpa penyatuan antara logika dan listrik ini, kita masih akan terjebak dalam dunia mekanis yang lambat.
\end{description}

\section{1938: Embrio Komputer Modern}

\begin{description}
    \item[Konrad Zuse \& Z1] \textit{Saat pertama kali dibuat}, Z1 adalah komputer biner mekanis pertama di dunia. Zuse mengerjakannya sendiri di ruang tamu orang tuanya di Berlin. Mesin ini menggunakan plat logam tipis yang digerakkan oleh motor listrik, namun perintahnya dibaca dari pita film 35mm yang dilubangi.

    \textit{Pada saat buku ini dibuat}, kreativitas Zuse yang membangun masa depan di tengah keterbatasan ruang keluarga tetap menjadi inspirasi bagi para \textit{Artisan}. Ia membuktikan bahwa visi yang kuat jauh lebih berharga daripada laboratorium yang paling canggih sekalipun.
\end{description}
