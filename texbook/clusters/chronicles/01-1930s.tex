\chapter{The Genesis of Logic (1930 -- 1939)}

Segala sesuatu yang kita bangun hari ini—AI peracik kode, sistem otonom, hingga jaringan global—memiliki akar yang sama di dekade ini. Ini adalah era di mana komputer belum berbentuk fisik bagi kebanyakan orang, melainkan sebuah gagasan matematis murni yang sedang bergejolak di balik pintu-pintu universitas dan laboratorium murni. Sebagai seorang Artisan, penting bagi kita untuk menarik diri sejenak dari hingar bingar modernitas dan memperhatikan bagaimana "keheningan" di dekade 1930-an sebenarnya adalah badai intelektual yang sedang bersiap mengubah wajah peradaban.

\section{1930: Analogi vs. Digital (Vannevar Bush)}

Di fajar dekade ini, dunia belum mengenal apa itu "komputer" dalam arti biner yang kita pahami hari ini. Yang ada adalah mesin-mesin hitung mekanis yang luar biasa kompleks.

\begin{description}
    \item[Differential Analyzer] \textit{Saat pertama kali dibuat} oleh Vannevar Bush di MIT, ini adalah puncak dari komputasi analog. Mesin raksasa ini terbuat dari poros baja, roda gigi, dan disk yang saling terhubung untuk menyelesaikan persamaan diferensial. Ia merepresentasikan angka bukan sebagai bit diskrit, melainkan sebagai kuantitas fisik yang kontinu—seperti sudut rotasi sebuah poros. Ia adalah monster mekanis yang membutuhkan keahlian teknik tingkat tinggi hanya untuk mengaturnya sebelum sebuah perhitungan dimulai.

    \textit{Pada saat buku ini dibuat di tahun 2026}, kita melihat Differential Analyzer sebagai pengingat akan jalan yang tidak kita ambil. Meskipun kita telah sepenuhnya berpindah ke dunia digital, impian Bush untuk memiliki mesin yang mampu memodelkan realitas fisik yang kompleks tetap hidup dalam setiap simulasi \textit{Digital Twin} yang kita jalankan hari ini. Bedanya, hari ini kita melakukannya dengan triliunan transistor, sementara Bush melakukannya dengan oli dan baja. Sebagai Artisan, kita harus menghargai keberanian Bush untuk membangun "otak mekanis" di masa ketika ide itu terdengar seperti sihir.
\end{description}

\section{1931: Batas Logika (Kurt Gödel)}

Tahun ini menandai ledakan yang paling dahsyat dalam sejarah pemikiran manusia, yang secara ironis, membuktikan kelemahan fundamental dari logika itu sendiri.

\begin{description}
    \item[Incompleteness Theorems] \textit{Saat pertama kali dibuat} oleh matematikawan asal Austria, Kurt Gödel, teorema ini menghancurkan mimpi para matematikawan abad ke-20 untuk menemukan sistem logika yang lengkap dan konsisten. Gödel membuktikan bahwa dalam sistem formal apa pun, akan selalu ada pernyataan yang benar tetapi tidak dapat dibuktikan di dalam sistem tersebut. Ini adalah "cacat" matematis yang indah.

    \textit{Pada saat buku ini dibuat}, Teorema Gödel adalah alasan mengapa kita memahami batas dari apa yang bisa dilakukan oleh kecerdasan buatan. Sehebat apa pun agen otonom atau model bahasa yang kita miliki di tahun 2026, mereka tetap terikat oleh batasan logika yang ditemukan Gödel hampir seabad lalu. Memahami Gödel berarti memahami bahwa ada hal-hal yang tidak akan pernah bisa kita "hitung"—sebuah pengakuan rendah hati yang harus dimiliki oleh setiap pengembang teknologi sejati. Ia mengajari kita bahwa di balik setiap baris kode, ada misteri yang tak terpecahkan.
\end{description}

\section{1932: Pionir Kecepatan (The MIT Rock}

\section{1932: Mimpi Howard Aiken dan Ambisi Presisi}

Di sisi lain samudra, di Universitas Harvard, seorang pria bernama Howard Aiken mulai merasa frustrasi dengan keterbatasan alat hitung yang tersedia untuk riset doktoralnya. Ia memimpikan sebuah mesin yang bisa melakukan kalkulasi saintifik secara otomatis tanpa kesalahan manusia.

\begin{description}
    \item[Visi Harvard Mark I] \textit{Saat pertama kali dibuat} (dalam bentuk proposal konseptual), pemikiran Aiken masih sangat dipengaruhi oleh karya Charles Babbage dari abad sebelumnya. Aiken ingin membangkitkan kembali "Analytical Engine" yang belum sempat diselesaikan Babbage. Ia tidak memikirkan tabung vakum—ia memikirkan baja, poros, dan sakelar elektromagnetik. Ini adalah upaya untuk mendorong mekanika hingga batas kemampuannya. 

    \textit{Pada saat buku ini dibuat}, kita melihat upaya Aiken sebagai jembatan yang diperlukan. Meskipun mesinnya (yang kelak menjadi Harvard Mark I) akan segera terlihat kuno dibandingkan dengan revolusi elektronik yang akan datang, disiplin yang ia terapkan dalam pemrosesan data otomatis meletakkan dasar bagi komputasi saintifik modern. Sebagai Artisan, kita belajar dari Aiken bahwa terkadang kita harus menoleh ke masa lalu—ke inspirasi Babbage—untuk melihat masa depan yang belum terwujud. Kita tidak mengejar kebaruan demi kebaruan, kita mengejar solusi atas masalah yang paling mendalam.
\end{description}

\section{1933: Pilihan Biner Sang Artisan (Konrad Zuse)}

Di Berlin, di tengah gejolak politik yang mulai memanas, seorang insinyur sipil muda bernama Konrad Zuse mulai merasa jemu dengan perhitungan manual yang membosankan dalam rekayasa struktur. Ia memutuskan untuk membangun sebuah mesin yang bisa melakukan itu untuknya. Namun, ia membuat keputusan yang akan mengubah segalanya: ia memilih sistem biner.

\begin{description}
    \item[Binary vs. Decimal Logic] \textit{Saat pertama kali dibuat}, keputusan Zuse untuk menggunakan sistem basis dua (biner) daripada sistem basis sepuluh (desimal) yang lazim digunakan pada mesin hitung kala itu adalah tindakan keberanian intelektual yang murni. Zuse menyadari bahwa sistem biner—yang hanya mengenal "ya" atau "tidak", "ada arus" atau "tidak ada arus"—jauh lebih mudah diimplementasikan dalam mesin mekanis maupun elektrik. Ia memilih kesederhanaan biner untuk menangani kompleksitas perhitungan.

    \textit{Pada saat buku ini dibuat}, sistem biner telah menjadi "bahasa ibu" dari seluruh realitas digital kita. Keputusan Zuse di tahun 1933 adalah contoh sempurna dari \textit{The Artisan's Choice}. Ia tidak mengikuti tren desimal yang dominan di industri; ia memilih jalur yang paling logis dan efisien untuk visinya sendiri. Di tahun 2026, setiap kali kita menulis kode dalam bahasa tingkat tinggi apa pun, kita sebenarnya sedang membangun di atas fondasi biner yang dipilih Zuse di ruang tamunya sembilan puluh tiga tahun yang lalu.
\end{description}

\section{1934: Kematangan Intelektual Alan Turing}

Tahun ini sering kali terabaikan, namun di sinilah Alan Turing mulai mematangkan ide-idenya di Cambridge. Ia mulai mempertimbangkan pertanyaan-pertanyaan dasar tentang apa yang bisa dihitung dan apa yang tidak.

\begin{description}
    \item[The Seed of Alan's Mind] \textit{Saat pertama kali dibuat}, pemikiran Turing di babak awal ini bukanlah tentang membangun mesin dari logam. Ia sedang bergulat dengan "Entscheidungsproblem" (Masalah Keputusan) yang diajukan oleh David Hilbert. Turing mulai membayangkan sebuah "State Machine" matematis yang bisa berpindah-pindah keadaan berdasarkan instruksi tertentu. Ini adalah momen di mana "perangkat lunak" mulai diletakkan fondasinya, bahkan sebelum "perangkat keras" yang mumpuni lahir.

    \textit{Pada saat buku ini dibuat}, kita memahami bahwa ide Turing di tahun 1934 adalah "nyawa" yang mendahului tubuh. Di tahun 2026, saat kita merancang arsitektur sistem yang kompleks, kita sebenarnya sedang melakukan apa yang Turing lakukan: memodelkan aliran logika sebelum memikirkan infrastruktur fisiknya. Sebagai Artisan, kita harus memiliki kemampuan untuk "melihat" sistem berjalan di dalam pikiran kita sebelum jari kita menyentuh papan tik.
\end{description}

\section{1936: Kitab Suci Komputasi (Alan Turing)}

Jika ada satu tahun yang harus dianggap sebagai "Tahun Nol" bagi peradaban digital kita, itu adalah tahun 1936. Di tahun ini, Alan Turing menerbitkan makalah berjudul "On Computable Numbers, with an Application to the Entscheidungsproblem". Makalah ini bukan sekadar riset akademik; ia adalah janji masa depan yang kita tinggali hari ini.

\begin{description}
    \item[The Universal Turing Machine (UTM)] \textit{Saat pertama kali dibuat}, konsep UTM adalah solusi Turing atas tantangan David Hilbert tentang batasan pembuktian matematis. Turing membayangkan sebuah mesin yang tidak hanya bisa melakukan satu tugas (seperti mesin hitung tradisional), tapi sebuah mesin yang bisa mensimulasikan mesin lainnya jika "deskripsi" (program) mesin tersebut diberikan kepadanya. Inilah inti dari apa yang kita sebut "komputer general-purpose". Turing memisahkan antara entitas fisik (mesin) dengan instruksi yang menjalankannya (perangkat lunak).

    \textit{Pada saat buku ini dibuat di tahun 2026}, gagasan Turing tentang pemisahan antara \textit{hardware} dan \textit{software} adalah pilar yang menopang seluruh ekonomi global. Setiap baris kode Python, Rust, atau C++ yang kita tulis di tahun 2026 adalah "instruksi" yang sedang dibaca oleh UTM fisik (prosesor kita). Di era Agentic AI saat ini, kita bahkan melihat realisasi paling ekstrem dari ide Turing: agen AI yang secara otonom menulis "deskripsi" mesin baru untuk menyelesaikan tugas-tugas yang belum pernah ada sebelumnya. Sebagai Artisan, kita harus memahami bahwa kita tidak sedang memanipulasi elektron; kita sedang memanipulasi simbol-simbol di atas pita tak terbatas Turing.
\end{description}

\section{1937: Jembatan Antara Logika dan Elektrik (Claude Shannon)}

Sementara Turing membangun fondasi logikanya di Inggris, seorang pemuda berusia 21 tahun bernama Claude Shannon sedang menulis tesis master di MIT yang akan memberikan "tubuh" pada logika tersebut.

\begin{description}
    \item[The Most Important Master's Thesis] \textit{Saat pertama kali dibuat}, tesis Shannon yang berjudul "A Symbolic Analysis of Relay and Switching Circuits" membuktikan sebuah hal yang revolusioner: sirkuit sakelar elektrik (Relay) dapat melakukan operasi logika Boolean (AND, OR, NOT). Shannon menunjukkan bahwa kita bisa membangun "otak" elektrik hanya dengan merangkai sakelar-sakelar sederhana. Ini adalah titik di mana matematika murni bertemu dengan teknik elektro.

    \textit{Pada saat buku ini dibuat}, tesis Shannon adalah alasan mengapa komputer kita terbuat dari sirkuit elektrik dan bukan dari roda gigi mekanis. Setiap gerbang logika di dalam CPU modern kita adalah turunan langsung dari sirkuit relay Shannon. Penemuan ini memungkinkan abstraksi tingkat tinggi; kita tidak perlu lagi memikirkan tegangan listrik, kita hanya perlu memikirkan aliran logika "0" dan "1". Sebagai Artisan, pemahaman Shannon tentang "informasi" sebagai sesuatu yang bisa diukur dan dimanipulasi adalah alat paling tajam dalam kotak peralatan kita.
\end{description}

\section{1938: Z1 Selesai dan Kelahiran Mekanika Biner}

Di tengah ancaman perang di Eropa, Konrad Zuse akhirnya menyelesaikan Z1 di ruang tamu keluarganya.

\begin{description}
    \item[The Mechanical Binary Brain] \textit{Saat pertama kali dibuat}, Z1 adalah keajaiban mekanika yang sangat presisi. Karena Zuse tidak memiliki akses ke relay elektrik yang mahal, ia membuat ribuan plat logam tipis yang dipotong tangan untuk menyimpan status biner. Meskipun Z1 sering mengalami kemacetan karena masalah gesekan mekanis, secara arsitektural ia sudah sempurna: ia memiliki unit kontrol, memori biner, dan unit aritmatika \textit{floating-point}.

    \textit{Pada saat buku ini dibuat}, Z1 tetap menjadi simbol dari keteguhan hati seorang Artisan. Zuse membangun masa depan tanpa dana hibah besar atau laboratorium nasional. Ia membuktikan bahwa desain yang solid lebih penting daripada material yang mewah. Di tahun 2026, kita sering kali mengeluh tentang keterbatasan RAM atau GPU, namun Zuse mengingatkan kita bahwa dengan logika biner yang tepat, bahkan potongan logam pun bisa "berpikir".
\end{description}

\section{1939: Fajar Elektronik dan Garasi Silicon Valley}

Dekade ini ditutup dengan dua peristiwa yang sangat kontras namun sama-sama krusial: kelahiran komputer elektronik pertama dan kelahiran budaya \textit{startup}.

\begin{description}
    \item[ABC (Atanasoff-Berry Computer)] \textit{Saat pertama kali dibuat} di Iowa State College, ABC adalah upaya pertama untuk menggunakan tabung vakum (vacuum tubes) untuk mempercepat perhitungan aljabar. ABC memperkenalkan penggunaan kapasitor untuk menyimpan memori biner (\textit{regenerative capacitor memory})—metode yang secara konseptual mirip dengan bagaimana DRAM modern kita bekerja.

    \textit{Pada saat buku ini dibuat}, tabung vakum ABC telah lama digantikan oleh semikonduktor, namun prinsip penyimpanan memori dinamisnya tetap menjadi standar emas. ABC adalah pengingat bahwa kecepatan membutuhkan transisi dari mekanik ke elektronik.

    \item[Kelahiran Hewlett-Packard (HP)] Di satu sisi lain, Bill Hewlett dan David Packard mendirikan HP di sebuah garasi di Palo Alto. Produk pertama mereka bukanlah komputer, melainkan osilator audio (HP 200A). 

    \textit{Pada saat buku ini dibuat}, "Garasi HP" telah menjadi mitos suci bagi setiap Artisan teknologi. Ia melambangkan filosofi \textit{Garage to Global}. Dekade 1930-an berakhir dengan pesan yang kuat: masa depan IT sedang dimasak di ruang tamu dan garasi, oleh individu-individu yang berani menantang status quo.
\end{description}

\section{Membangun di Atas Reruntuhan: Konteks Sosio-Teknikal 1930-an}

Kita tidak bisa memahami babak Genesis ini tanpa melihat latar belakang dunia saat itu. Dekade 1930-an adalah masa \textit{The Great Depression}. Di saat ekonomi global hancur dan pengangguran merajalela, para pionir teknologi ini bekerja dalam isolasi yang hampir total. Konrad Zuse tidak memiliki laboratorium di universitas ternama; ia memiliki ruang tamu orang tuanya. Alan Turing tidak memiliki superkomputer; ia hanya memiliki pena, kertas, dan pikirannya yang tajam.

\textit{Saat pertama kali dibuat}, penemuan-penemuan ini lahir dari keterbatasan yang ekstrem. Ketiadaan dana paksa membuat mereka harus kreatif—sebuah kualitas artisan yang paling murni. Mereka tidak mengejar profit triwulanan; mereka mengejar kebenaran logika. Inilah mengapa komputasi modern memiliki "jiwa" yang begitu kuat; ia lahir dari hasrat untuk memecahkan kebuntuan intelektual di tengah kekacauan dunia fisik.

\textit{Pada saat buku ini dibuat di tahun 2026}, kita sering kali terlena dengan kelimpahan sumber daya. Kita memiliki akses ke cloud yang tak terbatas dan perangkat keras yang melampaui imajinasi Zuse atau Turing. Namun, apakah kita masih memiliki kreativitas yang lahir dari keterbatasan tersebut? Sebagai Artisan, tugas kita adalah menjaga agar kelimpahan tidak membunuh ketajaman pemikiran kita. Kita harus belajar dari dekade 1930-an bahwa batasan bukanlah penghalang, melainkan katalisator bagi inovasi yang paling fundamental.

\section{Filosofi Algoritma: Sebelum Nama "Komputer" Lahir}

Penting untuk dicatat bahwa di hampir seluruh dekade ini, kata "komputer" lebih sering digunakan untuk merujuk pada manusia yang melakukan perhitungan manual (human computers). Mesin-mesin yang kita bahas di sini sering kali disebut sebagai "calculating engines" atau "automatic calculators".

\textit{Saat pertama kali dibuat}, algoritma adalah sebuah konsep elegan yang murni matematis. Ketika Turing merumuskan mesin universalnya, ia sedang memikirkan prosedur—sebuah set instruksi yang tidak ambigu. Ini adalah pemisahan besar pertama dalam sejarah: antara "pemikir" (instruksi) dan "pelaksana" (mesin). Inilah momen di mana kita mulai memahami bahwa kecerdasan dapat dikodifikasikan.

\textit{Pada saat buku ini dibuat}, algoritma telah menjadi arsitek tak terlihat dari kehidupan kita. Di tahun 2026, algoritma tidak hanya menghitung angka—ia menyusun alur kerja kita, mengarahkan perhatian kita, dan bahkan mulai menggantikan intuisi kita melalui model-model generatif. Namun, prinsip dasarnya tetap sama dengan apa yang dirumuskan di tahun 1930-an: sebuah prosedur diskrit menuju sebuah tujuan. Sebagai Artisan, kita harus selalu ingat bahwa kita adalah tuan dari prosedur ini, bukan budaknya. Kita mengarahkan arus logika ini untuk melayani kemanusiaan, bukan sebaliknya.

\section{Atmosfer Era: Keheningan Sebelum Badai}

Untuk benar-benar menangkap esensi dari dekade 1930-an, kita harus membayangkan sebuah dunia di mana informasi mengalir lewat surat fisik dan publikasi jurnal yang memakan waktu berbulan-bulan untuk sampai ke seberang lautan. Tidak ada internet, tidak ada kolaborasi real-time. Keheningan inilah yang memungkinkan pemikiran-pemikiran mendalam seperti milik Turing atau Gödel untuk tumbuh tanpa interupsi.

\textit{Saat pertama kali dibuat}, riset-riset ini sering kali dianggap sebagai cabang matematika yang sangat esoterik dan hampir tidak memiliki aplikasi praktis. Siapa yang butuh mesin universal ketika masalah terbesar dunia saat itu adalah kelaparan dan ancaman perang? Namun, justru di dalam "ketidakrelevanan" itulah letak kekuatan sang Artisan. Mereka membangun fondasi bukan untuk hari esok, tapi untuk selamanya.

\textit{Pada saat buku ini dibuat di tahun 2026}, kita hidup dalam dunia yang sangat bising. Kita dibombardir oleh tren teknologi setiap jam. Kita kehilangan kemampuan untuk "berpikir dalam" seperti para raksasa di tahun 1930-an. Sebagai Artisan, tantangan terbesar kita di tahun 2026 bukanlah mempelajari bahasa pemrograman baru, melainkan menciptakan "ruang batin" yang tenang di tengah kebisingan digital agar kita bisa melahirkan ide-ide yang memiliki daya tahan melampaui siklus update mingguan. Inilah yang saya sebut sebagai \textit{The 1930s Mindset}—kemampuan untuk fokus pada esensi di tengah kekacauan.

\section{Transisi Logika: Dari Desimal ke Biner}

Salah satu detail teknis yang sering dilewatkan adalah betapa sulitnya orang-orang saat itu untuk melepaskan sistem desimal. Kita telah menghitung dengan sepuluh jari selama ribuan tahun. Membangun mesin biner bukan hanya tantangan teknik, tapi tantangan psikologis.

\textit{Saat pertama kali dibuat}, mesin-mesin hitung seperti yang ada di IBM (Hollerith machines) sepenuhnya desimal. Menggunakan biner dianggap sebagai pemborosan ruang karena membutuhkan lebih banyak posisi untuk merepresentasikan angka yang sama. Namun, Zuse dan Shannon menyadari sesuatu yang lebih dalam: keandalan. Dalam sistem biner, sebuah sakelar hanya perlu berada dalam kondisi "hidup" atau "mati". Tidak ada zona abu-abu. Ini adalah kunci menuju komputasi yang bebas kesalahan.

\textit{Pada saat buku ini dibuat}, keandalan biner adalah apa yang memungkinkan kita untuk menjalankan sistem mission-critical tanpa henti. Dari kendali reaktor nuklir hingga algoritma perdagangan saham frekuensi tinggi, semuanya bergantung pada ketegasan 0 dan 1. Sebagai Artisan, kita harus menghargai kejernihan biner. Ia mengajarkan kita bahwa dalam dunia sistem yang kompleks, kesederhanaan di level paling dasar adalah satu-satunya cara untuk menjaga integritas keseluruhan struktur.

\section{Etos Sang Artisan: Warisan Murni 1930-an}

Sebagai penutup akhir dari babak Genesis ini, kita harus menyadari bahwa warisan terbesar dari dekade 1930-an bukanlah mesinnya, melainkan etos kerjanya. Para pionir ini membuktikan bahwa dengan ketajaman logika dan kejernihan visi, seseorang dapat membangun imperium intelektual di atas meja kayu yang sederhana. Mereka mengajari kita bahwa teknologi, pada intinya, adalah perpanjangan dari kehendak manusia yang dikoordinasikan secara matematis.

\textit{Saat pertama kali dibuat}, ide-ide ini adalah bentuk perlawanan terhadap ketidakteraturan. Di tengah dunia yang sedang menuju kehancuran, mereka menciptakan sistem yang sepenuhnya teratur dan dapat diprediksi. Inilah perlindungan terakhir sang Artisan: kemampuan untuk membangun keteraturan di dalam kekacauan.

\textit{Pada saat buku ini dibuat di tahun 2026}, kita memegang warisan ini di tangan kita setiap hari. Saat kita menekan tombol \textit{compile} atau memberikan perintah pada AI kita, kita sedang mengaktifkan rantai logika yang panjangnya mencapai hampir satu abad ke belakang. Menghargai 1930-an berarti menyadari bahwa kita tidak pernah benar-benar membangun sesuatu dari awal; kita hanya sedang melanjutkan dialog teknis yang dimulai oleh para raksasa ini di tengah keheningan masa antarperang. Sebagai Artisan 2026, kita adalah penjaga estafet logika ini, memastikan bahwa kejernihan dan integritas yang mereka perjuangkan tetap hidup dalam setiap bit yang kita olah.

\section{Disiplin Sang Artisan: Pelajaran dari Masa Antarperang}

Sebagai penutup dari eksplorasi mendalam kita terhadap dekade 1930-an, kita harus merenungkan satu hal yang paling berharga: disiplin. Para raksasa yang kita bahas—Turing, Zuse, Shannon, Gödel—bekerja dalam kondisi di mana "umpan balik" (feedback loop) sangat lambat. Mereka tidak memiliki kemewahan untuk melakukan kompilasi ulang dalam hitungan detik. Setiap ide harus dimatangkan di dalam pikiran sebelum diwujudkan dalam tulisan atau mesin.

\textit{Saat pertama kali dibuat}, ide-ide ini adalah hasil dari kontemplasi yang sangat panjang. Kedalaman pemikiran mereka adalah fungsi dari kesabaran mereka. Mereka tidak terburu-buru untuk "merilis" sesuatu; mereka terobsesi untuk "memahami" sesuatu. Inilah yang membedakan seorang Artisan dari sekadar pekerja teknis. Artisan membangun dengan kesadaran bahwa apa yang mereka buat harus memiliki integritas internal yang sempurna.

\textit{Pada saat buku ini dibuat di tahun 2026}, kita hidup di era *instant gratification*. Kita ingin AI memberikan jawaban dalam waktu kurang dari satu detik. Kita ingin fitur baru dideploy setiap hari. Namun, kita sering kali mengorbankan kedalaman demi kecepatan. Pelajaran dari tahun 1930-an adalah bahwa \textit{quality is a function of time and focus}. Jika kita ingin membangun sistem yang bertahan selama 100 tahun ke depan, kita harus belajar untuk kembali ke kecepatan 1930-an dalam hal perancangan konseptual. Jangan biarkan alat-alat modern kita yang cepat membuat pemikiran kita menjadi dangkal.

\section{Refleksi Dekade: Fondasi yang Tak Tergoyahkan}

Dekade 1930-an ditutup dengan dunia yang berada di ambang perang besar kedua. Mesin-mesin logika yang lahir di masa ini akan segera diuji dalam api pertempuran yang nyata. Dari ruang tamu Zuse hingga kantor-kantor riset di Princeton dan Cambridge, benih digital telah ditanam.

\begin{description}
    \item[Warisan Sang Artisan] \textit{Saat pertama kali dibuat}, dekade ini memberikan kita tiga pilar: Biner (Zuse/Shannon), Universality (Turing), dan Limits (Gödel). Tiga pilar ini adalah "segitiga suci" komputasi. Tanpa biner, sirkuit kita akan terlalu kompleks; tanpa universality, kita tidak akan memiliki perangkat lunak; dan tanpa pemahaman tentang batasan, kita akan tersesat dalam ambisi yang sia-sia.

    \textit{Pada saat buku ini dibuat}, kita melihat bahwa meskipun teknologi telah berevolusi dari relay mekanis ke tabung vakum, dari transistor ke silikon, dan dari mikrochip ke komputer kuantum, struktur logika dasarnya tidak pernah berubah. Kita masih hidup di dunia yang didefinisikan oleh pria-pria di tahun 1930-an. Inilah kekuatan dari sebuah karya artisan sejati: ia melampaui waktu dan terus relevan bahkan satu abad kemudian.
\end{description}
