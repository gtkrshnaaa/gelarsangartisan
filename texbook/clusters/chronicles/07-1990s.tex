\chapter{The Internet Explosion (1990 -- 1999)}

Jika 1980-an adalah tentang standarisasi mesin tunggal, maka 1990-an adalah tentang konektivitas massal. Ini adalah dekade di mana kabel-kabel tembaga dan serat optik mulai menjahit benua menjadi satu saraf digital raksasa. Sebagai Artisan, kita melihat 1990-an sebagai era "Demokratisasi Alat Produksi". Kode tidak lagi hanya milik korporasi besar; ia mulai mengalir di tangan individu melalui gerakan Open Source.

\section{1990: Fondasi Web dan Bahasa Masa Depan}

Tahun ini menandai kelahiran infrastruktur yang akan kita gunakan untuk membangun peradaban digital.

\begin{description}
    \item[The First Web Server \& Browser] \textit{Saat pertama kali dibuat} oleh Tim Berners-Lee di CERN menggunakan komputer NeXT yang canggih, server web pertama (httpd) dan peramban pertama (WorldWideWeb) mulai berjalan. Ia menggunakan protokol sederhana yang disebut HTTP (\textit{Hypertext Transfer Protocol}). Di balik layar, HTTP/0.9 pada saat itu hanyalah protokol satu baris: "GET /path/ke/file". Tidak ada header, tidak ada status code, hanya pengambilan dokumen mentah.

    \textit{Pada saat buku ini dibuat di tahun 2026}, kita melihat HTTP sebagai denyut nadi dunia. Sebagai Artisan, kita harus merenungkan kejeniusan di balik alamat URL. Di balik layar, URL adalah sistem penamaan universal yang memungkinkan dokumen di satu komputer dirujuk oleh komputer mana pun di dunia. Pengaruh Berners-Lee datang dari kesederhanaan protokolnya yang agnostik terhadap sistem operasi. Inilah \textit{The Art of Influence}: menciptakan sistem yang begitu mudah diadopsi sehingga ia menyebar seperti virus kebaikan melalui arsitektur desentralisasi.

    \item[Python 0.9.0 (Guido van Rossum)] \textit{Saat pertama kali dibuat} di CWI Belanda sebagai hobi liburan Natal, Python dirancang sebagai penerus bahasa ABC dengan fokus pada keterbacaan kode dan produktivitas pengembang. Guido ingin menciptakan bahasa yang "enak dibaca bagi manusia, tapi tetap perkasa bagi mesin". Di balik layar, Python menggunakan mekanisme *bytecode* dan *Virtual Machine* sederhana untuk mengeksekusi instruksi.

    \textit{Pada saat buku ini dibuat}, Python adalah bahasa nomor satu untuk Data Science, Automation, dan AI. Sebagai Artisan, Python mengajari kita tentang "Estetika Kode". Di balik layar, indentasi wajib bukan sekadar gaya visual, tapi cara teknis untuk memaksa struktur kode yang konsisten tanpa perlu kurung kurawal yang berantakan. Pengaruh Python datang dari filosofinya yang tertuang dalam *The Zen of Python*. Inilah cara Artisan memengaruhi orang lain untuk menulis kode yang tidak hanya berjalan, tapi juga bercerita dengan indah.
\end{description}

\section{1991: Kelahiran Sang Raja Terbuka dan Visualisasi}

Tahun di mana sejarah sistem operasi berubah selamanya oleh sebuah email sederhana dari seorang mahasiswa di Finlandia.

\begin{description}
    \item[Linux Kernel (Linus Torvalds)] \textit{Saat pertama kali dibuat}, "Linux" hanyalah sebuah hobi, proyek kecil yang tidak akan menjadi "besar dan profesional seperti GNU". Linus ingin membuat sebuah kernel mirip Unix yang bisa berjalan di prosesor Intel 386 miliknya untuk mengeksploitasi fitur \textit{switching task} 32-bit. Di balik layar, Linux menggunakan arsitektur \textit{monolithic kernel} yang sangat efisien dalam manajemen interupsi dan akses perangkat keras.

    \textit{Pada saat buku ini dibuat}, Linux menjalankan hampir seluruh server di internet, superkomputer, hingga pusat data AI terbesar. Sebagai Artisan, Linux adalah contoh \textit{Influence} melalui keterbukaan radikal. Di balik layar, sistem *GPL (General Public License)* memastikan bahwa inovasi tetap menjadi milik publik. Pelajaran bagi Artisan: jangan takut untuk memulai sesuatu yang kecil di garasi Anda dan membiarkan dunia membantu menyempurnakannya. Kekuatan sejati Linux bukan pada baris kodenya, tapi pada ekosistem kolaborasi yang ia ciptakan secara organik.

    \item[Visual Basic 1.0] \textit{Saat pertama kali dibuat} oleh Microsoft, VB membawa konsep "Drag-and-Drop" ke pengembangan aplikasi Windows. Di balik layar, ia menggunakan mesin \textit{P-Code} (Pseudo Code) yang ditafsirkan oleh runtime library (*vbrun100.dll*).

    \textit{Pada saat buku ini dibuat}, kita melihat VB sebagai pelopor pembangunan aplikasi cepat (RAD). Sebagai Artisan, VB mengajarkan kita tentang "Penyederhanaan Antarmuka Pengembang". Terkadang, abstraksi visual adalah kunci untuk mempercepat inovasi. Namun, ingatlah disiplin Artisan: meskipun antarmuka ditarik dengan mouse, logika di baliknya harus tetap kokoh dan efisien untuk mencegah "bloatware".
\end{description}

\section{1992: Multimedia dan Grafis 3D Rumahan}

Komputer mulai menjadi mesin hiburan yang serius dengan kemampuan grafis yang melampaui teks.

\begin{description}
    \item[OpenGL 1.0] \textit{Saat pertama kali dibuat} oleh Silicon Graphics Inc. (SGI), OpenGL menjadi standar antarmuka pemrograman aplikasi (API) untuk grafis 2D dan 3D.

    \textit{Pada saat buku ini dibuat}, OpenGL adalah fondasi dari ribuan game dan perangkat lunak desain. Di balik layar, ia memungkinkan perangkat lunak berbicara dengan perangkat keras grafis menggunakan bahasa yang sama. Sebagai Artisan, OpenGL mengajari kita tentang "Abstraksi Perangkat Keras". Pengaruh besar dicapai ketika kita menyediakan cara standar untuk mengontrol mesin yang sangat kompleks (GPU) dengan instruksi yang elegan.

    \item[The browser wars begin (Cello \& ViolaWWW)] \textit{Saat pertama kali dibuat}, peramban-peramban awal ini mencoba membayangkan bagaimana informasi harus ditampilkan. Mereka adalah langkah awal sebelum Mosaic meledak.

    \textit{Pada saat buku ini dibuat}, kita menyadari bahwa antarmuka informasi adalah medan perang pengaruh yang paling nyata. Siapa yang menguasai jendela (peramban), ia menguasai aliran informasi.
\end{description}

\section{1993: Ledakan Web (Mosaic) dan Standar Komunikasi}

Tahun di mana internet benar-benar "memiliki wajah" dan siap masuk ke ruang tamu setiap orang.

\begin{description}
    \item[NCSA Mosaic] \textit{Saat pertama kali dibuat} oleh Marc Andreessen dan Eric Bina, Mosaic adalah peramban pertama yang bisa menampilkan gambar di dalam teks (inline images). Sebelumnya, gambar harus diunduh secara terpisah.

    \textit{Pada saat buku ini dibuat}, kita melihat Mosaic sebagai "The Big Bang" dari ekonomi web modern. Di balik layar, kemampuan untuk merender gambar bersama teks mengubah internet dari perpustakaan membosankan menjadi majalah interaktif. Sebagai Artisan, Mosaic mengajarkan kita tentang "Kekuatan Presentasi Visual". Pengaruh bukan hanya soal data yang benar, tapi tentang bagaimana data tersebut disajikan hingga menggugah emosi manusia.

    \item[Debian GNU/Linux] \textit{Saat pertama kali dibuat} oleh Ian Murdock, Debian mendefinisikan apa itu distribusi Linux yang demokratis dan stabil.

    \textit{Pada saat buku ini dibuat}, Debian adalah dasar dari Ubuntu dan ribuan distronya. Sebagai Artisan, Debian mengajari kita tentang "Disiplin Standar". Di balik layar, sistem manajemen paket `.deb` adalah keajaiban organisasi. Pengaruh Debian dibangun di atas komitmen terhadap kebebasan perangkat lunak dan stabilitas jangka panjang.
\end{description}
\section{1994: Revolusi Bahasa Web dan Kelahiran Raksasa}

Tahun di mana internet mulai memiliki "otak" di sisi server dan platform baru mulai bermunculan.

\begin{description}
    \item[PHP (Personal Home Page)] \textit{Saat pertama kali dibuat} oleh Rasmus Lerdorf sebagai sekumpulan skrip CGI dalam bahasa C untuk melacak pengunjung resume-nya. Di balik layar, PHP adalah cara sederhana untuk menyematkan logika pemrograman langsung di dalam file HTML.

    \textit{Pada saat buku ini dibuat}, PHP menjalankan lebih dari 75\% web dinamis di dunia (termasuk WordPress). Sebagai Artisan, PHP mengajari kita tentang "Pragmatisme di atas Purisme". Ia mungkin bukan bahasa tercantik, tapi ia menyelesaikan masalah dengan sangat cepat. Pengaruh PHP datang dari kemudahannya bagi pemula untuk mulai membangun sesuatu yang berguna. Inilah \textit{The Art of Influence} melalui aksesibilitas.

    \item[Java 1.0 (The Green Project)] \textit{Saat pertama kali dibuat} oleh James Gosling di Sun Microsystems sebagai bahasa untuk perangkat elektronik konsumen, Java dijanjikan dengan jargon "Write Once, Run Anywhere" (WORA). Di balik layar, kejeniusannya terletak pada \textit{Java Virtual Machine} (JVM). Kode dikompilasi menjadi *bytecode* netral-arsitektur yang dijalankan oleh mesin virtual.

    \textit{Pada saat buku ini dibuat}, JVM adalah salah satu platform paling stabil untuk sistem skala besar korporat. Sebagai Artisan, Java mengajari kita tentang "Abstraksi Platform". Di balik layar, manajemen memori otomatis melalui *Garbage Collection* membebaskan programmer dari beban manual \textit{malloc/free}, namun menuntut pemahaman mendalam tentang siklus hidup objek agar tidak terjadi *Memory Leak*. Pengaruh Java datang dari standarisasi objek dan sistem tipe data yang sangat ketat yang memaksa kedisiplinan massal.

    \item[The Birth of eBay \& Amazon] \textit{Saat pertama kali dibuat}, Amazon hanyalah toko buku \textit{online} yang dijalankan dari garasi Jeff Bezos, dan eBay (awalnya AuctionWeb) adalah situs hobi untuk menjual laser pointer rusak. Di balik layar, keberhasilan mereka adalah tentang skalabilitas database dan sistem transaksi yang aman di web yang masih mentah. Inilah awal mula kapitalisme digital yang akan mengubah perilaku belanja manusia selamanya.
\end{description}

\section{1995: Tahun Ledakan Teknologi Modern}

Tahun yang menentukan wajah komputasi pribadi dan interaktivitas internet selama puluhan tahun mendatang.

\begin{description}
    \item[Windows 95] \textit{Saat pertama kali dibuat}, ini adalah lompatan raksasa dari Windows 3.1 yang berbasis DOS murni. Ia memperkenalkan tombol "Start", taskbar, dan arsitektur 32-bit yang mendukung \textit{Preemptive Multitasking}. Di balik layar, Windows 95 menggunakan mekanisme *Virtual Model Machine* (VMM) untuk menjalankan aplikasi 16-bit dan 32-bit secara berdampingan tanpa saling menghancurkan memori.

    \textit{Pada saat buku ini dibuat}, Windows 95 adalah standar emas bagi sejarah antarmuka desktop. Sebagai Artisan, kita belajar tentang pentingnya "Evolusi Antarmuka". Di balik layar, Microsoft harus melakukan sinkronisasi yang sangat rumit antara kernel baru dengan ribuan perangkat keras lama. Inilah \textit{The Art of Influence}: memandu pengguna ke depan dengan memberikan kenyamanan yang familiar, sambil secara perlahan mengganti pondasi teknis di bawah kaki mereka.

    \item[JavaScript (Mocha/LiveScript)] \textit{Saat pertama kali dibuat} oleh Brendan Eich di Netscape hanya dalam waktu 10 hari yang intens, JavaScript dirancang untuk menjadi "bahasa lem" yang ringan bagi desainer web. Di balik layar, ia menggunakan model *prototypal inheritance* yang sangat berbeda dari Java yang berbasis kelas.

    \textit{Pada saat buku ini dibuat}, JavaScript adalah penguasa mutlak dunia pemrograman, dari frontend hingga backend (Node.js). Sebagai Artisan, JavaScript adalah bukti bahwa "Penempatan Strategis" seringkali mengalahkan kesempurnaan arsitektural. Pengaruh JavaScript datang dari keberadaannya di setiap inci peramban di bumi. Pelajaran Artisan: jadilah solusi terkecil yang ada di setiap tempat.

    \item[The MP3 Format (ISO/IEC 11172-3)] \textit{Saat pertama kali dibuat} oleh Fraunhofer Society, MP3 menggunakan kompresi \textit{psychoacoustic} untuk membuang data suara yang tidak terdengar oleh telinga manusia. Di balik layar, ini adalah keajaiban pemrosesan sinyal digital yang memangkas ukuran file musik hingga 90\% tanpa banyak kehilangan kualitas. Ini adalah awal dari revolusi musik digital yang akan melahirkan Napster dan iPod.
\end{description}

\section{1996: Standar Gaya dan Era Pencarian Awal}

Tahun di mana web mulai terlihat cantik dan tertata, dan benih-benih mesin pencari raksasa mulai tumbuh di laboratorium kampus.

\begin{description}
    \item[CSS (Cascading Style Sheets)] \textit{Saat pertama kali dibuat} oleh Håkon Wium Lie di CERN, CSS bertujuan memisahkan konten (HTML) dari presentasi (Visual). Di balik layar, ia menggunakan algoritma \textit{cascading}—sebuah sistem prioritas yang menghitung bobot spesifisitas selektor untuk menentukan aturan mana yang menang. Ini adalah keajaiban logika pewarisan properti visual.

    \textit{Pada saat buku ini dibuat}, CSS adalah kanvas mutlak bagi desainer web digital. Sebagai Artisan, CSS mengajari kita tentang "Pemisahan Kepentingan" (\textit{Separation of Concerns}). Di balik layar, kemampuan untuk mendefinisikan variabel dan fungsi dalam CSS modern (seperti CSS Variables) adalah evolusi dari prinsip Artisan: jangan ulangi dirimu sendiri (\textit{DRY}). Pengaruh besar dicapai ketika kita bisa mengubah jiwa visual seluruh sistem hanya dengan mengubah satu file gaya pusat.

    \item[Flash (FutureSplash Animator)] \textit{Saat pertama kali dibuat} oleh FutureWave, ini adalah cara untuk membawa animasi vektor yang ringan ke web. Di balik layar, Flash menggunakan representasi matematis untuk bentuk geometris, menyimpannya dalam format biner SWF yang sangat kecil dibandingkan video \textit{bitmap}.

    \textit{Pada saat buku ini dibuat}, Flash telah mati demi standar HTML5 yang terbuka. Namun, ia mengajari kita tentang "Batasan Teknologi yang Memacu Kreativitas". Sebagai Artisan, kita harus menghargai Flash sebagai pionir yang memaksa internet untuk menjadi interaktif dan bersuara jauh sebelum infrastruktur web resmi siap.

    \item[The Apache HTTP Server] \textit{Saat pertama kali dibuat} sebagai sekumpulan \textit{patches} untuk server NCSA oleh kelompok pengembang sukarela, Apache menjadi server paling populer di dunia. Di balik layar, arsitektur modularnya memungkinkan fitur-fitur baru ditambahkan tanpa menyentuh kode inti. Inilah awal mula dominasi *The LAMP Stack*.
\end{description}

\section{1997: Kecerdasan Silikon dan Kebebasan Udara}

Tahun di mana mesin menunjukkan dominasi dalam catur dan konektivitas nirkabel mulai membebaskan manusia dari kabel.

\begin{description}
    \item[Deep Blue vs Garry Kasparov] \textit{Saat pertama kali dibuat} oleh IBM, Deep Blue adalah monster pemrosesan paralel yang dirancang khusus untuk bermain catur. Di balik layar, ia menggunakan chip VLSI kustom yang mampu mengevaluasi 200 juta posisi per detik. Strateginya bukan "berpikir" seperti manusia, tapi menggunakan fungsi evaluasi yang sangat kompleks untuk memindai pohon kemungkinan (\textit{game tree}) lebih dalam daripada siapa pun.

    \textit{Pada saat buku ini dibuat}, kita melihat ini sebagai bukti bahwa "Kekuatan Pemrosesan Terfokus" bisa mengalahkan intuisi manusia terbaik. Sebagai Artisan, Deep Blue mengajari kita tentang pentingnya memahami domain masalah secara mendalam. Di balik layar, keunggulan Deep Blue bukan hanya pada kecepatannya, tapi pada bagaimana para ahli catur "memasukkan" pengetahuan mereka ke dalam algoritma tersebut. Inilah \textit{The Art of Influence}: mentransfer kepakaran manusia ke dalam logika mesin melalui kode yang efisien.

    \item[Wi-Fi (IEEE 802.11)] \textit{Saat pertama kali dibuat}, standar Wi-Fi pertama dirilis dengan kecepatan hanya 2 Mbps pada spektrum \textit{Industrial, Scientific, and Medical} (ISM). Di balik layar, ia menggunakan teknologi \textit{Spread Spectrum} untuk meminimalkan dampak interferensi.

    \textit{Pada saat buku ini dibuat}, WiFi adalah kebutuhan primer bagi peradaban digital. Sebagai Artisan, kita belajar tentang protokol yang "Bertahan dalam Kekacauan". Di balik layar, protokol CSMA/CA memastikan data terkirim dengan benar meskipun banyak perangkat berbicara di frekuensi yang sama. Pengaruh besar dicapai melalui ketahanan sistem yang cerdas.

    \item[SSH-1 (Tatu Ylönen)] \textit{Saat pertama kali dibuat} oleh seorang peneliti di Finlandia setelah serangan \textit{sniffer} di jaringannya, SSH memperkenalkan enkripsi untuk akses \textit{remote}. Di balik layar, ia menggunakan kriptografi kunci publik untuk mengamankan data yang lewat di jaringan publik. Sebagai Artisan, ini adalah alat suci yang memberi kita rasa aman dalam mengontrol mesin di mana pun.
\end{description}

\section{1998: Kelahiran Gerbang Informasi (Google \& Open Source)}

Tahun yang mendefinisikan bagaimana kita mencari informasi dan bagaimana dunia memandang kepemilikan kode.

\begin{description}
    \item[Google (PageRank Algorithm)] \textit{Saat pertama kali dibuat} oleh Larry Page dan Sergey Brin di Stanford, Google bukan sekadar mesin pencari. Di balik layar, ia menggunakan algoritma PageRank yang didasarkan pada aljabar linear dan teori graf. Ia menghitung "vektor kepentingan" bagi setiap halaman web. Sebuah tautan dianggap sebagai "voting" yang kredibilitasnya bergantung pada siapa yang memberikan tautan tersebut.

    \textit{Pada saat buku ini dibuat}, Google adalah penjaga gerbang pengetahuan manusia. Sebagai Artisan, Google mengajari kita tentang "Sistem Evaluasi yang Adil melalui Sains". Pengaruh PageRank datang dari objektivitas matematisnya. Inilah cara Artisan memengaruhi persepsi manusia: dengan membangun algoritma yang mencerminkan otoritas yang jujur secara teknis.

    \item[Open Source Definition (OSI)] \textit{Saat pertama kali dibuat} oleh Bruce Perens dan Eric S. Raymond, istilah ini dirancang untuk membuang stigma politik "Free Software" dan menyajikannya sebagai model pengembangan bisnis yang superior. Di balik layar, ada 10 kriteria ketat yang memastikan aksesibilitas kode bagi semua orang.

    \textit{Pada saat buku ini dibuat}, Open Source adalah jantung dari ekonomi digital. Sebagai Artisan, OSI mengajari kita tentang "Strategi Kompromi untuk Keberlanjutan". Pengaruh kita akan berlipat ganda ketika kita memberikan izin kepada orang lain untuk meretas dan memperbaiki karya kita.

    \item[XML 1.0 (Extensible Markup Language)] \textit{Saat pertama kali dibuat}, XML bertujuan untuk membuat data menjadi \textit{self-describing}. Di balik layar, ia membawa struktur pohon (\textit{tree structure}) yang ketat pada pertukaran data antar sistem yang berbeda. Artisans belajar: kejelasan struktur data adalah dasar dari keandalan sistem berskala besar.
\end{description}

\section{1999: Abad Milenium dan Awal Mula Awan (SaaS)}

Dekade ini ditutup dengan ketakutan antikelimaks Y2K, namun melahirkan benih bisnis yang akan mematikan model instalasi lokal secara perlahan.

\begin{description}
    \item[Salesforce (Birth of SaaS)] \textit{Saat pertama kali dibuat} oleh Marc Benioff, Salesforce menantang status quo dengan arsitektur \textit{multi-tenancy}. Di balik layar, banyak pelanggan menggunakan basis kode yang sama yang dipisahkan secara logis di level database. Inilah efisiensi awan (\textit{Cloud}) yang sesungguhnya.

    \textit{Pada saat buku ini dibuat}, model langganan/SaaS adalah penguasa ekonomi perangkat lunak. Sebagai Artisan, Salesforce mengajari kita tentang "Transformasi Kepemilikan menjadi Akses". Pengaruh besar dicapai ketika kita menghapus beban pemeliharaan teknis dari pundak pengguna dan menanggungnya sendiri melalui layanan yang tak terputus.

    \item[Bluetooth 1.0] \textit{Saat pertama kali dibuat}, ini adalah protokol daya rendah untuk menghubungkan perangkat seluler tanpa kabel. Di balik layar, ia menggunakan \textit{Frequency-Hopping Spread Spectrum} untuk melompat di antara 79 frekuensi radio yang berbeda dalam spektrum 2.4 GHz.

    \textit{Pada saat buku ini dibuat}, Bluetooth adalah lem nirkabel bagi ekosistem pribadi kita. Pelajaran Artisan: "Interoperabilitas Jarak Pendek". Pengaruh tidak selalu harus menjangkau seluruh dunia; terkadang, cukup dengan menghubungkan perangkat di dalam kantong Anda secara elegan.
\end{description}

\section{Atmosfer Era: Dari Menara Gading ke Kamar Kost}

1990-an adalah era di mana teknologi benar-benar menjadi "pop" dan internet menjadi ruang publik baru.

\textit{Saat pertama kali dibuat}, suasana ini melahirkan etika peretas (\textit{hacker ethic}) yang masuk ke arus utama. Ada rasa kegembiraan yang naif bahwa internet akan menghapus batas-batas fisik dan menyatukan kemanusiaan. Teknologi bukan lagi benda dingin di laboratorium; ia adalah tempat kita mengobrol (mIRC), mencari informasi, dan membangun komunitas baru.

\textit{Pada saat buku ini dibuat di tahun 2026}, kita melihat kembali 1990-an sebagai era "Kejujuran Digital". Sebagai Artisan, kita harus menjaga semangat "Kamar Kost" ini—semangat untuk bereksperimen karena ingin tahu, bukan hanya karena ingin mengejar valuasi pasar.

\section{Disiplin Sang Artisan: Navigasi di Lautan Informasi}

Pelajaran terpenting dari dekade ini adalah: Kemampuan untuk "Menyaring" adalah kekuatan super sejati bagi sang Artisan.

\textit{Saat pertama kali dibuat}, ledakan informasi web membuat banyak orang kewalahan. Para Artisan di masa itu harus disiplin untuk tetap fokus pada kualitas data. Mereka belajar menggunakan mesin pencari dengan operator pencarian yang tepat (\textit{Boolean search}) dan membangun pustaka referensi yang kokoh.

\textit{Pada saat buku ini dibuat}, kita memiliki AI yang bisa merangkum segalanya, tapi risiko kehilangan kemampuan berpikir kritis sangat nyata. Disiplin Artisan di tahun 2026 adalah untuk tetap memiliki "Kedaulatan Kognitif". Jangan biarkan algoritma rekomendasi menentukan apa yang Anda pelajari. Masuklah ke balik layar, temukan sumber aslinya, dan bangunlah pemahaman yang mendalam secara mandiri.

\section{Refleksi Dekade: Jaringan yang Menyatukan}

Dekade 1990-an berakhir dengan langkah besar menuju era digital yang sesungguhnya.

\begin{description}
    \item[Warisan Sang Artisan] \textit{Saat pertama kali dibuat}, dekade ini memberikan kita Linux, Windows 95, Java, JavaScript, Google, dan pondasi Open Source. Ini adalah dekade yang menjahit peradaban manusia ke dalam jaringan global.

    \textit{Pada saat buku ini dibuat}, kita menyadari bahwa kita adalah anak-anak dari revolusi 90-an. Sebagai Artisan, kita menghargai era ini dengan cara terus membangun platform yang memberdayakan, bukan membelenggu—melalui kode yang terbuka, standar yang bersih, dan tanggung jawab sosial yang nyata.
\end{description}
