\chapter{The Internet Explosion (1990 -- 1999)}

Jika dekade 1980-an adalah tentang membangun struktur (TCP/IP, GUI, OOP), maka dekade 1990-an adalah tentang menghancurkan dinding.
Ini adalah dekade di mana "Informasi" berhenti menjadi komoditas langka yang dijaga ketat oleh institusi, dan menjadi sungai deras yang mengalir bebas ke setiap rumah tangga.

Dua revolusi besar terjadi secara bersamaan, saling memicu satu sama lain seperti reaksi fisi nuklir:
1.  \textbf{Revolusi Web (The Web Revolution)}: Antarmuka universal untuk mengakses pengetahuan manusia.
2.  \textbf{Revolusi Kode Terbuka (The Open Source Revolution)}: Metode universal untuk membangun perkakas manusia.

Dekade ini dimulai dengan sunyi: seorang mahasiswa Finlandia mengirim email pemalu tentang hobi sistem operasinya.
Dekade ini berakhir dengan ledakan: gelembung ekonomi terbesar dalam sejarah manusia, di mana perusahaan yang tidak memiliki keuntungan bernilai miliaran dolar hanya karena memiliki akhiran ".com".

Bagi Artisan di tahun 2026, era 90-an adalah \textbf{Era Emas Kebebasan}.
Ini adalah masa di mana internet masih liar, belum terjamah oleh algoritma korporat yang memonopoli perhatian. Ini adalah masa di mana seorang remaja di kamar tidurnya bisa meruntuhkan model bisnis raksasa musik global hanya dengan menulis aplikasi berbagi file.
Semangat 90-an adalah semangat \textbf{Distribusi Tanpa Izin} (\textit{Permissionless Distribution}). Jika Anda punya modem dan kode, Anda bisa mengubah dunia. Tidak ada \textit{App Store} yang harus menyetujui aplikasi Anda. Tidak ada \textit{Gatekeeper}. Hanya Anda, server Anda, dan seluruh umat manusia.

\section{1990 -- 1991: Kelahiran Web dan Raja yang Tidak Sengaja}

Pada Natal 1990, di CERN (Organisasi Riset Nuklir Eropa), Tim Berners-Lee menyalakan server web pertama di dunia pada komputer NeXT-nya.
Alamatnya: \texttt{info.cern.ch}.
Halaman web pertama itu sangat sederhana. Teks hitam di latar belakang putih. Tidak ada gambar. Tidak ada video. Hanya penjelasan tentang apa itu World Wide Web.
Namun, implikasinya sangat dalam. Tim Berners-Lee memberikan tiga hadiah kepada dunia: HTML (bahasa), HTTP (protokol), dan URL (alamat). Dan yang paling penting: dia memberikannya secara \textbf{Gratis}. CERN melepaskan teknologi Web ke domain publik pada tahun 1993, memastikan bahwa tidak ada satu perusahaan pun yang bisa memilikinya.

Sementara itu, di Helsinki, Finlandia, pada 25 Agustus 1991, \textbf{Linus Torvalds} mengirim pesan bersejarah ke grup berita \texttt{comp.os.minix}:
\begin{quote}
"I'm doing a (free) operating system (just a hobby, won't be big and professional like gnu) for 386(486) AT clones..."
\end{quote}
Linus tidak berniat menghancurkan Microsoft. Dia hanya frustrasi karena sistem operasi UNIX komersial terlalu mahal untuk mahasiswa, dan MINIX (sistem operasi pendidikan) terlalu terbatas.
Dia ingin membuat terminal emulator untuk mengakses komputer universitas dari rumah.
Proyek "hobi" itu dinamai \textbf{Linux}.

Linus melakukan sesuatu yang tidak lazim: Dia merilis kodenya ke internet \textit{sebelum} kodenya selesai. Dia mengundang orang lain untuk memperbaikinya.
"Release early, release often," menjadi mantranya.
Ratusan, lalu ribuan programmer dari seluruh dunia mulai mengirimkan perbaikan (\textit{patch}).
Mereka memperbaiki driver hard disk. Mereka menambahkan dukungan jaringan. Mereka memportingnya ke arsitektur lain.
Tanpa disadari, Linus telah menemukan \textbf{Hukum Linus}: \textit{"Given enough eyeballs, all bugs are shallow."} (Dengan cukup banyak mata yang melihat, semua kutu akan terlihat dangkal).

Model pengembangan kolaboratif ini—di mana ribuan orang asing bekerja sama membangun sesuatu yang rumit tanpa bayaran dan tanpa manajemen pusat—adalah antitesis dari model "Katedral" perusahaan besar (seperti Microsoft) di mana kode dibuat oleh tim elit tertutup.
Linux membuktikan bahwa \textbf{Bazaar} (pasar terbuka yang kacau) bisa menghasilkan perangkat lunak yang lebih stabil, lebih cepat, dan lebih aman daripada Katedral.

\section{1993 -- 1994: Mosaic dan Awal Mula E-Commerce}

Hingga tahun 1993, Web masih merupakan tempat yang sunyi bagi para akademisi. Teks saja. Tidak ada gambar.
Lalu datanglah \textbf{Mosaic}.
Marc Andreessen dan Eric Bina di NCSA (National Center for Supercomputing Applications) merilis peramban (\textit{browser}) Mosaic.
Fitur pembunuhnya? \textbf{Tag \texttt{<img>}}.
Ya, kemampuan untuk menampilkan gambar \textit{di dalam} halaman teks (inline image).
Tiba-tiba, Web menjadi majalah berwarna. Web menjadi visual.
Mosaic bisa diunduh gratis dan mudah diinstal di Windows.
Dalam semalam, lalu lintas Web meledak. Orang-orang biasa mulai masuk. "Surfing the Web" menjadi istilah rumah tangga.

Pada tahun 1994, Andreessen mendirikan \textbf{Netscape Communications}. Peramban mereka, \textbf{Netscape Navigator}, menjadi standar de-facto untuk mengakses internet. Mereka menguasai 90\% pangsa pasar.

Sementara itu, Jeff Bezos, seorang eksekutif Wall Street, membaca statistik bahwa penggunaan web tumbuh 2.300\% per tahun. Dia berhenti dari pekerjaannya, pindah ke Seattle, dan mendirikan \textbf{Amazon.com} di garasinya.
Mengapa buku? Karena ada jutaan judul buku, lebih banyak daripada yang bisa ditampung oleh toko fisik manapun, dan buku mudah dikirim.
Bezos tidak hanya membangun toko; dia membangun \textbf{Teknologi Logistik}.
Sistem rekomendasi ("Orang yang membeli ini juga membeli..."), ulasan pengguna, dan "1-Click Ordering" adalah inovasi perangkat lunak yang mengubah cara manusia berdagang.

Di sisi lain, Pierre Omidyar mendirikan \textbf{AuctionWeb} (kemudian menjadi \textbf{eBay}) sebagai hobi untuk membantu pacarnya mengoleksi dispenser permen Pez.
eBay membuktikan sesuatu yang mengejutkan: \textbf{Kepercayaan Digital}.
Orang asing mau mengirim uang ke orang asing lain untuk barang yang belum mereka lihat, hanya berdasarkan sistem reputasi bintang (\textit{Feedback Score}).
Ini adalah revolusi sosial. Database reputasi eBay menciptakan kepercayaan di tempat yang sebelumnya tidak ada (\textit{Trustless Environment}).

\section{1995: Tahun Ledakan Besar}

Tahun 1995 mungkin adalah tahun tunggal paling penting dalam sejarah internet. Tiga bahasa pemrograman lahir, satu sistem operasi dominan dirilis, dan "Gold Rush" dimulai.

\textbf{1. Java (Sun Microsystems)}
James Gosling menciptakan Java dengan slogan \textit{"Write Once, Run Anywhere"}.
Idenya adalah membuat kode yang dikompilasi menjadi \textit{Bytecode} universal, yang kemudian dijalankan oleh \textit{Java Virtual Machine} (JVM) di mesin apa pun (Windows, Mac, Linux, bahkan pemanggang roti).
Java membawa keamanan (\textit{Memory Safety}) dengan \textit{Garbage Collection} otomatis, membebaskan programmer dari mimpi buruk manajemen memori C++.
Netscape segera memasukkan Java ke dalam browser mereka melalui \textit{Applet}. Tiba-tiba, web bisa menjalankan program nyata, bukan hanya teks statis.

\textbf{2. PHP (Rasmus Lerdorf)}
Sementara Java adalah bahasa bagi insinyur berseragam, PHP adalah bahasa bagi rakyat jelata.
Rasmus Lerdorf membuat sekumpulan skrip Perl/C sederhana untuk melacak pengunjung resume onlinenya. Ia menyebutnya \textit{Personal Home Page Tools}.
PHP jelek. Sintaksnya tidak konsisten. Penamaan fungsinya kacau.
Tapi PHP memiliki satu keunggulan mematikan: \textbf{Mudah}.
Anda cukup menyisipkan tag \texttt{<?php ... ?>} di tengah HTML Anda, dan \textit{voila}, halaman web Anda menjadi dinamis. Tidak perlu kompilasi. Tidak perlu konfigurasi server yang rumit.
PHP mendemokratisasi backend. Facebook, Wikipedia, dan WordPress semuanya dibangun di atas kekacauan yang indah ini.

\textbf{3. JavaScript (Brendan Eich)}
Netscape membutuhkan bahasa skrip ringan untuk peramban mereka. Sesuatu untuk desainer, bukan insinyur sistem.
Brendan Eich ditugaskan membuatnya. Dia hanya punya waktu \textbf{10 hari} sebelum rilis beta Netscape.
Dia mengambil sintaks C (agar terlihat familier), sistem objek \textit{Self} (prototypal inheritance), dan fungsi kelas satu \textit{Scheme}.
Ia menamainya \textbf{JavaScript} (marketing stunt untuk membonceng popularitas Java, padahal tidak ada hubungannya).
Hasilnya adalah bahasa yang aneh, penuh perilaku ganjil (seperti \texttt{[] + [] = ""}), tapi sangat fleksibel.
JavaScript menjadi satu-satunya bahasa yang dimengerti oleh setiap peramban di dunia. Ia menjadi \textit{"Bahasa Web"}.

\textbf{4. Windows 95}
Microsoft merilis sistem operasi yang mengubah PC dari alat kerja menjadi alat gaya hidup. Tombol "Start", Taskbar, dan dukungan \textit{Plug and Play} membuat komputer jauh lebih ramah. Puncaknya: Bundling \textbf{Internet Explorer} gratis. Ini adalah awal dari \textbf{Perang Browser} pertama yang akan membawa Microsoft ke pengadilan antimonopoli.

\section{1996 -- 1998: Perang Browser dan Portal}

Netscape vs. Microsoft.
Ini adalah perang total. Microsoft, yang terlambat menyadari potensi internet, menggunakan kekuatan monopolinya di desktop untuk menghancurkan Netscape.
Mereka memberikan Internet Explorer (IE) secara gratis (Netscape memungut biaya). Mereka mengintegrasikan IE begitu dalam ke Windows sehingga sulit dihapus.
Secara teknis, IE (versi 4.0 ke atas) sebenarnya sangat bagus. Mereka memperkenalkan \textbf{AJAX} (melalui ActiveX/XMLHTTP) jauh sebelum istilah itu ada.
Namun, taktik "Embrace, Extend, Extinguish" mereka membuat komunitas web marah.

Di tengah perang ini, muncul fenomena \textbf{Portal} (Yahoo!, AOL, MSN).
Idenya adalah menjadi "Halaman Depan Internet". Direktori web yang dikurasi manusia.
Yahoo! dimulai sebagai daftar tautan favorit Jerry Yang dan David Filo.
Namun, web tumbuh terlalu cepat untuk dikurasi oleh manusia. Direktori menjadi usang saat diterbitkan.

Solusinya datang dari Stanford pada tahun 1998.
Larry Page dan Sergey Brin merilis \textbf{Google}.
Mereka tidak menggunakan kurasi manusia. Mereka menggunakan matematika.
\textbf{PageRank}: Algoritma yang menilai otoritas sebuah halaman berdasarkan berapa banyak halaman lain yang menaut padanya. Sebuah tautan dianggap sebagai "suara" (\textit{vote}). Halaman penting mendapat suara lebih berat.
Kotak pencarian putih bersih Google adalah antitesis dari Portal yang penuh iklan dan berita selebriti.
Google menang karena mereka menghormati satu hal: \textbf{Intensi Pengguna}. Orang datang untuk \textit{pergi} ke tempat lain, bukan untuk \textit{tinggal} di portal.

\section{1999: Napster dan Puncak Gelembung}

Tahun terakhir abad ini ditandai dengan pemberontakan paling ikonik.
\textbf{Shawn Fanning}, remaja 19 tahun, merilis \textbf{Napster}.
Musik MP3 sudah ada (ditemukan oleh Fraunhofer Institute), tapi sulit ditemukan. Napster menggabungkan MP3 dengan teknologi \textbf{Peer-to-Peer (P2P)}.
Alih-alih mengunduh dari server pusat, pengguna Napster mengunduh langsung dari hard disk pengguna lain.
Dalam semalam, koleksi musik terbesar di dunia terbentuk tanpa satu sen pun dibayarkan ke label rekaman.
60 juta pengguna.
Industri musik panik. Mereka menuntut Napster dan mematikannya pada tahun 2001.
Tapi mereka tidak bisa mematikan idenya. Napster membuktikan bahwa \textbf{Informasi Ingin Bebas}. Begitu sesuatu didigitalkan, biaya distribusinya menjadi nol. Model bisnis yang bergantung pada kelangkaan buatan (\textit{artificial scarcity}) sudah mati.

Di pasar saham, kegilaan mencapai puncaknya.
Perusahaan seperti \textbf{Pets.com} (menjual makanan anjing online) melakukan IPO, sahamnya naik ribuan persen, padahal mereka merugi pada setiap penjualan karena biaya pengiriman.
"Ini ekonomi baru!" teriak para investor. "Keuntungan tidak penting; pertumbuhan yang penting!"
Insinyur sistem memperingatkan tentang \textbf{Y2K Bug} (Masalah Tahun 2000). Kode lama hanya menyimpan tahun sebagai 2 digit ('99), sehingga tahun 2000 akan terbaca sebagai 1900, berpotensi mengacaukan bank dan penerbangan.
Miliaran dolar dihabiskan untuk memperbaiki kode COBOL tua.
Ketika jam berdentang tengah malam 1 Januari 2000, dunia tidak berakhir. Pesawat tidak jatuh. ATM tetap bekerja.
Banyak yang menyebut Y2K sebagai histeria berlebihan (\textit{hoax}). Tapi bagi Artisan, itu adalah kemenangan diam-diam: Bencana dicegah karena kerja keras ribuan insinyur yang memperbaiki fondasi busuk tepat pada waktunya.

\section{Refleksi Dekade: Jaring yang Menyatukan Manusia}

Dekade 90-an adalah masa remaja umat manusia digital.
Penuh energi, penuh pemberontakan, sedikit ceroboh, tapi sangat optimis.

Kita belajar bahwa \textbf{Keterbukaan Mengalahkan Ketertutupan}.
Protokol terbuka (HTTP, HTML, TCP/IP) mengalahkan protokol tertutup (MSN, AOL).
Sistem operasi terbuka (Linux) mulai mengguncang server Unix proprieter (Solaris, AIX).
Browser terbuka (Netscape/Mozilla) meletakkan dasar bagi web modern.

Warisan 90-an bagi Artisan 2026 adalah \textbf{Semangat Hacker}.
Bahwa Anda tidak perlu izin untuk membangun sesuatu yang hebat.
Bahwa kode yang jelek tapi berguna (seperti PHP awal atau web HTML mentah) lebih baik daripada kode yang indah tapi tidak ada yang pakai.
Bahwa kekuatan terbesar bukanlah pada server pusat, tetapi pada \textbf{Ujung Jaringan} (\textit{The Edge})—pada Anda, pada saya, pada setiap individu yang terhubung.
