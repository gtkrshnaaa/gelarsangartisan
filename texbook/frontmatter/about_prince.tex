\chapter*{Manifesto Sang Pangeran}

\begin{center}
    \textit{"Gelar bukanlah apa yang diberikan orang lain, melainkan apa yang kita tempa sendiri."}
\end{center}

\vspace{1cm}

\noindent\textbf{Titled Artisans Prince.}

Ini bukan gelar kerajaan, juga bukan jabatan perusahaan.
Ini adalah manifestasi dari sebuah perjalanan panjang menembus hutan belantara teknologi.

Saya adalah seorang Artisan yang telah lama mengembara.
Saya telah melihat \textit{framework} datang dan pergi seperti musim.
Saya telah melihat startup bernilai miliaran dolar runtuh karena utang teknis yang tak terbayar.
Saya telah melihat insinyur-insinyur brilian terbakar habis (\textit{burnout}) karena mengejar \textit{hype} yang tak berujung.

Buku ini, \textit{Gelar Sang Artisan}, lahir dari keinginan sederhana:
Untuk mewariskan peta jalan bagi mereka yang tersesat.
Untuk memberikan pijakan yang kokoh di tengah badai informasi.
Untuk mengingatkan kembali bahwa di balik layar monitor yang dingin, ada manusia yang bernapas, bermimpi, dan mencipta.

Saya menulis ini bukan sebagai guru yang tahu segalanya, tapi sebagai rekan seperjalanan yang telah lebih dulu tersandung batu.
Setiap baris kode dalam buku ini adalah opini. Opini yang ditempa oleh kegagalan, dikeraskan oleh \textit{bug} produksi, dan dipoles oleh solusi yang elegan.

Selamat datang di dunia saya.
Dunia di mana teknologi tunduk pada visi manusia, bukan sebaliknya.

\vspace{2cm}

\begin{flushright}
  \textbf{Gilang Teja Krishna} \\
  \textit{Titled Artisans Prince}
\end{flushright}
