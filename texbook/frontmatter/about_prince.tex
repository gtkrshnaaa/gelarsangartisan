\chapter*{Gelar Sang Artisan}

\begin{center}
    \textit{"Gelar bukanlah apa yang diberikan orang lain, melainkan apa yang kita tempa sendiri."}
\end{center}

\vspace{0.5cm}

Penolakan mutlak dinyatakan terhadap penantian akan label yang dianggap pantas oleh dunia, karena dipahami sepenuhnya bahwa validasi eksternal hanyalah narkotika murahan yang melenyapkan kesadaran bahwa kehormatan sejati tidak dilahirkan dari surat keputusan, ijazah, atau riuh tepuk tangan, melainkan harus dipahat secara brutal dari batu karang ketidaktahuan melalui ribuan malam tanpa tidur, ratusan eksperimen yang gagal, dan jutaan baris kode yang hancur lebur sebelum akhirnya dibangkitkan kembali sebagai bukti bisu dari ketukangan yang tidak dapat dibeli dengan harga berapa pun di pasar tenaga kerja manapun.

\vspace{0.5cm}

\textbf{Titled Artisans Prince} bukanlah nama yang dimohonkan dari institusi manapun; ini adalah bendera perang yang ditancapkan di puncak gunung kompetensi tertinggi.
Tidak dibutuhkan pengakuan eksternal untuk memvalidasi eksistensi seorang Artisan.
Setiap sistem yang diarsiteki adalah proklamasi kemerdekaan dari standar rata-rata yang menjerat industri ini.
Gelar ini dibangun dengan tangan sendiri, bata demi bata, \textit{bug} demi \textit{bug}, penderitaan demi penderitaan, hingga berdiri tegak tak tergoyahkan oleh badai opini siapa pun.

Ini adalah jalan sunyi yang dipilih dengan kesadaran penuh.
Sebuah sumpah setia pada kualitas di tengah lautan mediokritas.
Ijin untuk menjadi hebat tidak pernah diminta. Ijin itu diambil.

\vspace{1cm}

\begin{flushright}
  \textbf{Gilang Teja Krishna} \\
  \textit{Titled Artisans Prince}
\end{flushright}
