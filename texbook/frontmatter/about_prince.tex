\chapter*{Gelar Sang Artisan}

\begin{center}
    \textit{"Teknologi tidak peduli dengan estetikamu. Ia hanya peduli apakah sistemmu bertahan atau hancur."}
\end{center}

\vspace{1cm}

Gelar ini bukan mahkota. Ini adalah bekas luka.

Dunia ini tidak butuh kode yang puitis. Dunia butuh sistem yang tetap berjalan saat \textit{traffic} melonjak ribuan persen. Dunia butuh \textit{database} yang tidak korup saat listrik mati mendadak.
Kita tidak dibayar untuk bermimpi indah. Kita dibayar untuk menyelesaikan masalah yang orang lain Bahkan takut untuk menyentuhnya.

Menjadi Artisan bukan tentang gaya. Itu tentang bertahan hidup.
Tentang ratusan jam menatap layar terminal yang hitam.
Tentang kopi dingin dan debat panas di \textit{code review}.
Tentang keberanian menghapus kode sendiri demi solusi yang lebih waras.

Buku ini bukan dongeng motivasi. Ini adalah manual lapangan dari garis depan.
Berisi strategi tempur melawan kompleksitas, \textit{hype} yang menyesatkan, dan ego kita sendiri.

Selamat datang di realitas.
Dingin. Keras. Tanpa ampun.
Di sinilah kita bekerja.

\vspace{2cm}

\begin{flushright}
  \textbf{Gilang Teja Krishna} \\
  \textit{Titled Artisans Prince}
\end{flushright}
