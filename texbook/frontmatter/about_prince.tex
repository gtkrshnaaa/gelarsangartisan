\chapter*{Gelar Sang Artisan}

\begin{center}
    \textit{"Gelar bukanlah apa yang diberikan orang lain, melainkan apa yang kita tempa sendiri."}
\end{center}

\vspace{0.5cm}

Dipilihnya jalan sunyi ini bukanlah sebuah penarikan diri dari realitas, melainkan sebuah manuver strategis untuk mengamati keriuhan dunia dari ketinggian yang tidak terjangkau oleh standar rata-rata, karena disadari sepenuhnya bahwa kompetensi yang melampaui zamannya seringkali terlihat seperti arogansi bagi mata yang belum terlatih, padahal ia hanyalah konsekuensi logis dari ribuan jam dedikasi yang diinvestasikan dalam diam untuk menajamkan intuisi hingga setajam silet, mengubah setiap tantangan teknis menjadi kanvas di mana dominasi intelektual ditegakkan tanpa perlu sepatah kata pun terucap untuk membenarkannya.

\vspace{0.5cm}

\textbf{Titled Artisans Prince} bukanlah gelar yang dipinjam dari otoritas mana pun. Ia adalah sebuah anomali yang sengaja diciptakan untuk mengganggu kenyamanan mediokritas.
Tidak ada spanduk yang dikibarkan. Tidak ada pidato yang dibacakan.
Hanya ada standar kualitas yang begitu tinggi hingga memaksa sekelilingnya untuk tunduk atau tersingkir secara alami.
Ini adalah seni perang tanpa pertumpahan darah; sebuah penaklukan absolut melalui presisi yang tak terbantahkan.

Dunia tidak perlu tahu siapa yang mengendalikan arus di balik layar.
Cukup nikmati karyanya. Patuhi standarnya.
Sisanya adalah sejarah yang ditulis oleh pemenang dalam keheningan.

\vspace{1cm}

\begin{flushright}
  \textbf{Gilang Teja Krishna} \\
  \textit{Titled Artisans Prince}
\end{flushright}
