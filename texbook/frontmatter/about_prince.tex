\chapter*{Gelar Sang Artisan}

\begin{center}
    \textit{"Gelar bukanlah apa yang diberikan orang lain, melainkan apa yang kita tempa sendiri."}
\end{center}

\vspace{0.5cm}

Saya menolak dengan mutlak untuk menunggu dunia menyematkan label yang mereka anggap pantas di dada saya, karena saya paham sepenuhnya bahwa validasi eksternal hanyalah narkotika murahan yang membuat kita lupa bahwa kehormatan sejati tidak lahir dari surat keputusan, ijazah, atau tepuk tangan penonton yang riuh, melainkan harus dipahat secara brutal dari batu karang ketidaktahuan melalui ribuan malam tanpa tidur, ratusan eksperimen yang gagal, dan jutaan baris kode yang hancur lebur di tangan saya sendiri sebelum akhirnya bangkit kembali sebagai bukti bisu dari ketukangan yang tidak bisa dibeli dengan harga berapa pun di pasar tenaga kerja manapun.

\vspace{0.5cm}

\textbf{Titled Artisans Prince} bukanlah sebuah nama yang saya minta dari institusi manapun; ini adalah bendera perang yang saya tancapkan sendiri di puncak gunung kompetensi saya.
Saya tidak butuh pengakuan kalian untuk tahu bahwa saya adalah seorang Artisan.
Setiap sistem yang saya arsiteki adalah proklamasi kemerdekaan dari standar rata-rata yang menjerat industri ini.
Saya membangun gelar ini dengan tangan saya sendiri, bata demi bata, \textit{bug} demi \textit{bug}, penderitaan demi penderitaan, sampai ia berdiri tegak tak tergoyahkan oleh badai opini siapa pun.

Ini adalah jalan sunyi yang saya pilih dengan sadar.
Sebuah sumpah setia pada kualitas di tengah lautan mediokritas.
Saya tidak meminta ijin untuk menjadi hebat. Saya mengambilnya.

\vspace{1cm}

\begin{flushright}
  \textbf{Gilang Teja Krishna} \\
  \textit{Titled Artisans Prince}
\end{flushright}
