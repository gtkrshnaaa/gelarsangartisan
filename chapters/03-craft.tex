\chapter{Living the Tech}

\chapter{Living the Tech}

Teknologi bukan hanya untuk diselesaikan di kantor. Ini adalah bagian dari kehidupan sehari-hari. Dari otomatisasi rumah hingga efisiensi kerja. Bagi seorang \textit{artisan}, teknologi adalah perpanjangan dari efisiensi berpikir kita.

\section{Otomatisasi sebagai Gaya Hidup}

Saya tidak suka mengerjakan hal yang sama dua kali. Jika sebuah tugas memakan waktu lebih dari 5 menit dan akan berulang besok, saya akan menulis skrip untuknya.

\begin{itemize}
    \item \textbf{Dotfiles}: Seluruh lingkungan kerja saya adalah sebuah kode. Dengan satu perintah \textit{git clone}, saya bisa bekerja di mesin mana pun dengan kenyamanan yang sama.
    \item \textbf{Local AI Agents}: Di tahun 2026, saya menjalankan model AI lokal untuk membantu menyortir email, menjadwalkan pertemuan, dan bahkan memberikan \textit{code review} awal sebelum saya melakukan komit.
\end{itemize}

\section{The Artisan's Cheatsheet}

Berikut adalah rangkuman cepat alat dan filosofi yang saya gunakan setiap hari:

\subsection{Development Tools}
\begin{itemize}
    \item \textbf{Editor}: Neovim atau VS Code dengan \textit{custom configuration}.
    \item \textbf{Terminal}: Alacritty + Tmux untuk manajemen sesi.
    \item \textbf{Shell}: Zsh dengan \textit{p10k} untuk informasi visual yang cepat.
\end{itemize}

\subsection{Productivity Commands}
\begin{itemize}
    \item \texttt{git standup}: Skrip \textit{custom} untuk melihat apa yang saya kerjakan kemarin.
    \item \texttt{deploy-all}: Otomatisasi CI/CD dari terminal.
\end{itemize}

\subsection{Mindset Ritual}
\begin{itemize}
    \item \textit{Read Code Every Day}: Membaca kode orang lain sama pentingnya dengan menulis kode sendiri.
    \item \textit{Delete More, Build Less}: Menghapus 100 baris kode yang redundan lebih memuaskan daripada menulis 100 baris kode baru.
\end{itemize}

\section{Penutup: Meninggalkan Jejak}

Gelar \textit{Artisan} bukan diberikan oleh institusi, tapi oleh dedikasi pada kualitas. Buku ini adalah bukti perjalanan saya, dan saya harap ini menjadi kompas bagi kamu yang ingin menempuh jalan yang sama.
